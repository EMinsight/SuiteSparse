\newpage
%-------------------------------------------------------------------------------
\subsection{The GxB\_Container object and its methods}
%-------------------------------------------------------------------------------
\label{container}

A new set of \verb'load/unload' methods are introduced in GraphBLAS v10 to move
data between a \verb'GrB_Matrix' or \verb'GrB_Vector' and a new
\verb'GxB_Container' object.  This object is non-opaque but contains opaque
objects.  Its primary components are five dense \verb'GrB_Vectors' that hold
the contents of the matrix/vector.  The data in these dense vectors can then be
loaded/unloaded via \verb'GxB_Vector_load' and \verb'GxB_Vector_unload'.

Moving data from a \verb'GrB_Matrix' into user-visible C arrays is a two-step
process.  The data is first moved into a \verb'GxB_Container' using \newline
\verb'GxB_unload_Matrix_into_Container', and then from the Container into C
arrays with \verb'GxB_Vector_unload'.  Moving data in the opposite direction is
also a two-step process: first load the C array into a \verb'GrB_Vector'
component of a \verb'GxB_Container' with \verb'GxB_Vector_load', and then from
the \verb'GxB_Container' into a \verb'GrB_Matrix' using the
\verb'GxB_load_Matrix_from_Container' method.

The following methods are available.  The first two do not use the
Container object, but instead move data to/from a dense \verb'GrB_Vector':

\begin{itemize}

\item \verb'GxB_Vector_load':  this method moves data in O(1) time from a
    user-visible C array into a \verb'GrB_Vector'.  The vector length and type
    are revised to match the new data from the C array.  Ownership is normally
    transferred to the \verb'GrB_Vector', but this can be revised with a
    \verb'handling' parameter.  The C array is passed in as a \verb'void *'
    pointer, and its type is indicated by a \verb'GrB_Type' parameter.  See
    Section~\ref{vector_load} for details.

\item \verb'GxB_Vector_unload': this method moves data in O(1) time from a
    \verb'GrB_Vector' into a user-visible C array.  The length of the
    \verb'GrB_Vector' is reduced to zero, to denote that it no longer holds any
    content.  The vector must be dense; it must have the same number of entries
    as its size (that is \verb'GrB_Vector_nvals' and \verb'GrB_Vector_size'
    must return the same value).  The C array is returned as a \verb'void *'
    pointer, and its type is indicated by a \verb'GrB_Type' parameter.  See
    Section~\ref{vector_unload} for details.

\end{itemize}

The next six methods rely on the \verb'GxB_Container' object:

\begin{itemize}
\item \verb'GxB_Container_new': creates a container
    (see Section~\ref{container_new}).

\item \verb'GxB_Container_free': frees a container
    (see Section~\ref{container_free}).

\item \verb'GxB_load_Matrix_from_Container': moves all of the data from a
    \verb'GxB_Container' into a \verb'GrB_Matrix' in O(1) time
    (see Section~\ref{load_matrix_from_container}).

\item \verb'GxB_load_Vector_from_Container': moves all of the data from a
    \verb'GxB_Container' into a \verb'GrB_Vector' in O(1) time
    (see Section~\ref{load_vector_from_container}).

\item \verb'GxB_unload_Matrix_into_Container': moves all of the data from
    a \verb'GrB_Matrix' into a \verb'GxB_Container' in O(1) time
    (see Section~\ref{unload_matrix_into_container}).

\item \verb'GxB_unload_Vector_into_Container': moves all of the data from
    a \verb'GrB_Vector' into a \verb'GxB_Container' in O(1) time
    (see Section~\ref{unload_vector_into_container}).

\end{itemize}

%-------------------------------------------------------------------------------
\subsubsection{{\sf GxB\_Vector\_load:} load data into a vector}
%-------------------------------------------------------------------------------
\label{vector_load}

\begin{mdframed}[userdefinedwidth=6in]
{\footnotesize
\begin{verbatim}
GrB_Info GxB_Vector_load
(
    // input/output:
    GrB_Vector V,           // vector to load from the C array X
    void **X,               // numerical array to load into V
    // input:
    GrB_Type type,          // type of X
    uint64_t n,             // # of entries in X
    uint64_t X_size,        // size of X in bytes (at least n*(sizeof the type))
    int handling,           // GrB_DEFAULT (0): transfer ownership to GraphBLAS
                            // GxB_IS_READONLY: X treated as read-only;
                            //      ownership kept by the user application
    const GrB_Descriptor desc   // currently unused; for future expansion
) ;
\end{verbatim}
} \end{mdframed}

\verb'GxB_Vector_load' loads data from a C array into a \verb'GrB_Vector' in
O(1) time.

On input, the \verb'GrB_Vector V' must already exist, but its content (type,
size, and entries) are ignored.  On output, any prior content of \verb'V' is
freed, and its data is replaced with the C array \verb'X' of length \verb'n'
entries, whose type is given by the \verb'type' parameter.  The size of
\verb'V' becomes \verb'n', and its type is changed to match the \verb'type'
parameter.

After this method returns, \verb'GrB_Vector_size' and \verb'GrB_Vector_nvals' 
will both return \verb'n'.  That is, the vector \verb'V' is a dense vector.
It is held in the \verb'GxB_FULL' data format, in \verb'GrB_COLMAJOR'
orientation.

The size in bytes of \verb'X' is also provided on input as \verb'X_size'; this
must be at least \verb'n' times the size of the given \verb'type', but it can
be larger.  This size is maintained and returned to the user application by
\verb'GxB_Vector_unload'.  The error code \verb'GrB_INVALID_VALUE' is returned
if \verb'X_size' is too small.

The \verb'handling' parameter provides control over the ownership of the array
\verb'X'.  By default, ownership of \verb'X' is handed over to the
\verb'GrB_Vector V'.  In this case, the vector \verb'V' acts as a normal
GraphBLAS vector.  It can be modified or freed as usual.  Freeing \verb'V' with
\verb'GrB_Vector_free' will also free \verb'X'.  The array \verb'X' is returned
as \verb'NULL' to denote this change of ownership.

If the \verb'handling' parameter is \verb'GxB_IS_READONLY', then \verb'X' is
still owned by the user application.  It remains non-\verb'NULL' when this
method returns.  The resulting vector \verb'V' can be used as an input to any
GraphBLAS method, but it cannot be modified (except that it can be freed).
If a call is made that attempts to modify a matrix with read-only components,
an error is returned (\verb'GxB_OUTPUT_IS_READONLY').
Freeing the vector \verb'V' does not modify \verb'X', however.  It simply
frees the rest of the object \verb'V'.

Note that this method does not rely on the \verb'GxB_Container' object.
Instead, it loads a C array \verb'X' directly into a dense \verb'GrB_Vector V'.

%-------------------------------------------------------------------------------
\subsubsection{{\sf GxB\_Vector\_unload:} unload data from a vector}
%-------------------------------------------------------------------------------
\label{vector_unload}

\begin{mdframed}[userdefinedwidth=6in]
{\footnotesize
\begin{verbatim}
GrB_Info GxB_Vector_unload
(
    // input/output:
    GrB_Vector V,           // vector to unload
    void **X,               // numerical array to unload from V
    // output:
    GrB_Type *type,         // type of X
    uint64_t *n,            // # of entries in X
    uint64_t *X_size,       // size of X in bytes (at least n*(sizeof the type))
    int *handling,          // see GxB_Vector_load
    const GrB_Descriptor desc   // currently unused; for future expansion
) ;
\end{verbatim}
} \end{mdframed}

\verb'GxB_Vector_unload' unloads data from \verb'GrB_Vector' into a C array in
O(1) time (unless the vector has pending work that must be finished first).

On input, the vector \verb'V' must have all possible entries present (that is,
\verb'GrB_Vector_nvals' and \verb'GrB_Vector_size' must return the same value).
The vector can be in any internal data format; it does not have to be in the
\verb'GxB_FULL' format on input, but it must be in \verb'GrB_COLMAJOR'
orientation.  If any entries are missing, the unload is not performed and
\verb'GrB_INVALID_OBJECT' is returned.

On output, the size of \verb'V' is reduced to zero, and it holds no entries but
its type is unchanged.  The array \verb'X' is returned to the caller with the
entries from the vector.  The type of \verb'X' is given by the \verb'type'
parameter.  The number of entries in \verb'V' is returned as \verb'n'.
The size of \verb'X' in bytes is returned as \verb'X_size'.

The \verb'handling' parameter informs the user application of the ownership of
the array \verb'X'.  If it was created by GraphBLAS, or passed into GraphBLAS
via \verb'GxB_Vector_load' with \verb'handling' set to \verb'GrB_DEFAULT', then
this is returned to the user as handling set to \verb'GrB_DEFAULT'.  This
denotes that ownership of the array \verb'X' has been transfered from GraphBLAS
to the user application.  The user application is now responsible for freeing
the array \verb'X'.

If the \verb'handling' parameter is returned as \verb'GxB_IS_READONLY', then
this array \verb'X' was originally passed to GraphBLAS via
\verb'GxB_Vector_load' with a \verb'handling' parameter of
\verb'GxB_IS_READONLY'.  The ownership of the array \verb'X' is not changed; it
remains owned by the user application.  The user application is still
responsible for freeing the array \verb'X', but caution must be observed so
that it is not freed twice.

Note that this method does not rely on the \verb'GxB_Container' object.
Instead, it unloads a dense \verb'GrB_Vector' \verb'V' directly into a
C array \verb'X'.

%-------------------------------------------------------------------------------
\subsubsection{{\sf GxB\_Container\_new:} create a container}
%-------------------------------------------------------------------------------
\label{container_new}

\begin{mdframed}[userdefinedwidth=6in]
{\footnotesize
\begin{verbatim}
GrB_Info GxB_Container_new (GxB_Container *Container) ;
\end{verbatim}
} \end{mdframed}

\verb'GxB_Container_new' creates a new container.  It has the following
definition (extra space for future expansion excluded for clarity):

\begin{mdframed}[userdefinedwidth=6in]
{\footnotesize
\begin{verbatim}
struct GxB_Container_struct
{
    uint64_t nrows, ncols ;
    int64_t nrows_nonempty, ncols_nonempty ;
    uint64_t nvals ;
    int32_t format ;      // GxB_HYPERSPARSE, GxB_SPARSE, GxB_BITMAP, or GxB_FULL
    int32_t orientation ; // GrB_ROWMAJOR or GrB_COLMAJOR
    GrB_Vector p, h, b, i, x ;  // dense vector components
    GrB_Matrix Y ;
    bool iso, jumbled ;
} ;
typedef struct GxB_Container_struct *GxB_Container ; \end{verbatim}
} \end{mdframed}

The \verb'Container' holds all of the data from a single GraphBLAS matrix or
vector, with any data type and any data format.  It contains extra space for
future data formats (not shown above).  Each scalar component is described
below:

\begin{itemize}
\item \verb'nrows': the number of rows of a matrix, or the size of a vector.
\item \verb'ncols': the number of columns of a matrix, or 1 for a vector. 
\item \verb'nrows_nonempty':  if known, this value must give the exact number
    of non-empty rows of a matrix or vector (that is, the number of
    rows have at least one entry).  If not known, this value must be
    set to -1.
\item \verb'ncols_nonempty':  if known, this value must give the exact number
    of non-empty columns of a matrix or vector (that is, the number of
    columns have at least one entry).  If not known, this value must be
    set to -1.
\item \verb'nvals': the number of entries in the matrix or vector.
\item \verb'format': this defines the data format of a matrix or vector.
    Currently, GraphBLAS supports four formats, described in
    Section~\ref{formats}, each with two orientations.  A \verb'GrB_Vector'
    cannot be held in \verb'GxB_HYPERSPARSE' format.
\item \verb'orientation':  whether the matrix is held by row or by column.
    This is always \verb'GrB_COLMAJOR' for a container holding data for a
    \verb'GrB_Vector', and for data holding an $n$-by-1 \verb'GrB_Matrix'
    with a single column.
\item \verb'iso': if true, all of the entries in the matrix have the same
    value, and only a single value is stored in the \verb'x' component of
    the Container.
\item \verb'jumbled': if true, the indices in any given row (if in row-major
    orientation) or column (if column-major) may appear out of order.
    Otherwise, they must appear in ascending order.
    A jumbled matrix or vector must not have any read-only components. 
\end{itemize}

The Container holds five dense \verb'GrB_Vector' objects that hold the primary
data for the matrix or vector, and a single \verb'GrB_Matrix' \verb'Y' that
holds an optional optimization structure called the hyper-hash.  These
components are fully described in Section~\ref{formats}.

\newpage
%-------------------------------------------------------------------------------
\subsubsection{{\sf GxB\_Container\_free:} free a container}
%-------------------------------------------------------------------------------
\label{container_free}

\begin{mdframed}[userdefinedwidth=6in]
{\footnotesize
\begin{verbatim}
GrB_Info GrB_free (GxB_Container *Container) ;
\end{verbatim}
} \end{mdframed}

\verb'GxB_Container_free' frees a container.  It also frees all its opaque
components.  Any read-only array inside these opaque objects is not freed.

%-------------------------------------------------------------------------------
\subsubsection{{\sf GxB\_load\_Matrix\_from\_Container:} load a matrix from a container}
%-------------------------------------------------------------------------------
\label{load_matrix_from_container}

\begin{mdframed}[userdefinedwidth=6in]
{\footnotesize
\begin{verbatim}
GrB_Info GxB_load_Matrix_from_Container     // GrB_Matrix <- GxB_Container
(
    GrB_Matrix A,               // matrix to load from the Container.  On input,
                                // A is a matrix of any size or type; on output
                                // any prior size, type, or contents is freed
                                // and overwritten with the Container.
    GxB_Container Container,    // Container with contents to load into A
    const GrB_Descriptor desc   // currently unused
) ;
\end{verbatim}
} \end{mdframed}

\verb'GxB_load_Matrix_from_Container' moves all of the data from a
\verb'GxB_Container' into a \verb'GrB_Matrix' in O(1) time.

%-------------------------------------------------------------------------------
\subsubsection{{\sf GxB\_load\_Vector\_from\_Container:} load a vector from a container}
%-------------------------------------------------------------------------------
\label{load_vector_from_container}

\begin{mdframed}[userdefinedwidth=6in]
{\footnotesize
\begin{verbatim}

GrB_Info GxB_load_Vector_from_Container     // GrB_Vector <- GxB_Container
(
    GrB_Vector V,               // GrB_Vector to load from the Container
    GxB_Container Container,    // Container with contents to load into V
    const GrB_Descriptor desc   // currently unused
) ;
\end{verbatim}
} \end{mdframed}

\verb'GxB_load_Vector_from_Container' moves all of the data from a
\verb'GxB_Container' into a \verb'GrB_Vector' in O(1) time.

\newpage
%-------------------------------------------------------------------------------
\subsubsection{{\sf GxB\_unload\_Matrix\_into\_Container:} unload a matrix into a container}
%-------------------------------------------------------------------------------
\label{unload_matrix_into_container}

\begin{mdframed}[userdefinedwidth=6in]
{\footnotesize
\begin{verbatim}
GrB_Info GxB_unload_Matrix_into_Container   // GrB_Matrix -> GxB_Container
(
    GrB_Matrix A,               // matrix to unload into the Container
    GxB_Container Container,    // Container to hold the contents of A
    const GrB_Descriptor desc   // currently unused
) ;
\end{verbatim}
} \end{mdframed}

\verb'GxB_unload_Matrix_into_Container': moves all of the data from
a \verb'GrB_Matrix' into a \verb'GxB_Container' in O(1) time.

%-------------------------------------------------------------------------------
\subsubsection{{\sf GxB\_unload\_Vector\_into\_Container:} unload a vector into a container}
%-------------------------------------------------------------------------------
\label{unload_vector_into_container}

\begin{mdframed}[userdefinedwidth=6in]
{\footnotesize
\begin{verbatim}
GrB_Info GxB_unload_Vector_into_Container   // GrB_Vector -> GxB_Container
(
    GrB_Vector V,               // vector to unload into the Container
    GxB_Container Container,    // Container to hold the contents of V
    const GrB_Descriptor desc   // currently unused
) ;
\end{verbatim}
} \end{mdframed}

\verb'GxB_unload_Vector_into_Container': moves all of the data from
a \verb'GrB_Vector' into a \verb'GxB_Container' in O(1) time.

%-------------------------------------------------------------------------------
\subsubsection{Container example: unloading/loading an entire matrix into C arrays}
%-------------------------------------------------------------------------------
\label{container_example}

The following example unloads a 
\verb'GrB_Matrix A' of size \verb'nrows'-by-\verb'ncols',
with \verb'nvals' entries, of type \verb'xtype'.  The example takes will take O(1) time,
and the only \verb'mallocs' are in \verb'GxB_Container_new' (which can be reused for
an arbitrary number of load/unload cycles), and the only frees are in
\verb'GxB_Container_free'.

Note that getting C arrays from a \verb'GrB_Matrix' is a 2-step process:
First unload the matrix A into a Container, giving \verb'GrB_Vector' objects
\verb'Container->p',
\verb'Container->i', \verb'Container->x', etc, and then unload those dense vectors into
C arrays.  This may seem tedious but it allows everything to be done in
O(1) time and space (often no new malloc'd space), and it allows support for
arbitrary integers for the \verb'p', \verb'h', and \verb'i' components of a matrix.
It also makes for a simple API overall.

{\footnotesize
\begin{verbatim}
GxB_Container_new (&Container) ;    // requires several O(1)-sized mallocs

// no malloc/free will occur below, until GxB_Container_free.

for (as many times as you like)
{

    GxB_unload_Matrix_into_Container (A, Container, desc) ;
    // A is now 0-by-0 with nvals(A)=0.  Its type is unchanged.

    // All of the following is optional; if any item in the Container is not
    // needed by the user, it can be left as-is, and then it will be put
    // back into A at the end.  (This is done for the Container->Y).

    // to extract numerical values from the Container:
    void *x = NULL ;
    uint64_t nvals = 0, nheld = 0 ;
    GrB_Type xtype = NULL ;
    int x_handling, p_handling, h_handling, i_handling, b_handling ;
    uint64_t x_size, p_size, h_size, i_size, b_size ;
    GxB_Vector_unload (Container->x, &x, &xtype, &nheld, &x_size, &x_handling,
        desc) ;

    // The C array x now has size nheld and contains the values of the original
    // GrB_Matrix A, with type xtype being the original type of the matrix A.
    // The Container->x GrB_Vector still exists but it now has length 0.
    // If the matrix A was iso-valued, nheld == 1.

    // to extract the sparsity pattern from the Container:
    GrB_Type ptype = NULL, htype = NULL, itype = NULL, btype = NULL ;
    void *p = NULL, *h = NULL, *i = NULL, *b = NULL ;
    uint64_t plen = 0, plen1 = 0, nheld = 0 ;

    switch (Container->format)
    {
        case GxB_HYPERSPARSE :
            // The Container->Y matrix can be unloaded here as well,
            // if desired.  Its use is optional.
            GxB_Vector_unload (Container->h, &h, &htype, &plen, &h_size,
                &h_handling, desc) ;
        case GxB_SPARSE :
            GxB_Vector_unload (Container->p, &p, &ptype, &plen1, &p_size,
                &p_handling, desc) ;
            GxB_Vector_unload (Container->i, &i, &itype, &nvals, &i_size,
                &i_handling, desc) ;
            break ;
        case GxB_BITMAP :
            GxB_Vector_unload (Container->b, &b, &btype, &nheld, &b_size,
                &b_handling, desc) ;
            break ;
    }

    // Now the C arrays (p, h, i, b, and x) are all populated and owned by the
    // user application.  They can be modified here, if desired.  Their C type
    // is (void *), and their actual types correspond to ptype, htype, itype,
    // btype, and xtype).

    // to load them back into A, first load them into the Container->[phbix]
    // vectors:
    switch (Container->format)
    {
        case GxB_HYPERSPARSE :
            // The Container->Y matrix can be loaded here as well,
            // if desired.  Its use is optional.
            GxB_Vector_load (Container->h, &h, htype, plen, h_size,
                h_handling, desc) ;
        case GxB_SPARSE :
            GxB_Vector_load (Container->p, &p, ptype, plen1, p_size,
                p_handling, desc) ;
            GxB_Vector_load (Container->i, &i, itype, nvals, i_size,
                i_handling, desc) ;
            break ;
        case GxB_BITMAP :
            GxB_Vector_load (Container->b, &b, btype, nheld, b_size,
                b_handling, desc) ;
            break ;
    }
    GxB_Vector_load (Container->x, &x, xtype, nheld, x_size,
        x_handling, desc) ;

    // Now the C arrays p, h, i, b, and x are all NULL.  They are in the
    // Container->p,h,b,i,x GrB_Vectors.  Load A from the non-opaque Container:

    GxB_load_Matrix_from_Container (A, Container, desc) ;
    // A is now back to its original state.  The Container and its p,h,b,i,x
    // GrB_Vectors exist but its vectors all have length 0.

}

GxB_Container_free (&Container) ;    // does several O(1)-sized free's
\end{verbatim}}

%-------------------------------------------------------------------------------
\subsubsection{Container example: unloading/loading, but not using C arrays}
%-------------------------------------------------------------------------------
\label{container_example2}

Using the container is very simple if the resulting Container \verb'GrB_Vector'
components are used directly by GraphBLAS, with no need for C arrays.  For
example, in a push/relabel maxflow algorithm, there is a need to extract the
tuples from a \verb'GrB_Vector' \verb'Delta', followed by a call to
\verb'GrB_Matrix_build' to create a matrix from that data.  In GraphBLAS v9 and
earlier, extracting the tuples requires a copy.  In v10, it can be done using
the container, without requiring a copy of the contents of \verb'Delta'.

{\footnotesize
\begin{verbatim}
GxB_Container_new (&Container) ;
for (...)
{
    GrB_Vector Delta, J_Vector ;      // computed by GraphBLAS (not shown)
    GrB_Matrix DeltaMatrix ;
    ...
    GxB_unload_Vector_into_Container (Delta, Container, desc) ;
    GxB_Matrix_build_Vector (DeltaMatrix, Container->i, J_vector,
        Container->x, GrB_PLUS_FP32, NULL) :
    GxB_load_Vector_from_Container (Delta, Container, desc) ;
}
GxB_Container_free (&Container) ;
\end{verbatim}}

The contents of the \verb'Delta' vector can be used unloaded in to the
container for use by \verb'GxB_Matrix_build_Vector', in O(1) time, and then
loaded back afterwards, also in O(1) time.  The construction of the
\verb'DeltaMatrix' takes the same time as \verb'GrB_Matrix_build', but the
extra copy that would be required for \verb'GrB_Vector_extractTuples' is
entirely avoided.

