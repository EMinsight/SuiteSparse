
\newpage
%%%%%%%%%%%%%%%%%%%%%%%%%%%%%%%%%%%%%%%%%%%%%%%%%%%%%%%%%%%%%%%%%%%%%%%%%%%%%%%%
\section{GraphBLAS Objects and their Methods} %%%%%%%%%%%%%%%%%%%%%%%%%%%%%%%%%%
%%%%%%%%%%%%%%%%%%%%%%%%%%%%%%%%%%%%%%%%%%%%%%%%%%%%%%%%%%%%%%%%%%%%%%%%%%%%%%%%
\label{objects}

SuiteSparse:GraphBLAS defines 14 different objects to represent matrices, vectors,
scalars, data types, operators (binary, unary, and index-unary), monoids,
semirings, a {\em descriptor} object used to specify optional parameters
that modify the behavior of a GraphBLAS operation, a {\em context}
object for controlling computational resources, a matrix/vector iterator,
and a container object for moving data in and out of the GraphBLAS opaque
matrix/vector objects.

The GraphBLAS API makes a distinction between {\em methods} and {\em
operations}.  A method is a function that works on a GraphBLAS object, creating
it, destroying it, or querying its contents.  An operation (not to be confused
with an operator) acts on matrices and/or vectors in a semiring.

\vspace{0.1in}
\noindent
{\small
\begin{tabular}{ll}
\hline
\verb'GrB_Type'      & a scalar data type \\
\verb'GrB_UnaryOp'   & a unary operator $z=f(x)$, where $z$ and $x$ are scalars\\
\verb'GrB_BinaryOp'  & a binary operator $z=f(x,y)$, where $z$, $x$, and $y$ are scalars\\
\verb'GrB_IndexUnaryOp'  & an index-unary operator \\
\verb'GxB_IndexBinaryOp'  & an index-binary operator \\
\verb'GrB_Monoid'    & an associative and commutative binary operator  \\
                     & and its identity value \\
\verb'GrB_Semiring'  & a monoid that defines the ``plus'' and a binary operator\\
                     & that defines the ``multiply'' for an algebraic semiring \\
\verb'GrB_Matrix'    & a 2D sparse matrix of any type \\
\verb'GrB_Vector'    & a 1D sparse column vector of any type \\
\verb'GrB_Scalar'    & a scalar of any type \\
\verb'GrB_Descriptor'& a collection of parameters that modify an operation \\
\verb'GxB_Context'   & allocating computational resources \\
\verb'GxB_Iterator'  & a matrix/vector iterator (See Section~\ref{iter}) \\
\verb'GxB_Container' & for moving data between GraphBLAS and a user application \\
\hline
\end{tabular}
}
\vspace{0.1in}

Most of these objects are implemented in C as an opaque handle, which is a
pointer to a data structure held by GraphBLAS.  User applications may not
examine the content of the object directly; instead, they can pass the handle
back to GraphBLAS which will do the work.  Assigning one handle to another
is valid but it does not make a copy of the underlying object.

The \verb'GxB_Container' object is non-opaque but includes opaque objects
as its contents.  All other objects are entirely opaque.


\newpage
%===============================================================================
\subsection{The GraphBLAS type: {\sf GrB\_Type}} %==============================
%===============================================================================
\label{type}

A GraphBLAS \verb'GrB_Type' defines the type of scalar values that a matrix or
vector contains, and the type of scalar operands for a unary or binary
operator.  There are 13 built-in types, and a user application can define
any types of its own as well.  The built-in types correspond to built-in types
in C (in the \verb'#include' files \verb'stdbool.h', \verb'stdint.h', and
\verb'complex.h') as listed in the following table.

\vspace{0.2in}
\noindent
{\footnotesize
\begin{tabular}{llll}
\hline
GraphBLAS         & C type           & description              & range \\
type              &                  &                          & \\
\hline
\verb'GrB_BOOL'   & \verb'bool'      & Boolean                  & true (1), false (0) \\
\hline
\verb'GrB_INT8'   & \verb'int8_t'    & 8-bit signed integer     & -128 to 127 \\
\verb'GrB_INT16'  & \verb'int16_t'   & 16-bit integer           & $-2^{15}$ to $2^{15}-1$ \\
\verb'GrB_INT32'  & \verb'int32_t'   & 32-bit integer           & $-2^{31}$ to $2^{31}-1$ \\
\verb'GrB_INT64'  & \verb'int64_t'   & 64-bit integer           & $-2^{63}$ to $2^{63}-1$ \\
\hline
\verb'GrB_UINT8'  & \verb'uint8_t'   & 8-bit unsigned integer   & 0 to 255 \\
\verb'GrB_UINT16' & \verb'uint16_t'  & 16-bit unsigned integer  & 0 to $2^{16}-1$ \\
\verb'GrB_UINT32' & \verb'uint32_t'  & 32-bit unsigned integer  & 0 to $2^{32}-1$ \\
\verb'GrB_UINT64' & \verb'uint64_t'  & 64-bit unsigned integer  & 0 to $2^{64}-1$ \\
\hline
\verb'GrB_FP32'   & \verb'float'     & 32-bit IEEE 754          & \verb'-Inf' to \verb'+Inf'\\
\verb'GrB_FP64'   & \verb'double'    & 64-bit IEEE 754          & \verb'-Inf' to \verb'+Inf'\\
\hline
\verb'GxB_FC32'   & \verb'float complex'  & 32-bit complex & \verb'-Inf' to \verb'+Inf'\\
\verb'GxB_FC64'   & \verb'double complex' & 64-bit complex & \verb'-Inf' to \verb'+Inf'\\
\hline
\end{tabular}
}
\vspace{0.2in}

The C11 definitions of \verb'float complex' and \verb'double complex'
are not always available.  The \verb'GraphBLAS.h' header defines them as
\verb'GxB_FC32_t' and \verb'GxB_FC64_t', respectively.

The user application can also define new types based on any \verb'typedef' in
the C language whose values are held in a contiguous region of memory of fixed
size.  For example, a user-defined \verb'GrB_Type' could be created to hold any
C \verb'struct' whose content is self-contained.  A C \verb'struct' containing
pointers might be problematic because GraphBLAS would not know to dereference
the pointers to traverse the entire ``scalar'' entry, but this can be done if
the objects referenced by these pointers are not moved.  A user-defined complex
type with real and imaginary types can be defined, or even a ``scalar'' type
containing a fixed-sized dense matrix (see Section~\ref{type_new}).  The
possibilities are endless.  GraphBLAS can create and operate on sparse matrices
and vectors in any of these types, including any user-defined ones.  For
user-defined types, GraphBLAS simply moves the data around itself (via
\verb'memcpy'), and then passes the values back to user-defined functions when
it needs to do any computations on the type.  The next sections describe the
methods for the \verb'GrB_Type' object:

\vspace{0.2in}
{\footnotesize
\begin{tabular}{lll}
\hline
GraphBLAS function       & purpose                          & Section \\
\hline
\verb'GrB_Type_new'      & create a user-defined type       & \ref{type_new} \\
\verb'GxB_Type_new'      & create a user-defined type,
                            with name and definition        & \ref{type_new_named} \\
\verb'GrB_Type_wait'     & wait for a user-defined type     & \ref{type_wait} \\
\verb'GxB_Type_from_name'& return the type from its name    & \ref{type_from_name} \\
\verb'GrB_Type_free'     & free a user-defined type         & \ref{type_free} \\
\verb'GrB_get'           & get properties of a type         & \ref{get_set_type} \\
\verb'GrB_set'           & set the type name/definitition   & \ref{get_set_type} \\
\hline
\end{tabular}
}

%-------------------------------------------------------------------------------
\subsubsection{{\sf GrB\_Type\_new:} create a user-defined type}
%-------------------------------------------------------------------------------
\label{type_new}

\begin{mdframed}[userdefinedwidth=6in]
{\footnotesize
\begin{verbatim}
GrB_Info GrB_Type_new           // create a new GraphBLAS type
(
    GrB_Type *type,             // handle of user type to create
    size_t sizeof_ctype         // size = sizeof (ctype) of the C type
) ;
\end{verbatim}
}\end{mdframed}

\verb'GrB_Type_new' creates a new user-defined type.  The \verb'type' is a
handle, or a pointer to an opaque object.  The handle itself must not be
\verb'NULL' on input, but the content of the handle can be undefined.  On
output, the handle contains a pointer to a newly created type.
The \verb'ctype' is the type in C that will be used to construct the new
GraphBLAS type.  It can be either a built-in C type, or defined by a
\verb'typedef'.
The second parameter should be passed as \verb'sizeof(ctype)'.  The only
requirement on the C type is that \verb'sizeof(ctype)' is valid in C, and
that the type reside in a contiguous block of memory so that it can be moved
with \verb'memcpy'.  For example, to create a user-defined type called
\verb'Complex' for double-precision complex values using the C11
\verb'double complex' type, the following can be used.  A complete example can
be found in the \verb'usercomplex.c' and \verb'usercomplex.h' files in the
\verb'Demo' folder.

    {\footnotesize
    \begin{verbatim}
    #include <math.h>
    #include <complex.h>
    GrB_Type Complex ;
    GrB_Type_new (&Complex, sizeof (double complex)) ;    \end{verbatim} }

To demonstrate the flexibility of the \verb'GrB_Type', consider a ``scalar''
consisting of 4-by-4 floating-point matrix and a string.  This type might be
useful for the 4-by-4 translation/rotation/scaling matrices that arise in
computer graphics, along with a string containing a description or even a
regular expression that can be parsed and executed in a user-defined operator.
All that is required is a fixed-size type, where \verb'sizeof(ctype)' is
a constant.

    {\footnotesize
    \begin{verbatim}
    typedef struct
    {
        float stuff [4][4] ;
        char whatstuff [64] ;
    }
    wildtype ;
    GrB_Type WildType ;
    GrB_Type_new (&WildType, sizeof (wildtype)) ; \end{verbatim} }

With this type a sparse matrix can be created in which each entry consists of a
4-by-4 dense matrix \verb'stuff' and a 64-character string \verb'whatstuff'.
GraphBLAS treats this 4-by-4 as a ``scalar.'' Any GraphBLAS method or operation
that simply moves data can be used with this type without any further
information from the user application.  For example, entries of this type can
be assigned to and extracted from a matrix or vector, and matrices containing
this type can be transposed.  A working example (\verb'wildtype.c'
in the \verb'Demo' folder) creates matrices and multiplies them with
a user-defined semiring with this type.

Performing arithmetic on matrices and vectors with user-defined types requires
operators to be defined.  Refer to Section~\ref{user} for more details on these
example user-defined types.

User defined types created by \verb'GrB_Type_new' will not work with
the JIT; use \verb'GxB_Type_new' instead.

%-------------------------------------------------------------------------------
\subsubsection{{\sf GxB\_Type\_new:} create a user-defined type (with name and definition)}
%-------------------------------------------------------------------------------
\label{type_new_named}

\begin{mdframed}[userdefinedwidth=6in]
{\footnotesize
\begin{verbatim}
GrB_Info GxB_Type_new           // create a new named GraphBLAS type
(
    GrB_Type *type,             // handle of user type to create
    size_t sizeof_ctype,        // size = sizeof (ctype) of the C type
    const char *type_name,      // name of the type (max 128 characters)
    const char *type_defn       // typedef for the type (no max length)
) ;
\end{verbatim}
}\end{mdframed}

\verb'GxB_Type_new' creates a type with a name and definition that are known to
GraphBLAS, as strings.  The \verb'type_name' is any valid string (max length of 128
characters, including the required null-terminating character) that may
appear as the name of a C type created by a C \verb'typedef' statement.  It must
not contain any white-space characters.  For example, to create a type of size
16*4+1 = 65 bytes, with a 4-by-4 dense float array and a 32-bit integer:

    {\footnotesize
    \begin{verbatim}
    typedef struct { float x [4][4] ; int color ; } myquaternion ;
    GrB_Type MyQtype ;
    GxB_Type_new (&MyQtype, sizeof (myquaternion), "myquaternion",
        "typedef struct { float x [4][4] ; int color ; } myquaternion ;") ; \end{verbatim}}

The \verb'type_name' and \verb'type_defn' are both null-terminated strings.
The two strings are optional, but are
required to enable the JIT compilation of kernels that use this type.
At most \verb'GxB_MAX_NAME_LEN' (128) characters are accessed in \verb'type_name';
characters beyond that limit are silently ignored.

If the \verb'sizeof_ctype' is zero, and the strings are valid, a
JIT kernel is compiled just to determine the size of the type.  This is
feature useful for interfaces in languages other than C, which could create
valid strings for C types but would not have a reliable way to determine the
size of the type.

The above example is identical to the following usage, except that
\verb'GrB_Type_new' requires \verb'sizeof_ctype' to be nonzero, and equal
to the size of the C type.

    {\footnotesize
    \begin{verbatim}
    typedef struct { float x [4][4] ; int color ; } myquaternion ;
    GrB_Type MyQtype ;
    GxB_Type_new (&MyQtype, sizeof (myquaternion)) ;
    GrB_set (MyQtype, "myquaternion", GxB_JIT_C_NAME) ;
    GrB_set (MyQtype, "typedef struct { float x [4][4] ; int color ; } myquaternion ;"
        GxB_JIT_C_DEFINITION) ; \end{verbatim}}

%-------------------------------------------------------------------------------
\subsubsection{{\sf GrB\_Type\_wait:} wait for a type}
%-------------------------------------------------------------------------------
\label{type_wait}

\begin{mdframed}[userdefinedwidth=6in]
{\footnotesize
\begin{verbatim}
GrB_Info GrB_wait               // wait for a user-defined type
(
    GrB_Type type,              // type to wait for
    int mode                    // GrB_COMPLETE or GrB_MATERIALIZE
) ;
\end{verbatim}
}\end{mdframed}

After creating a user-defined type, a GraphBLAS library may choose to exploit
non-blocking mode to delay its creation.  Currently, SuiteSparse:GraphBLAS
currently does nothing except to ensure that \verb'type' is valid.

\newpage
%-------------------------------------------------------------------------------
\subsubsection{{\sf GxB\_Type\_from\_name:} return the type from its name}
%-------------------------------------------------------------------------------
\label{type_from_name}

\begin{mdframed}[userdefinedwidth=6in]
{\footnotesize
\begin{verbatim}
GrB_Info GxB_Type_from_name     // return the built-in GrB_Type from a name
(
    GrB_Type *type,             // built-in type, or NULL if user-defined
    const char *type_name       // array of size at least GxB_MAX_NAME_LEN
) ;
\end{verbatim}
}\end{mdframed}

Returns the built-in type from the corresponding name of the type.  The following
examples both return \verb'type' as \verb'GrB_BOOL'.

{\footnotesize
\begin{verbatim}
    GxB_Type_from_name (&type, "bool") ;
    GxB_Type_from_name (&type, "GrB_BOOL") ; \end{verbatim} }

If the name is from a user-defined type, the \verb'type' is returned as
\verb'NULL'.  This is not an error condition.  The user application must itself
do this translation since GraphBLAS does not keep a registry of all
user-defined types.

With this function, a user application can manage the translation for
both built-in types and its own user-defined types, as in the following
example.

{\footnotesize
\begin{verbatim}
    typedef struct { double x ; char stuff [16] ; } myfirsttype ;
    typedef struct { float z [4][4] ; int color ; } myquaternion ;
    GrB_Type MyType1, MyQType ;
    GxB_Type_new (&MyType1, sizeof (myfirsttype), "myfirsttype",
        "typedef struct { double x ; char stuff [16] ; } myfirsttype ;") ;
    GxB_Type_new (&MyQType, sizeof (myquaternion), "myquaternion",
        "typedef struct { float z [4][4] ; int color ; } myquaternion ;") ;

    GrB_Matrix A ;
    // ... create a matrix A of some built-in or user-defined type

    // later on, to query the type of A:
    size_t typesize ;
    GrB_Scalar_new (s, GrB_UINT64) ;
    GrB_get (type, s, GrB_SIZE) ;
    GrB_Scalar_extractElement (&typesize, GrB_UINT64) ;
    GrB_Type atype ;
    char atype_name [GxB_MAX_NAME_LEN] ;
    GrB_get (A, atype_name, GrB_EL_TYPE_STRING) ;
    GxB_Type_from_name (&atype, atype_name) ;
    if (atype == NULL)
    {
        // This is not yet an error.  It means that A has a user-defined type.
        if ((strcmp (atype_name, "myfirsttype")) == 0) atype = MyType1 ;
        else if ((strcmp (atype_name, "myquaternion")) == 0) atype = MyQType ;
        else { ... this is now an error ... the type of A is unknown.  }
    }\end{verbatim} }

%-------------------------------------------------------------------------------
\subsubsection{{\sf GrB\_Type\_free:} free a user-defined type}
%-------------------------------------------------------------------------------
\label{type_free}

\begin{mdframed}[userdefinedwidth=6in]
{\footnotesize
\begin{verbatim}
GrB_Info GrB_free               // free a user-defined type
(
    GrB_Type *type              // handle of user-defined type to free
) ;
\end{verbatim}
}\end{mdframed}

\verb'GrB_Type_free' frees a user-defined type.
Either usage:

    {\small
    \begin{verbatim}
    GrB_Type_free (&type) ;
    GrB_free (&type) ; \end{verbatim}}

\noindent
frees the user-defined \verb'type' and
sets \verb'type' to \verb'NULL'.
It safely does nothing if passed a \verb'NULL'
handle, or if \verb'type == NULL' on input.

It is safe to attempt to free a built-in type.  SuiteSparse:GraphBLAS silently
ignores the request and returns \verb'GrB_SUCCESS'.  A user-defined type should
not be freed until all operations using the type are completed.
SuiteSparse:GraphBLAS attempts to detect this condition but it must query a
freed object in its attempt.  This is hazardous and not recommended.
Operations on such objects whose type has been freed leads to undefined
behavior.

It is safe to first free a type, and then a matrix of that type, but after the
type is freed the matrix can no longer be used.  The only safe thing that can
be done with such a matrix is to free it.

The function signature of \verb'GrB_Type_free' uses the generic name
\verb'GrB_free', which can free any GraphBLAS object. See Section~\ref{free}
details.  GraphBLAS includes many such generic functions.  When describing a
specific variation, a function is described with its specific name in this User
Guide (such as \verb'GrB_Type_free').  When discussing features applicable to
all specific forms, the generic name is used instead (such as \verb'GrB_free').




\newpage
%===============================================================================
\subsection{GraphBLAS unary operators: {\sf GrB\_UnaryOp}, $z=f(x)$} %==========
%===============================================================================
\label{unaryop}

A unary operator is a scalar function of the form $z=f(x)$.  The domain (type)
of $z$ and $x$ need not be the same.

In the notation in the tables
below, $T$ is any of the 13 built-in types and is a place-holder for
\verb'BOOL', \verb'INT8', \verb'UINT8', ...
\verb'FP32', \verb'FP64', \verb'FC32', or \verb'FC64'.
For example, \verb'GrB_AINV_INT32' is a unary operator that computes
\verb'z=-x' for two values \verb'x' and \verb'z' of type \verb'GrB_INT32'.

The notation $R$ refers to any real type (all but \verb'FC32' and \verb'FC64'),
$I$ refers to any integer type (\verb'INT*' and \verb'UINT*'),
$F$ refers to any real or complex floating point type
(\verb'FP32', \verb'FP64', \verb'FC32', or \verb'FC64'),
$Z$ refers to any complex floating point type
(\verb'FC32' or \verb'FC64'),
and $N$ refers to \verb'INT32' or \verb'INT64'.

The logical negation operator \verb'GrB_LNOT' only works on Boolean types.  The
\verb'GxB_LNOT_'$R$ functions operate on inputs of type $R$, implicitly
typecasting their input to Boolean and returning result of type $R$, with a
value 1 for true and 0 for false.  The operators \verb'GxB_LNOT_BOOL' and
\verb'GrB_LNOT' are identical.

\vspace{0.2in}
{\footnotesize
\begin{tabular}{|llll|}
\hline
\multicolumn{4}{|c|}{Unary operators for all types} \\
\hline
GraphBLAS name          & types (domains)   & $z=f(x)$      & description \\
\hline
\verb'GxB_ONE_'$T$      & $T \rightarrow T$ & $z = 1$       & one \\
\verb'GrB_IDENTITY_'$T$ & $T \rightarrow T$ & $z = x$       & identity \\
\verb'GrB_AINV_'$T$     & $T \rightarrow T$ & $z = -x$      & additive inverse \\
\verb'GrB_MINV_'$T$     & $T \rightarrow T$ & $z = 1/x$     & multiplicative inverse \\
\hline
\end{tabular}

\vspace{0.2in}
\begin{tabular}{|llll|}
\hline
\multicolumn{4}{|c|}{Unary operators for real and integer types} \\
\hline
GraphBLAS name          & types (domains)   & $z=f(x)$      & description \\
\hline
\verb'GrB_ABS_'$T$      & $R \rightarrow R$ & $z = |x|$     & absolute value \\
\verb'GrB_LNOT'         & \verb'bool'
                          $\rightarrow$
                          \verb'bool'       & $z = \lnot x$ & logical negation \\
\verb'GxB_LNOT_'$R$     & $R \rightarrow R$ & $z = \lnot (x \ne 0)$ & logical negation \\
\verb'GrB_BNOT_'$I$     & $I \rightarrow I$ & $z = \lnot x$ & bitwise negation \\
\hline
\end{tabular}

\vspace{0.2in}
\begin{tabular}{|llll|}
\hline
\multicolumn{4}{|c|}{Index-based unary operators for any type (including user-defined)} \\
\hline
GraphBLAS name            & types (domains)   & $z=f(a_{ij})$      & description \\
\hline
\verb'GxB_POSITIONI_'$N$  & $ \rightarrow N$  & $z = i$       & row index (0-based) \\
\verb'GxB_POSITIONI1_'$N$ & $ \rightarrow N$  & $z = i+1$     & row index (1-based) \\
\verb'GxB_POSITIONJ_'$N$  & $ \rightarrow N$  & $z = j$       & column index (0-based) \\
\verb'GxB_POSITIONJ1_'$N$ & $ \rightarrow N$  & $z = j+1$     & column index (1-based) \\
\hline
\end{tabular}
\vspace{0.2in}

\begin{tabular}{|llll|}
\hline
\multicolumn{4}{|c|}{Unary operators for floating-point types (real and complex)} \\
\hline
GraphBLAS name          & types (domains)   & $z=f(x)$      & description \\
\hline
\verb'GxB_SQRT_'$F$     & $F \rightarrow F$ & $z = \sqrt(x)$       & square root \\
\verb'GxB_LOG_'$F$      & $F \rightarrow F$ & $z = \log_e(x)$      & natural logarithm \\
\verb'GxB_EXP_'$F$      & $F \rightarrow F$ & $z = e^x$            & natural exponent \\
\hline
\verb'GxB_LOG10_'$F$    & $F \rightarrow F$ & $z = \log_{10}(x)$   & base-10 logarithm \\
\verb'GxB_LOG2_'$F$     & $F \rightarrow F$ & $z = \log_2(x)$      & base-2 logarithm \\
\verb'GxB_EXP2_'$F$     & $F \rightarrow F$ & $z = 2^x$            & base-2 exponent \\
\hline
\verb'GxB_EXPM1_'$F$    & $F \rightarrow F$ & $z = e^x - 1$        & natural exponent - 1 \\
\verb'GxB_LOG1P_'$F$    & $F \rightarrow F$ & $z = \log(x+1)$      & natural log of $x+1$ \\
\hline
\verb'GxB_SIN_'$F$      & $F \rightarrow F$ & $z = \sin(x)$        & sine \\
\verb'GxB_COS_'$F$      & $F \rightarrow F$ & $z = \cos(x)$        & cosine \\
\verb'GxB_TAN_'$F$      & $F \rightarrow F$ & $z = \tan(x)$        & tangent \\
\hline
\verb'GxB_ASIN_'$F$     & $F \rightarrow F$ & $z = \sin^{-1}(x)$        & inverse sine \\
\verb'GxB_ACOS_'$F$     & $F \rightarrow F$ & $z = \cos^{-1}(x)$        & inverse cosine \\
\verb'GxB_ATAN_'$F$     & $F \rightarrow F$ & $z = \tan^{-1}(x)$        & inverse tangent \\
\hline
\verb'GxB_SINH_'$F$     & $F \rightarrow F$ & $z = \sinh(x)$        & hyperbolic sine \\
\verb'GxB_COSH_'$F$     & $F \rightarrow F$ & $z = \cosh(x)$        & hyperbolic cosine \\
\verb'GxB_TANH_'$F$     & $F \rightarrow F$ & $z = \tanh(x)$        & hyperbolic tangent \\
\hline
\verb'GxB_ASINH_'$F$    & $F \rightarrow F$ & $z = \sinh^{-1}(x)$        & inverse hyperbolic sine \\
\verb'GxB_ACOSH_'$F$    & $F \rightarrow F$ & $z = \cosh^{-1}(x)$        & inverse hyperbolic cosine \\
\verb'GxB_ATANH_'$F$    & $F \rightarrow F$ & $z = \tanh^{-1}(x)$        & inverse hyperbolic tangent \\
\hline
\verb'GxB_SIGNUM_'$F$   & $F \rightarrow F$ & $z = \sgn(x)$                 & sign, or signum function \\
\verb'GxB_CEIL_'$F$     & $F \rightarrow F$ & $z = \lceil x \rceil $       & ceiling function \\
\verb'GxB_FLOOR_'$F$    & $F \rightarrow F$ & $z = \lfloor x \rfloor $     & floor function \\
\verb'GxB_ROUND_'$F$    & $F \rightarrow F$ & $z = \mbox{round}(x)$        & round to nearest \\
\verb'GxB_TRUNC_'$F$    & $F \rightarrow F$ & $z = \mbox{trunc}(x)$        & round towards zero \\
\hline
\verb'GxB_ISINF_'$F$    & $F \rightarrow $ \verb'bool' & $z = \mbox{isinf}(x)$ & true if $\pm \infty$ \\
\verb'GxB_ISNAN_'$F$    & $F \rightarrow $ \verb'bool' & $z = \mbox{isnan}(x)$ & true if \verb'NaN' \\
\verb'GxB_ISFINITE_'$F$ & $F \rightarrow $ \verb'bool' & $z = \mbox{isfinite}(x)$ & true if finite \\
\hline
\end{tabular}
\vspace{0.2in}

\begin{tabular}{|llll|}
\hline
\multicolumn{4}{|c|}{Unary operators for floating-point types (real only)} \\
\hline
GraphBLAS name          & types (domains)   & $z=f(x)$      & description \\
\hline
\verb'GxB_LGAMMA_'$R$   & $R \rightarrow R$ & $z = \log(|\Gamma (x)|)$  & log of gamma function \\
\verb'GxB_TGAMMA_'$R$   & $R \rightarrow R$ & $z = \Gamma(x)$        & gamma function \\
\verb'GxB_ERF_'$R$      & $R \rightarrow R$ & $z = \erf(x)$          & error function \\
\verb'GxB_ERFC_'$R$     & $R \rightarrow R$ & $z = \erfc(x)$         & complimentary error function \\
\verb'GxB_CBRT_'$R$     & $R \rightarrow R$ & $z = x^{1/3}$          & cube root \\
\hline
\verb'GxB_FREXPX_'$R$   & $R \rightarrow R$ & $z = \mbox{frexpx}(x)$  & normalized fraction \\
\verb'GxB_FREXPE_'$R$   & $R \rightarrow R$ & $z = \mbox{frexpe}(x)$  & normalized exponent \\
\hline
\end{tabular}
\vspace{0.2in}

\begin{tabular}{|llll|}
\hline
\multicolumn{4}{|c|}{Unary operators for complex types} \\
\hline
GraphBLAS name          & types (domains)   & $z=f(x)$      & description \\
\hline
\verb'GxB_CONJ_'$Z$    & $Z \rightarrow Z$ & $z = \overline{x}$     & complex conjugate \\
\verb'GxB_ABS_'$Z$     & $Z \rightarrow F$ & $z = |x|$              & absolute value \\
\verb'GxB_CREAL_'$Z$   & $Z \rightarrow F$ & $z = \mbox{real}(x)$   & real part \\
\verb'GxB_CIMAG_'$Z$   & $Z \rightarrow F$ & $z = \mbox{imag}(x)$   & imaginary part \\
\verb'GxB_CARG_'$Z$    & $Z \rightarrow F$ & $z = \mbox{carg}(x)$   & angle \\
\hline
\end{tabular}
}
\vspace{0.2in}

Built-in index-based unary operators return the row or column index of an entry.  For a
matrix $z=f(a_{ij})$ returns $z = i$ or $z = j$, or +1 for 1-based indices.
The latter is useful in the MATLAB/Octave interface, where row and column indices are
1-based.  When applied to a vector, $j$ is always zero, and $i$ is the index in
the vector.  These built-in unary operators come in two types: \verb'INT32' and
\verb'INT64', which is the type of the output, $z$.  The functions are agnostic
to the type of their inputs; they only depend on the position of the entries,
not their values.
User-defined index-based operators cannot be defined by \verb'GrB_UnaryOp_new';
use \verb'GrB_IndexUnaryOp_new' instead; see Section~\ref{idxunop}.

\verb'GxB_FREXPX' and \verb'GxB_FREXPE' return the mantissa and exponent,
respectively, from the C11 \verb'frexp' function.  The exponent is
returned as a floating-point value, not an integer.

The operators \verb'GxB_EXPM1_FC*' and \verb'GxB_LOG1P_FC*' for complex
types are currently not accurate.  They will be revised in a future version.

The functions \verb'casin', \verb'casinf', \verb'casinh', and \verb'casinhf'
provided by Microsoft Visual Studio for computing $\sin^{-1}(x)$ and
$\sinh^{-1}(x)$ when $x$ is complex do not compute the correct result.  Thus,
the unary operators \verb'GxB_ASIN_FC32', \verb'GxB_ASIN_FC64'
\verb'GxB_ASINH_FC32', and \verb'GxB_ASINH_FC64' do not work properly if the MS
Visual Studio compiler is used.  These functions work properly if the gcc, icc,
or clang compilers are used on Linux or MacOS.

Integer division by zero normally terminates an application, but this is
avoided in SuiteSparse:GraphBLAS.  For details, see the binary
\verb'GrB_DIV_'$T$ operators.

\begin{alert}
{\bf SPEC:} The definition of integer division by zero is an extension to the
specification.
\end{alert}

The next sections define the following methods for the \verb'GrB_UnaryOp'
object:

\vspace{0.1in}
{\footnotesize
\noindent
\begin{tabular}{lll}
\hline
GraphBLAS function   & purpose                                      & Section \\
\hline
\verb'GrB_UnaryOp_new'   & create a user-defined unary operator         & \ref{unaryop_new} \\
\verb'GxB_UnaryOp_new'   & create a named user-defined unary operator   & \ref{unaryop_new_named} \\
\verb'GrB_UnaryOp_wait'  & wait for a user-defined unary operator       & \ref{unaryop_wait} \\
\verb'GrB_UnaryOp_free'  & free a user-defined unary operator   & \ref{unaryop_free} \\
\verb'GrB_get'           & get properties of an operator    & \ref{get_set_unop} \\
\verb'GrB_set'           & set the operator name/definition & \ref{get_set_unop} \\
\hline
\end{tabular}
}
\vspace{0.1in}

% \newpage
%-------------------------------------------------------------------------------
\subsubsection{{\sf GrB\_UnaryOp\_new:} create a user-defined unary operator}
%-------------------------------------------------------------------------------
\label{unaryop_new}

\begin{mdframed}[userdefinedwidth=6in]
{\footnotesize
\begin{verbatim}
GrB_Info GrB_UnaryOp_new            // create a new user-defined unary operator
(
    GrB_UnaryOp *unaryop,           // handle for the new unary operator
    void *function,                 // pointer to the unary function
    GrB_Type ztype,                 // type of output z
    GrB_Type xtype                  // type of input x
) ;
\end{verbatim} }\end{mdframed}

\verb'GrB_UnaryOp_new' creates a new unary operator.  The new operator is
returned in the \verb'unaryop' handle, which must not be \verb'NULL' on input.
On output, its contents contains a pointer to the new unary operator.

The two types \verb'xtype' and \verb'ztype' are the GraphBLAS types of the
input $x$ and output $z$ of the user-defined function $z=f(x)$.  These types
may be built-in types or user-defined types, in any combination.  The two types
need not be the same, but they must be previously defined before passing them
to \verb'GrB_UnaryOp_new'.

The \verb'function' argument to \verb'GrB_UnaryOp_new' is a pointer to a
user-defined function with the following signature:

    {\footnotesize
    \begin{verbatim}
    void (*f) (void *z, const void *x) ; \end{verbatim} }

When the function \verb'f' is called, the arguments \verb'z' and \verb'x' are
passed as \verb'(void *)' pointers, but they will be pointers to values of the
correct type, defined by \verb'ztype' and \verb'xtype', respectively, when the
operator was created.

{\bf NOTE:}
The pointers passed to a user-defined operator may not be unique.  That is, the
user function may be called with multiple pointers that point to the same
space, such as when \verb'z=f(z,y)' is to be computed by a binary operator, or
\verb'z=f(z)' for a unary operator.  Any parameters passed to the user-callable
function may be aliased to each other.

\newpage
%-------------------------------------------------------------------------------
\subsubsection{{\sf GxB\_UnaryOp\_new:} create a named user-defined unary operator}
%-------------------------------------------------------------------------------
\label{unaryop_new_named}

\begin{mdframed}[userdefinedwidth=6in]
{\footnotesize
\begin{verbatim}
GrB_Info GxB_UnaryOp_new            // create a new user-defined unary operator
(
    GrB_UnaryOp *unaryop,           // handle for the new unary operator
    GxB_unary_function function,    // pointer to the unary function
    GrB_Type ztype,                 // type of output z
    GrB_Type xtype,                 // type of input x
    const char *unop_name,          // name of the user function
    const char *unop_defn           // definition of the user function
) ;
\end{verbatim} }\end{mdframed}

Creates a named \verb'GrB_UnaryOp'.  Only the first 127 characters of
\verb'unop_name' are used.  The \verb'unop_defn' is a string containing the
entire function itself.  For example:

    {\footnotesize
    \begin{verbatim}
    void square (double *z, double *x) { (*z) = (*x) * (*x) ; } ;
    ...
    GrB_Type Square ;
    GxB_UnaryOp_new (&Square, square, GrB_FP64, GrB_FP64, "square",
        "void square (double *z, double *x) { (*z) = (*x) * (*x) ; } ;") ;
    \end{verbatim}}

The two strings \verb'unop_name' and \verb'unop_defn' are optional, but are
required to enable the JIT compilation of kernels that use this operator.

If JIT compilation is enabled, or if the corresponding JIT kernel has been
copied into the \verb'PreJIT' folder, the \verb'function' may be \verb'NULL'.
In this case, a JIT kernel is compiled that contains just the user-defined
function.  If the JIT is disabled and the \verb'function' is \verb'NULL', this
method returns \verb'GrB_NULL_POINTER'.

The above example is identical to the following usage, except that
\verb'GrB_UnaryOp_new' requires a non-NULL function pointer.

    {\footnotesize
    \begin{verbatim}
    void square (double *z, double *x) { (*z) = (*x) * (*x) ; } ;
    ...
    GrB_Type Square ;
    GrB_UnaryOp_new (&Square, square, GrB_FP64, GrB_FP64) ;
    GrB_set (Square, "square", GxB_JIT_C_NAME) ;
    GrB_set (Square, "void square (double *z, double *x) { (*z) = (*x) * (*x) ; } ;",
        GxB_JIT_C_DEFINITION) ; \end{verbatim}}

% \newpage
%-------------------------------------------------------------------------------
\subsubsection{{\sf GrB\_UnaryOp\_wait:} wait for a unary operator}
%-------------------------------------------------------------------------------
\label{unaryop_wait}

\begin{mdframed}[userdefinedwidth=6in]
{\footnotesize
\begin{verbatim}
GrB_Info GrB_wait               // wait for a user-defined unary operator
(
    GrB_UnaryOp unaryop,        // unary operator to wait for
    int mode                    // GrB_COMPLETE or GrB_MATERIALIZE
) ;
\end{verbatim}
}\end{mdframed}

After creating a user-defined unary operator, a GraphBLAS library may choose to
exploit non-blocking mode to delay its creation.  Currently,
SuiteSparse:GraphBLAS currently does nothing except to ensure that the
\verb'unaryop' is valid.

% \newpage
%-------------------------------------------------------------------------------
\subsubsection{{\sf GrB\_UnaryOp\_free:} free a user-defined unary operator}
%-------------------------------------------------------------------------------
\label{unaryop_free}

\begin{mdframed}[userdefinedwidth=6in]
{\footnotesize
\begin{verbatim}
GrB_Info GrB_free                   // free a user-created unary operator
(
    GrB_UnaryOp *unaryop            // handle of unary operator to free
) ;
\end{verbatim}
}\end{mdframed}

\verb'GrB_UnaryOp_free' frees a user-defined unary operator.
Either usage:

    {\small
    \begin{verbatim}
    GrB_UnaryOp_free (&unaryop) ;
    GrB_free (&unaryop) ; \end{verbatim}}

\noindent
frees the \verb'unaryop' and sets \verb'unaryop' to \verb'NULL'.
It safely does nothing if passed a \verb'NULL'
handle, or if \verb'unaryop == NULL' on input.
It does nothing at all if passed a built-in unary operator.




\newpage
%===============================================================================
\subsection{GraphBLAS binary operators: {\sf GrB\_BinaryOp}, $z=f(x,y)$} %======
%===============================================================================
\label{binaryop}

A binary operator is a scalar function of the form $z=f(x,y)$.  The types of
$z$, $x$, and $y$ need not be the same.  The built-in binary operators are
listed in the tables below.  The notation $T$ refers to any of the 13
built-in types, but two of those types are SuiteSparse extensions
(\verb'GxB_FC32' and \verb'GxB_FC64').  For those types, the operator name
always starts with \verb'GxB', not \verb'GrB').
The notation $R$ refers to any real type (all but \verb'FC32' and \verb'FC64').

The six \verb'GxB_IS*' comparators and the \verb'GxB_*' logical
operators all return a result one for true and zero for false, in the same
domain $T$ or $R$ as their inputs.  These six comparators are useful
as ``multiply'' operators for creating semirings with non-Boolean monoids.

\vspace{0.2in}
{\footnotesize
\begin{tabular}{|llll|}
\hline
\multicolumn{4}{|c|}{Binary operators for all 13 types} \\
\hline
GraphBLAS name        & types (domains)            & $z=f(x,y)$      & description \\
\hline
% numeric TxT->T
\verb'GrB_FIRST_'$T$  & $T \times T \rightarrow T$ & $z = x$         & first argument \\
\verb'GrB_SECOND_'$T$ & $T \times T \rightarrow T$ & $z = y$         & second argument \\
\verb'GxB_ANY_'$T$    & $T \times T \rightarrow T$ & $z = x$ or $y$  & pick $x$ or $y$ arbitrarily \\
\verb'GrB_ONEB_'$T$   & $T \times T \rightarrow T$ & $z = 1$         & one \\
\verb'GxB_PAIR_'$T$   & $T \times T \rightarrow T$ & $z = 1$         & one (historical) \\
\verb'GrB_PLUS_'$T$   & $T \times T \rightarrow T$ & $z = x+y$       & addition \\
\verb'GrB_MINUS_'$T$  & $T \times T \rightarrow T$ & $z = x-y$       & subtraction \\
\verb'GxB_RMINUS_'$T$ & $T \times T \rightarrow T$ & $z = y-x$       & reverse subtraction \\
\verb'GrB_TIMES_'$T$  & $T \times T \rightarrow T$ & $z = xy$        & multiplication \\
\verb'GrB_DIV_'$T$    & $T \times T \rightarrow T$ & $z = x/y$       & division \\
\verb'GxB_RDIV_'$T$   & $T \times T \rightarrow T$ & $z = y/x$       & reverse division \\
\verb'GxB_POW_'$T$    & $T \times T \rightarrow T$ & $z = x^y$       & power \\
\hline
% TxT->T comparators
\verb'GxB_ISEQ_'$T$   & $T \times T \rightarrow T$ & $z = (x == y)$  & equal \\
\verb'GxB_ISNE_'$T$   & $T \times T \rightarrow T$ & $z = (x \ne y)$ & not equal \\
\hline
\end{tabular}
}
\vspace{0.2in}

The \verb'GxB_POW_*' operators for real types do not return a complex result,
and thus $z = f(x,y) = x^y$ is undefined if $x$ is negative and $y$ is not an
integer.  To compute a complex result, use \verb'GxB_POW_FC32' or
\verb'GxB_POW_FC64'.

Operators that require the domain to be ordered (\verb'MIN', \verb'MAX',
less-than, greater-than, and so on) are not defined for
complex types.  These are listed in the following table:

\vspace{0.2in}
{\footnotesize
\begin{tabular}{|llll|}
\hline
\multicolumn{4}{|c|}{Binary operators for all non-complex types} \\
\hline
GraphBLAS name        & types (domains)            & $z=f(x,y)$      & description \\
\hline
% numeric RxR->R
\verb'GrB_MIN_'$R$    & $R \times R \rightarrow R$ & $z = \min(x,y)$ & minimum \\
\verb'GrB_MAX_'$R$    & $R \times R \rightarrow R$ & $z = \max(x,y)$ & maximum \\
\hline
% RxR->R comparators
\verb'GxB_ISGT_'$R$   & $R \times R \rightarrow R$ & $z = (x >   y)$ & greater than \\
\verb'GxB_ISLT_'$R$   & $R \times R \rightarrow R$ & $z = (x <   y)$ & less than  \\
\verb'GxB_ISGE_'$R$   & $R \times R \rightarrow R$ & $z = (x \ge y)$ & greater than or equal \\
\verb'GxB_ISLE_'$R$   & $R \times R \rightarrow R$ & $z = (x \le y)$ & less than or equal  \\
\hline
% RxR->R logical
\verb'GxB_LOR_'$R$    & $R \times R \rightarrow R$ & $z = (x \ne 0) \vee    (y \ne 0) $ & logical OR \\
\verb'GxB_LAND_'$R$   & $R \times R \rightarrow R$ & $z = (x \ne 0) \wedge  (y \ne 0) $ & logical AND \\
\verb'GxB_LXOR_'$R$   & $R \times R \rightarrow R$ & $z = (x \ne 0) \veebar (y \ne 0) $ & logical XOR \\
\hline
\end{tabular}
}
\vspace{0.2in}

Another set of six kinds of built-in comparators have the form $T
\times T \rightarrow $\verb'bool'.  Note that when $T$ is \verb'bool', the six
operators give the same results as the six \verb'GxB_IS*_BOOL' operators in the
table above.  These six comparators are useful as ``multiply''
operators for creating semirings with Boolean monoids.

\vspace{0.2in}
{\footnotesize
\begin{tabular}{|llll|}
\hline
\multicolumn{4}{|c|}{Binary comparators for all 13 types} \\
\hline
GraphBLAS name        & types (domains)            & $z=f(x,y)$      & description \\
\hline
% 6 TxT -> bool comparators
\verb'GrB_EQ_'$T$     & $T \times T \rightarrow $\verb'bool' & $z = (x == y)$  & equal \\
\verb'GrB_NE_'$T$     & $T \times T \rightarrow $\verb'bool' & $z = (x \ne y)$ & not equal \\
\hline
\multicolumn{4}{ }{\mbox{ }} \\
\hline
\multicolumn{4}{|c|}{Binary comparators for non-complex types} \\
\hline
GraphBLAS name        & types (domains)            & $z=f(x,y)$      & description \\
\hline
\verb'GrB_GT_'$R$     & $R \times R \rightarrow $\verb'bool' & $z = (x >   y)$ & greater than \\
\verb'GrB_LT_'$R$     & $R \times R \rightarrow $\verb'bool' & $z = (x <   y)$ & less than  \\
\verb'GrB_GE_'$R$     & $R \times R \rightarrow $\verb'bool' & $z = (x \ge y)$ & greater than or equal \\
\verb'GrB_LE_'$R$     & $R \times R \rightarrow $\verb'bool' & $z = (x \le y)$ & less than or equal  \\
\hline
\end{tabular}
}
\vspace{0.2in}

GraphBLAS has four built-in binary operators that operate purely in
the Boolean domain.  The first three are identical to the \verb'GxB_L*_BOOL'
operators described above, just with a shorter name.  The \verb'GrB_LXNOR'
operator is the same as \verb'GrB_EQ_BOOL'.

\vspace{0.2in}
{\footnotesize
\begin{tabular}{|llll|}
\hline
\multicolumn{4}{|c|}{Binary operators for the boolean type only} \\
\hline
GraphBLAS name        & types (domains)            & $z=f(x,y)$      & description \\
\hline
% 3 bool x bool -> bool
\verb'GrB_LOR'        & \verb'bool'
                        $\times$ \verb'bool'
                        $\rightarrow$ \verb'bool'  & $z = x \vee    y $ & logical OR \\
\verb'GrB_LAND'       & \verb'bool'
                        $\times$ \verb'bool'
                        $\rightarrow$ \verb'bool'  & $z = x \wedge  y $ & logical AND \\
\verb'GrB_LXOR'       & \verb'bool'
                        $\times$ \verb'bool'
                        $\rightarrow$ \verb'bool'  & $z = x \veebar y $ & logical XOR \\
\verb'GrB_LXNOR'      & \verb'bool'
                        $\times$ \verb'bool'
                        $\rightarrow$ \verb'bool'  & $z = \lnot (x \veebar y) $ & logical XNOR \\
\hline
\end{tabular}
}
\vspace{0.2in}

The following operators are defined for real floating-point types only (\verb'GrB_FP32' and  \verb'GrB_FP64').
They are identical to the C11 functions of the same name.  The last one in the table constructs
the corresponding complex type.

\vspace{0.2in}
{\footnotesize
\begin{tabular}{|llll|}
\hline
\multicolumn{4}{|c|}{Binary operators for the real floating-point types only} \\
\hline
GraphBLAS name        & types (domains)            & $z=f(x,y)$      & description \\
\hline
\verb'GxB_ATAN2_'$F$     & $F \times F \rightarrow F$ & $z = \tan^{-1}(y/x)$ & 4-quadrant arc tangent  \\
\verb'GxB_HYPOT_'$F$     & $F \times F \rightarrow F$ & $z = \sqrt{x^2+y^2}$ & hypotenuse \\
\verb'GxB_FMOD_'$F$      & $F \times F \rightarrow F$ &                      & C11 \verb'fmod' \\
\verb'GxB_REMAINDER_'$F$ & $F \times F \rightarrow F$ &                      & C11 \verb'remainder' \\
\verb'GxB_LDEXP_'$F$     & $F \times F \rightarrow F$ &                      & C11 \verb'ldexp' \\
\verb'GxB_COPYSIGN_'$F$  & $F \times F \rightarrow F$ &                      & C11 \verb'copysign' \\
\hline
\verb'GxB_CMPLX_'$F$     & $F \times F \rightarrow Z$ & $z = x + y \times i$ & complex from real \& imag \\
\hline
\end{tabular}
}
\vspace{0.2in}

Eight bitwise operators are predefined for signed and unsigned integers.

\vspace{0.2in}
{\footnotesize
\begin{tabular}{|llll|}
\hline
\multicolumn{4}{|c|}{Binary operators for signed and unsigned integers} \\
\hline
GraphBLAS name        & types (domains)            & $z=f(x,y)$      & description \\
\hline
\verb'GrB_BOR_'$I$    & $I \times I \rightarrow I$ & \verb'z=x|y'    & bitwise logical OR \\
\verb'GrB_BAND_'$I$   & $I \times I \rightarrow I$ & \verb'z=x&y'    & bitwise logical AND \\
\verb'GrB_BXOR_'$I$   & $I \times I \rightarrow I$ & \verb'z=x^y'    & bitwise logical XOR \\
\verb'GrB_BXNOR_'$I$  & $I \times I \rightarrow I$ & \verb'z=~(x^y)' & bitwise logical XNOR \\
\hline
\verb'GxB_BGET_'$I$    & $I \times I \rightarrow I$  & & get bit y of x \\
\verb'GxB_BSET_'$I$    & $I \times I \rightarrow I$  & & set bit y of x \\
\verb'GxB_BCLR_'$I$    & $I \times I \rightarrow I$  & & clear bit y of x \\
\verb'GxB_BSHIFT_'$I$  & $I \times $\verb'int8'$  \rightarrow I$ & & bit shift \\
\hline
\end{tabular}
}
\vspace{0.2in}

There are two sets of built-in comparators in SuiteSparse:Graph\-BLAS,
but they are not redundant.  They are identical except for the type (domain) of
their output, $z$.  The \verb'GrB_EQ_'$T$ and related operators compare their
inputs of type $T$ and produce a Boolean result of true or false.  The
\verb'GxB_ISEQ_'$T$ and related operators compute the same thing and produce a
result with same type $T$ as their input operands, returning one for true or
zero for false.  The \verb'IS*' comparators are useful when combining
comparators with other non-Boolean operators.  For example, a \verb'PLUS-ISEQ'
semiring counts how many terms are true.  With this semiring,
matrix multiplication ${\bf C=AB}$ for two weighted undirected graphs ${\bf A}$
and ${\bf B}$ computes $c_{ij}$ as the number of edges node $i$ and $j$ have in
common that have identical edge weights.  Since the output type of the
``multiplier'' operator in a semiring must match the type of its monoid, the
Boolean \verb'EQ' cannot be combined with a non-Boolean \verb'PLUS' monoid to
perform this operation.

Likewise, SuiteSparse:GraphBLAS has two sets of logical OR, AND, and XOR
operators.  Without the \verb'_'$T$ suffix, the three operators \verb'GrB_LOR',
\verb'GrB_LAND', and \verb'GrB_LXOR' operate purely in the Boolean domain,
where all input and output types are \verb'GrB_BOOL'.  The second set
(\verb'GxB_LOR_'$T$ \verb'GxB_LAND_'$T$ and \verb'GxB_LXOR_'$T$) provides
Boolean operators to all 11 real domains, implicitly typecasting their inputs from
type $T$ to Boolean and returning a value of type $T$ that is 1 for true or
zero for false.  The set of \verb'GxB_L*_'$T$ operators are useful since they
can be combined with non-Boolean monoids in a semiring.

Floating-point operations follow the IEEE 754 standard.  Thus, computing $x/0$
for a floating-point $x$ results in \verb'+Inf' if $x$ is positive, \verb'-Inf'
if $x$ is negative, and \verb'NaN' if $x$ is zero.  The application is not
terminated.  However, integer division by zero normally terminates an
application.  SuiteSparse:GraphBLAS avoids this by adopting the same rules as
MATLAB, which are analogous to how the IEEE standard handles floating-point
division by zero.  For integers, when $x$ is positive, $x/0$ is the largest
positive integer, for negative $x$ it is the minimum integer, and 0/0 results
in zero.  For example, for an integer $x$ of type \verb'GrB_INT32', 1/0 is
$2^{31}-1$ and (-1)/0 is $-2^{31}$.  Refer to Section~\ref{type} for a list of
integer ranges.

%===============================================================================
\subsubsection{GraphBLAS binary operators based on index binary operators}
%===============================================================================

Eight binary operators based on underlying index binary operators are
predefined.  They differ when used in a semiring and when used in
\verb'GrB_eWise*' and \verb'GrB_apply'.  These index-based binary operators
cannot be used in \verb'GrB_build', nor can they be used as the \verb'accum'
operator for any operation.

The built-in index-based binary operators do not depend on the type or numerical value
of their inputs, just their position in a matrix or vector.  For a vector, $j$
is always 0, and $i$ is the index into the vector.  There are two types $N$
available: \verb'INT32' and \verb'INT64', which is the type of the output $z$.
User-defined index-based operators are not defined by \verb'GrB_BinaryOp_new',
but by \verb'GxB_BinaryOp_new_IndexOp' instead.  See Section~\ref{idxbinaryop}
for details.

\vspace{0.2in}
{\footnotesize
\begin{tabular}{|llll|}
\hline
\multicolumn{4}{|c|}{Index-based binary operators for any type (including user-defined)} \\
\multicolumn{4}{|c|}{when used as a multiplicative operator in a semiring} \\
\hline
GraphBLAS name            & types (domains)   & $z=f(a_{ik},b_{kj})$      & description \\
\hline
\verb'GxB_FIRSTI_'$N$    & $ \rightarrow N$  & $z = i$       & row index of $a_{ik}$ (0-based) \\
\verb'GxB_FIRSTI1_'$N$   & $ \rightarrow N$  & $z = i+1$     & row index of $a_{ik}$ (1-based) \\
\verb'GxB_FIRSTJ_'$N$    & $ \rightarrow N$  & $z = k$       & column index of $a_{ik}$ (0-based) \\
\verb'GxB_FIRSTJ1_'$N$   & $ \rightarrow N$  & $z = k+1$     & column index of $a_{ik}$ (1-based) \\
\verb'GxB_SECONDI_'$N$   & $ \rightarrow N$  & $z = k$       & row index of $b_{kj}$ (0-based) \\
\verb'GxB_SECONDI1_'$N$  & $ \rightarrow N$  & $z = k+1$     & row index of $b_{kj}$ (1-based) \\
\verb'GxB_SECONDJ_'$N$   & $ \rightarrow N$  & $z = j$       & column index of $b_{kj}$ (0-based) \\
\verb'GxB_SECONDJ1_'$N$  & $ \rightarrow N$  & $z = j+1$     & column index of $b_{kj}$ (1-based) \\
\hline
\end{tabular}
}

\vspace{0.2in}
{\footnotesize
\begin{tabular}{|llll|}
\hline
\multicolumn{4}{|c|}{Index-based binary operators for any type (including user-defined)} \\
\multicolumn{4}{|c|}{when used in all other methods} \\
\hline
GraphBLAS name            & types (domains)   & $z=f(a_{ij},b_{ij})$      & description \\
\hline
\verb'GxB_FIRSTI_'$N$    & $ \rightarrow N$  & $z = i$       & row index of $a_{ij}$ (0-based) \\
\verb'GxB_FIRSTI1_'$N$   & $ \rightarrow N$  & $z = i+1$     & row index of $a_{ij}$ (1-based) \\
\verb'GxB_FIRSTJ_'$N$    & $ \rightarrow N$  & $z = j$       & column index of $a_{ij}$ (0-based) \\
\verb'GxB_FIRSTJ1_'$N$   & $ \rightarrow N$  & $z = j+1$     & column index of $a_{ij}$ (1-based) \\
\verb'GxB_SECONDI_'$N$   & $ \rightarrow N$  & $z = i$       & row index of $b_{ij}$ (0-based) \\
\verb'GxB_SECONDI1_'$N$  & $ \rightarrow N$  & $z = i+1$     & row index of $b_{ij}$ (1-based) \\
\verb'GxB_SECONDJ_'$N$   & $ \rightarrow N$  & $z = j$       & column index of $b_{ij}$ (0-based) \\
\verb'GxB_SECONDJ1_'$N$  & $ \rightarrow N$  & $z = j+1$     & column index of $b_{ij}$ (1-based) \\
\hline
\end{tabular}
}
\vspace{0.2in}

Finally, one special binary operator can only be used as input to
\verb'GrB_Matrix_build' or \verb'GrB_Vector_build': the \verb'GxB_IGNORE_DUP'
operator.  If \verb'dup' is \verb'NULL', any duplicates in the \verb'GrB*build'
methods result in an error.  If \verb'dup' is the special binary operator
\verb'GxB_IGNORE_DUP', then any duplicates are ignored.  If duplicates appear,
the last one in the list of tuples is taken and the prior ones ignored.  This
is not an error.

The next sections define the following methods for the \verb'GrB_BinaryOp'
object:

\vspace{0.2in}
{\footnotesize
\begin{tabular}{lll}
\hline
GraphBLAS function   & purpose                                      & Section \\
\hline
\verb'GrB_BinaryOp_new'   & create a user-defined binary operator   & \ref{binaryop_new} \\
\verb'GxB_BinaryOp_new'   & create a named user-defined binary operator   & \ref{binaryop_new_named} \\
\verb'GrB_BinaryOp_wait'  & wait for a user-defined binary operator & \ref{binaryop_wait} \\
\verb'GrB_BinaryOp_free'  & free a user-defined binary operator     & \ref{binaryop_free} \\
\verb'GrB_get'           & get properties of an operator    & \ref{get_set_binop} \\
\verb'GrB_set'           & set the operator name/definition & \ref{get_set_binop} \\
\hline
\end{tabular}
}
\vspace{0.2in}

\newpage
%-------------------------------------------------------------------------------
\subsubsection{{\sf GrB\_BinaryOp\_new:} create a user-defined binary operator}
%-------------------------------------------------------------------------------
\label{binaryop_new}

\begin{mdframed}[userdefinedwidth=6in]
{\footnotesize
\begin{verbatim}
GrB_Info GrB_BinaryOp_new
(
    GrB_BinaryOp *binaryop,         // handle for the new binary operator
    void *function,                 // pointer to the binary function
    GrB_Type ztype,                 // type of output z
    GrB_Type xtype,                 // type of input x
    GrB_Type ytype                  // type of input y
) ;
\end{verbatim}
}\end{mdframed}

\verb'GrB_BinaryOp_new' creates a new binary operator.  The new operator is
returned in the \verb'binaryop' handle, which must not be \verb'NULL' on input.
On output, its contents contains a pointer to the new binary operator.

The three types \verb'xtype', \verb'ytype', and \verb'ztype' are the GraphBLAS
types of the inputs $x$ and $y$, and output $z$ of the user-defined function
$z=f(x,y)$.  These types may be built-in types or user-defined types, in any
combination.  The three types need not be the same, but they must be previously
defined before passing them to \verb'GrB_BinaryOp_new'.

The final argument to \verb'GrB_BinaryOp_new' is a pointer to a user-defined
function with the following signature:

    {\footnotesize
    \begin{verbatim}
    void (*f) (void *z, const void *x, const void *y) ; \end{verbatim} }

When the function \verb'f' is called, the arguments \verb'z', \verb'x', and
\verb'y' are passed as \verb'(void *)' pointers, but they will be pointers to
values of the correct type, defined by \verb'ztype', \verb'xtype', and
\verb'ytype', respectively, when the operator was created.

{\bf NOTE:} SuiteSparse:GraphBLAS may call the function with the pointers
\verb'z' and \verb'x' equal to one another, in which case \verb'z=f(z,y)'
should be computed.  Future versions may use additional pointer aliasing.

\newpage
%-------------------------------------------------------------------------------
\subsubsection{{\sf GxB\_BinaryOp\_new:} create a named user-defined binary operator}
%-------------------------------------------------------------------------------
\label{binaryop_new_named}

\begin{mdframed}[userdefinedwidth=6in]
{\footnotesize
\begin{verbatim}
GrB_Info GxB_BinaryOp_new
(
    GrB_BinaryOp *op,               // handle for the new binary operator
    GxB_binary_function function,   // pointer to the binary function
    GrB_Type ztype,                 // type of output z
    GrB_Type xtype,                 // type of input x
    GrB_Type ytype,                 // type of input y
    const char *binop_name,         // name of the user function
    const char *binop_defn          // definition of the user function
) ;
\end{verbatim} }\end{mdframed}

Creates a named \verb'GrB_BinaryOp'.  Only the first 127 characters of
\verb'binop_name' are used.  The \verb'binop_defn' is a string containing the
entire function itself.  For example:

{\footnotesize
\begin{verbatim}
void absdiff (double *z, double *x, double *y) { (*z) = fabs ((*x) - (*y)) ; } ;
...
GrB_Type AbsDiff ;
GxB_BinaryOp_new (&AbsDiff, absdiff, GrB_FP64, GrB_FP64, GrB_FP64, "absdiff",
  "void absdiff (double *z, double *x, double *y) { (*z) = fabs ((*x) - (*y)) ; }") ; \end{verbatim}}

The two strings \verb'binop_name' and \verb'binop_defn' are optional, but are
required to enable the JIT compilation of kernels that use this operator.

If the JIT is enabled, or if the corresponding JIT kernel has been copied
into the \verb'PreJIT' folder, the \verb'function' may be \verb'NULL'.  In this
case, a JIT kernel is compiled that contains just the user-defined function.
If the JIT is disabled and the \verb'function' is \verb'NULL', this method
returns \verb'GrB_NULL_POINTER'.

The above example is identical to the following usage, except that
\verb'GrB_BinaryOp_new' requires a non-NULL function pointer.

{\footnotesize
\begin{verbatim}
void absdiff (double *z, double *x, double *y) { (*z) = fabs ((*x) - (*y)) ; } ;
...
GrB_Type AbsDiff ;
GrB_BinaryOp_new (&AbsDiff, absdiff, GrB_FP64, GrB_FP64, GrB_FP64) ;
GrB_set (AbsDiff, "absdiff", GxB_JIT_C_NAME) ;
GrB_set (AbsDiff,
  "void absdiff (double *z, double *x, double *y) { (*z) = fabs ((*x) - (*y)) ; }",
  GxB_JIT_C_DEFINITION) ;\end{verbatim}}

%-------------------------------------------------------------------------------
\subsubsection{{\sf GrB\_BinaryOp\_wait:} wait for a binary operator}
%-------------------------------------------------------------------------------
\label{binaryop_wait}

\begin{mdframed}[userdefinedwidth=6in]
{\footnotesize
\begin{verbatim}
GrB_Info GrB_wait               // wait for a user-defined binary operator
(
    GrB_BinaryOp binaryop,      // binary operator to wait for
    int mode                    // GrB_COMPLETE or GrB_MATERIALIZE
) ;
\end{verbatim}
}\end{mdframed}

After creating a user-defined binary operator, a GraphBLAS library may choose
to exploit non-blocking mode to delay its creation.  Currently,
SuiteSparse:GraphBLAS currently does nothing for except to ensure that the
\verb'binaryop' is valid.

%-------------------------------------------------------------------------------
\subsubsection{{\sf GrB\_BinaryOp\_free:} free a user-defined binary operator}
%-------------------------------------------------------------------------------
\label{binaryop_free}

\begin{mdframed}[userdefinedwidth=6in]
{\footnotesize
\begin{verbatim}
GrB_Info GrB_free                   // free a user-created binary operator
(
    GrB_BinaryOp *binaryop          // handle of binary operator to free
) ;
\end{verbatim}
} \end{mdframed}

\verb'GrB_BinaryOp_free' frees a user-defined binary operator.
Either usage:

    {\small
    \begin{verbatim}
    GrB_BinaryOp_free (&op) ;
    GrB_free (&op) ; \end{verbatim}}

\noindent
frees the \verb'op' and sets \verb'op' to \verb'NULL'.
It safely does nothing if passed a \verb'NULL'
handle, or if \verb'op == NULL' on input.
It does nothing at all if passed a built-in binary operator.

%-------------------------------------------------------------------------------
\subsubsection{{\sf ANY} and {\sf PAIR} ({\sf ONEB}) operators}
%-------------------------------------------------------------------------------
\label{any_pair}

The \verb'GxB_PAIR' operator (also called \verb'GrB_ONEB') is simple to describe:
just $f(x,y)=1$.  It is called
the \verb'PAIR' operator since it returns $1$ in a semiring when a pair of
entries $a_{ik}$ and $b_{kj}$ is found in the matrix multiply.  This operator
is simple yet very useful.  It allows purely structural computations to be
performed on matrices of any type, without having to typecast them to Boolean
with all values being true.  Typecasting need not be performed on the inputs to
the \verb'PAIR' operator, and the \verb'PAIR' operator does not need to access
the values of the matrix.  This cuts memory accesses, so it is a very fast
operator to use.

The \verb'GxB_PAIR_T' operator is a SuiteSparse:GraphBLAS extension.
It has since been added to the v2.0 C API Specification as \verb'GrB_ONEB_T'.
They are identical, but the latter name should be used for compatibility
with other GraphBLAS libraries.

The \verb'ANY' operator is very unusual, but very powerful.  It is the function
$f_{\mbox{any}}(x,y)=x$, or $y$, where GraphBLAS has to freedom to select
either $x$, or $y$, at its own discretion.  Do not confuse the \verb'ANY'
operator with the \verb'any' function in MATLAB/Octave, which computes a reduction
using the logical OR operator.

The \verb'ANY' function is associative and commutative, and can thus serve as
an operator for a monoid.  The selection of $x$ are $y$ is not randomized.
Instead, SuiteSparse:GraphBLAS uses this freedom to compute as fast a result as
possible.  When used as the monoid in a dot product, \[ c_{ij} = \sum_k a_{ik}
b_{kj} \] for example, the computation can terminate as soon as any matching
pair of entries is found.  When used in a parallel saxpy-style computation, the
\verb'ANY' operator allows for a relaxed form of synchronization to be used,
resulting in a fast benign race condition.

Because of this benign race condition, the result of the \verb'ANY' monoid can
be non-deterministic, unless it is coupled with the \verb'PAIR' multiplicative
operator.  In this case, the \verb'ANY_PAIR' semiring will return a
deterministic result, since $f_{\mbox{any}}(1,1)$ is always 1.

When paired with a different operator, the results are non-deterministic.  This
gives a powerful method when computing results for which any value selected by
the \verb'ANY' operator is valid.  One such example is the breadth-first-search
tree.  Suppose node $j$ is at level $v$, and there are multiple nodes $i$ at
level $v-1$ for which the edge $(i,j)$ exists in the graph.  Any of these nodes
$i$ can serve as a valid parent in the BFS tree.  Using the \verb'ANY'
operator, GraphBLAS can quickly compute a valid BFS tree; if it used again on
the same inputs, it might return a different, yet still valid, BFS tree, due to
the non-deterministic nature of intra-thread synchronization.



\newpage
%===============================================================================
\subsection{GraphBLAS IndexUnaryOp operators: {\sf GrB\_IndexUnaryOp}} %========
%===============================================================================
\label{idxunop}

An index-unary operator is a scalar function of the form
$z=f(a_{ij},i,j,y)$ that is applied to the entries $a_{ij}$ of an
$m$-by-$n$ matrix.  It can be used in \verb'GrB_apply' (Section~\ref{apply}) or
in \verb'GrB_select' (Section~\ref{select}) to select entries from a matrix or
vector.

The signature of the index-unary function \verb'f' is as follows:

{\footnotesize
\begin{verbatim}
void f
(
    void *z,            // output value z, of type ztype
    const void *x,      // input value x of type xtype; value of v(i) or A(i,j)
    GrB_Index i,        // row index of A(i,j)
    GrB_Index j,        // column index of A(i,j), or zero for v(i)
    const void *y       // input scalar y of type ytype
) ; \end{verbatim}}

The following built-in operators are available.  Operators that do not depend
on the value of \verb'A(i,j)' can be used on any matrix or vector, including
those of user-defined type.  In the table, \verb'y' is a
scalar whose type matches the suffix of the operator.  The \verb'VALUEEQ' and
\verb'VALUENE' operators are defined for any built-in type. The other
\verb'VALUE' operators are defined only for real (not complex) built-in types.
Any index computations are done in \verb'int64_t' arithmetic; the result is
typecasted to \verb'int32_t' for the \verb'*INDEX_INT32' operators.

\vspace{0.2in}
\noindent
{\footnotesize
\begin{tabular}{lll}
\hline
GraphBLAS name          & MATLAB/Octave     & description \\
                        & analog            & \\
\hline
\verb'GrB_ROWINDEX_INT32'  & \verb'z=i+y'       & row index of \verb'A(i,j)', as int32 \\
\verb'GrB_ROWINDEX_INT64'  & \verb'z=i+y'       & row index of \verb'A(i,j)', as int64 \\
\verb'GrB_COLINDEX_INT32'  & \verb'z=j+y'       & column index of \verb'A(i,j)', as int32 \\
\verb'GrB_COLINDEX_INT64'  & \verb'z=j+y'       & column index of \verb'A(i,j)', as int64 \\
\verb'GrB_DIAGINDEX_INT32' & \verb'z=j-(i+y)'   & column diagonal index of \verb'A(i,j)', as int32 \\
\verb'GrB_DIAGINDEX_INT64' & \verb'z=j-(i+y)'   & column diagonal index of \verb'A(i,j)', as int64 \\
\hline
\verb'GrB_TRIL'    & \verb'z=(j<=(i+y))'  & true for entries on or below the \verb'y'th diagonal \\
\verb'GrB_TRIU'    & \verb'z=(j>=(i+y))'  & true for entries on or above the \verb'y'th diagonal \\
\verb'GrB_DIAG'    & \verb'z=(j==(i+y))'  & true for entries on the \verb'y'th diagonal \\
\verb'GrB_OFFDIAG' & \verb'z=(j!=(i+y))'  & true for entries not on the \verb'y'th diagonal \\
\verb'GrB_COLLE'   & \verb'z=(j<=y)'      & true for entries in columns 0 to \verb'y' \\
\verb'GrB_COLGT'   & \verb'z=(j>y)'       & true for entries in columns \verb'y+1' and above \\
\verb'GrB_ROWLE'   & \verb'z=(i<=y)'      & true for entries in rows 0 to \verb'y' \\
\verb'GrB_ROWGT'   & \verb'z=(i>y)'       & true for entries in rows \verb'y+1' and above \\
\hline
\verb'GrB_VALUENE_T'     & \verb'z=(aij!=y)'    & true if \verb'A(i,j)' is not equal to \verb'y'\\
\verb'GrB_VALUEEQ_T'     & \verb'z=(aij==y)'    & true if \verb'A(i,j)' is equal to \verb'y'\\
\verb'GrB_VALUEGT_T'     & \verb'z=(aij>y)'     & true if \verb'A(i,j)' is greater than \verb'y' \\
\verb'GrB_VALUEGE_T'     & \verb'z=(aij>=y)'    & true if \verb'A(i,j)' is greater than or equal to \verb'y' \\
\verb'GrB_VALUELT_T'     & \verb'z=(aij<y)'     & true if \verb'A(i,j)' is less than \verb'y' \\
\verb'GrB_VALUELE_T'     & \verb'z=(aij<=y)'    & true if \verb'A(i,j)' is less than or equal to \verb'y' \\
%
\hline
\end{tabular}
}
\vspace{0.2in}


The following methods operate on the \verb'GrB_IndexUnaryOp' object:

\vspace{0.1in}
\noindent
{\footnotesize
\begin{tabular}{lll}
\hline
GraphBLAS function   & purpose                                      & Section \\
\hline
\verb'GrB_IndexUnaryOp_new'   & create a user-defined index-unary operator   & \ref{idxunop_new} \\
\verb'GxB_IndexUnaryOp_new'   & create a named user-defined index-unary operator   & \ref{idxunop_new_named} \\
\verb'GrB_IndexUnaryOp_wait'  & wait for a user-defined index-unary operator  & \ref{idxunop_wait} \\
\verb'GrB_IndexUnaryOp_free'  & free a user-defined index-unary operator      & \ref{idxunop_free} \\
\verb'GrB_get'           & get properties of an operator    & \ref{get_set_idxunop} \\
\verb'GrB_set'           & set the operator name/definition & \ref{get_set_idxunop} \\
\hline
\end{tabular}
}
\vspace{0.1in}

\newpage
%-------------------------------------------------------------------------------
\subsubsection{{\sf GrB\_IndexUnaryOp\_new:} create a user-defined index-unary operator}
%-------------------------------------------------------------------------------
\label{idxunop_new}

\begin{mdframed}[userdefinedwidth=6in]
{\footnotesize
\begin{verbatim}
GrB_Info GrB_IndexUnaryOp_new       // create a new user-defined IndexUnary op
(
    GrB_IndexUnaryOp *op,           // handle for the new IndexUnary operator
    void *function,                 // pointer to IndexUnary function
    GrB_Type ztype,                 // type of output z
    GrB_Type xtype,                 // type of input x (the A(i,j) entry)
    GrB_Type ytype                  // type of scalar input y
) ;
\end{verbatim} }\end{mdframed}


\verb'GrB_IndexUnaryOp_new' creates a new index-unary operator.  The new operator is
returned in the \verb'op' handle, which must not be \verb'NULL' on input.
On output, its contents contains a pointer to the new index-unary operator.

The \verb'function' argument to \verb'GrB_IndexUnaryOp_new' is a pointer to a
user-defined function whose signature is given at the beginning of
Section~\ref{idxunop}.  Given the properties of an entry $a_{ij}$ in a
matrix, the \verb'function' should return \verb'z' as \verb'true' if the entry
should be kept in the output of \verb'GrB_select', or \verb'false' if it should
not appear in the output.  If the return value is not \verb'GrB_BOOL',
it is typecasted to \verb'GrB_BOOL' by \verb'GrB_select'.

The type \verb'xtype' is the GraphBLAS type of the input $x$ of the
user-defined function $z=f(x,i,j,y)$, which is used for the
entry \verb'A(i,j)' of a matrix or \verb'v(i)' of a vector.  The type may be
built-in or user-defined.

The type \verb'ytype' is the GraphBLAS type of the scalar input $y$ of the
user-defined function $z=f(x,i,j,y)$.  The type may be built-in
or user-defined.

\newpage
%-------------------------------------------------------------------------------
\subsubsection{{\sf GxB\_IndexUnaryOp\_new:} create a named user-defined index-unary operator}
%-------------------------------------------------------------------------------
\label{idxunop_new_named}

\begin{mdframed}[userdefinedwidth=6in]
{\footnotesize
\begin{verbatim}
GrB_Info GxB_IndexUnaryOp_new   // create a named user-created IndexUnaryOp
(
    GrB_IndexUnaryOp *op,           // handle for the new IndexUnary operator
    GxB_index_unary_function function,    // pointer to index_unary function
    GrB_Type ztype,                 // type of output z
    GrB_Type xtype,                 // type of input x
    GrB_Type ytype,                 // type of scalar input y
    const char *idxop_name,         // name of the user function
    const char *idxop_defn          // definition of the user function
) ;
\end{verbatim} }\end{mdframed}

Creates a named \verb'GrB_IndexUnaryOp'.  Only the first 127 characters of
\verb'idxop_name' are used.  The \verb'ixdop_defn' is a string containing the
entire function itself.

The two strings \verb'idxop_name' and \verb'idxop_defn' are optional, but are
required to enable the JIT compilation of kernels that use this operator.
The strings can also be set the \verb'GrB_set' after the operator is created
with \verb'GrB_IndexUnaryOp_new'.  For example:

{\footnotesize
\begin{verbatim}
    void banded_idx
    (
        bool *z,
        const int64_t *x,   // unused
        int64_t i,
        int64_t j,
        const int64_t *thunk
    )
    {
        // d = abs (j-i)
        int64_t d = j-i ;
        if (d < 0) d = -d ;
        (*z) = (d <= *thunk) ;
    }

    #define BANDED_IDX_DEFN                     \
    "void banded_idx                        \n" \
    "(                                      \n" \
    "    bool *z,                           \n" \
    "    const int64_t *x,   // unused      \n" \
    "    int64_t i,                         \n" \
    "    int64_t j,                         \n" \
    "    const int64_t *thunk               \n" \
    ")                                      \n" \
    "{                                      \n" \
    "    int64_t d = j-i ;                  \n" \
    "    if (d < 0) d = -d ;                \n" \
    "    (*z) = (d <= *thunk) ;             \n" \
    "}"

    GxB_IndexUnaryOp_new (&Banded,
        (GxB_index_unary_function) banded_idx,
        GrB_BOOL, GrB_INT64, GrB_INT64,
        "banded_idx", BANDED_IDX_DEFN)) ;\end{verbatim}}

If JIT compilation is enabled, or if the corresponding JIT kernel has been
copied into the \verb'PreJIT' folder, the \verb'function' may be \verb'NULL'.
In this case, a JIT kernel is compiled that contains just the user-defined
function.  If the JIT is disabled and the \verb'function' is \verb'NULL', this
method returns \verb'GrB_NULL_POINTER'.

The above example is identical to the following usage
except that \verb'GrB_IndexUnaryOp_new' requires a non-NULL function pointer.
The \verb'banded_idx' function is defined the same as above.

{\footnotesize
\begin{verbatim}
    void banded_idx ... see above
    #define BANDED_IDX_DEFN  ... see above

    GrB_IndexUnaryOp_new (&Banded,
        (GxB_index_unary_function) banded_idx,
        GrB_BOOL, GrB_INT64, GrB_INT64) ;
    GrB_set (Banded, "banded_idx", GxB_JIT_C_NAME)) ;
    GrB_set (Banded, BANDED_IDX_DEFN, GxB_JIT_C_DEFINITION)) ;\end{verbatim}}

%-------------------------------------------------------------------------------
\subsubsection{{\sf GrB\_IndexUnaryOp\_wait:} wait for an index-unary operator}
%-------------------------------------------------------------------------------
\label{idxunop_wait}

\begin{mdframed}[userdefinedwidth=6in]
{\footnotesize
\begin{verbatim}
GrB_Info GrB_wait               // wait for a user-defined binary operator
(
    GrB_IndexUnaryOp op,        // index-unary operator to wait for
    int mode                    // GrB_COMPLETE or GrB_MATERIALIZE
) ;
\end{verbatim}
}\end{mdframed}

After creating a user-defined index-unary operator, a GraphBLAS library may choose
to exploit non-blocking mode to delay its creation.  Currently,
SuiteSparse:GraphBLAS currently does nothing except to ensure that the
\verb'op' is valid.

\newpage
%-------------------------------------------------------------------------------
\subsubsection{{\sf GrB\_IndexUnaryOp\_free:} free a user-defined index-unary operator}
%-------------------------------------------------------------------------------
\label{idxunop_free}

\begin{mdframed}[userdefinedwidth=6in]
{\footnotesize
\begin{verbatim}
GrB_Info GrB_free               // free a user-created index-unary operator
(
    GrB_IndexUnaryOp *op        // handle of IndexUnary to free
) ;
\end{verbatim}
}\end{mdframed}

\verb'GrB_IndexUnaryOp_free' frees a user-defined index-unary operator.  Either usage:

    {\small
    \begin{verbatim}
    GrB_IndexUnaryOp_free (&op) ;
    GrB_free (&op) ; \end{verbatim}}

\noindent
frees the \verb'op' and sets \verb'op' to \verb'NULL'.  It safely
does nothing if passed a \verb'NULL' handle, or if \verb'op == NULL' on
input.  It does nothing at all if passed a built-in index-unary operator.



\newpage
%===============================================================================
\subsection{GraphBLAS index-binary operators: {\sf GxB\_IndexBinaryOp}}
%===============================================================================
\label{idxbinaryop}

An index-binary operator is a scalar function of the following form:
\[
z=f(x,i_x,j_x,y,i_y,j_y,\Theta),
\]
where the value $x$ appears at row $i_x$ and column $j_x$ in its matrix,
and the value $y$ appears at row $i_y$ and column $j_y$ in its matrix.
The value $\Theta$ is a scalar that is the same for all uses of the operator.
See our IEEE HPEC'24 paper for more details (\cite{idxbinop}),
in the \verb'GraphBLAS/Doc' folder.

When used in an element-wise method for $\bf C = A \oplus B$ and related
methods (\verb'GrB_eWiseAdd', \verb'GxB_eWiseUnion', or \verb'GrB_eWiseMult'),
operator is used for a pair of entries
$a_{ij}$ and $b_{ij}$, as
\[
z=f(a_{ij},i,j,b_{ij},i,j,\Theta).
\]
When used in \verb'GrB_kronecker', it is used on a pair of entries
$a_{i_1,j_1}$ and $b_{i_2,j_2}$, as
\[
z=f(a_{ij},i_1,j_1,b_{ij},i_2,j_2,\Theta).
\]
When used as the multiplicative operator in a semiring, to compute
$\bf C = A \oplus.\otimes B$, the operator is used as
\[
z=f(a_{ik},i,k,b_{kj},k,j,\Theta)
\]
to compute an entry to be summed by the monoid of the semiring.

No GraphBLAS operations directly use the \verb'GxB_IndexBinaryOp'.  Instead,
the operator is coupled with a scalar \verb'Theta' value to create a new
index-based binary operator, which is simply a special case of a
\verb'GrB_BinaryOp'.  The resulting \verb'GrB_BinaryOp' can then be passed to
element-wise methods and as the multiplicative operator of a new semiring.

The signature of the index-binary function \verb'f' is as follows:

{\footnotesize
\begin{verbatim}
void f
(
    void *z,            // output value z, of type ztype
    const void *x,      // input value x of type xtype; value of v(ix) or A(ix,jx)
    GrB_Index ix,       // row index of v(ix) or A(ix,jx)
    GrB_Index jx,       // column index of A(ix,jx), or zero for v(ix)
    const void *y,      // input value y of type ytype; value of w(iy) or B(iy,jy)
    GrB_Index iy,       // row index of w(iy) or B(iy,jy)
    GrB_Index jy,       // column index of B(iy,jy), or zero for w(iy)
    const void *theta   // input scalar theta of type theta_type
) ; \end{verbatim}}

The following binary operators (\verb'GrB_BinaryOp' objects) are pre-defined,
where $N$ can be \verb'INT32' or \verb'INT64'.  These operators do not use
\verb'theta'.  Instead, the offset of 1 in \verb'GxB_FIRSTI1' is fixed into
the operator itself.

\vspace{0.2in}
{\footnotesize
\begin{tabular}{|llll|}
\hline
\multicolumn{4}{|c|}{Built-in index-based binary operators for any type} \\
\hline
GraphBLAS name            & types (domains)  & $z=f(x,y)$    & description \\
\hline
\verb'GxB_FIRSTI_'$N$    & $ \rightarrow N$  & $z = i_x$   & row index of $x$ (0-based) \\
\verb'GxB_FIRSTI1_'$N$   & $ \rightarrow N$  & $z = i_x+1$ & row index of $x$ (1-based) \\
\verb'GxB_FIRSTJ_'$N$    & $ \rightarrow N$  & $z = j_x$   & column index of $x$ (0-based) \\
\verb'GxB_FIRSTJ1_'$N$   & $ \rightarrow N$  & $z = j_x+1$ & column index of $x$ (1-based) \\
\verb'GxB_SECONDI_'$N$   & $ \rightarrow N$  & $z = i_y$   & row index of $y$ (0-based) \\
\verb'GxB_SECONDI1_'$N$  & $ \rightarrow N$  & $z = i_y+1$ & row index of $y$ (1-based) \\
\verb'GxB_SECONDJ_'$N$   & $ \rightarrow N$  & $z = j_y$   & column index of $y$ (0-based) \\
\verb'GxB_SECONDJ1_'$N$  & $ \rightarrow N$  & $z = j_y+1$ & column index of $y$ (1-based) \\
\hline
\end{tabular}
}

\vspace{0.2in}
The following methods operate on the \verb'GxB_IndexBinaryOp' object:

\vspace{0.1in}
\noindent
{\footnotesize
\begin{tabular}{lll}
\hline
GraphBLAS function   & purpose                                      & Section \\
\hline
\verb'GxB_IndexBinaryOp_new'   & create a named user-defined index-binary operator   & \ref{idxbinop_new_named} \\
\verb'GxB_IndexBinaryOp_wait'  & wait for a user-defined index-binary operator  & \ref{idxbinop_wait} \\
\verb'GxB_IndexBinaryOp_free'  & free a user-defined index-binary operator      & \ref{idxbinop_free} \\
\verb'GxB_BinaryOp_new_IndexOp' & create a new index-based \verb'GrB_BinaryOp' & \ref{binop_new_idxop} \\
\verb'GrB_get'           & get properties of an operator    & \ref{get_set_idxbinop} \\
\verb'GrB_set'           & set the operator name/definition & \ref{get_set_idxbinop} \\
\hline
\end{tabular}
}
\vspace{0.1in}

\newpage
%-------------------------------------------------------------------------------
\subsubsection{{\sf GxB\_IndexBinaryOp\_new:} create a user-defined index-binary operator}
%-------------------------------------------------------------------------------
\label{idxbinop_new_named}

\begin{mdframed}[userdefinedwidth=6in]
{\footnotesize
\begin{verbatim}
GrB_Info GxB_IndexBinaryOp_new
(
    GxB_IndexBinaryOp *op,          // handle for the new index binary operator
    GxB_index_binary_function function, // pointer to the index binary function
    GrB_Type ztype,                 // type of output z
    GrB_Type xtype,                 // type of input x
    GrB_Type ytype,                 // type of input y
    GrB_Type theta_type,            // type of input theta
    const char *idxbinop_name,      // name of the user function
    const char *idxbinop_defn       // definition of the user function
) ;
\end{verbatim} }\end{mdframed}

Creates a named \verb'GxB_IndexBinaryOp'.  Only the first 127 characters of
\verb'idxbinop_name' are used.  The \verb'ixdbinop_defn' is a string containing
the entire function itself.

The two strings \verb'idxbinop_name' and \verb'idxbinop_defn' are optional, but
are required to enable the JIT compilation of kernels that use this operator.
For example, the following operator can be used to compute the argmax of a
matrix with a single call to \verb'GrB_mxv'.  It returns a vector \verb'c'
where \verb'c(i) = (k,v)', where the largest value in the $i$th row of \verb'A'
has value \verb'v' and appears in column \verb'k'.  If multiple values in the
$i$th row have the same largest value, the one with the smallest column index
is returned.

{\footnotesize
\begin{verbatim}
    typedef struct { int64_t k ; double v ; } tuple_fp64 ;
    #define FP64_K "typedef struct { int64_t k ; double v ; } tuple_fp64 ;"
    void make_fp64 (tuple_fp64 *z,
        const double *x, GrB_Index ix, GrB_Index jx,
        const void   *y, GrB_Index iy, GrB_Index jy,
        const void *theta)
    {
        z->k = (int64_t) jx ;
        z->v = (*x) ;
    }
    void max_fp64 (tuple_fp64 *z, const tuple_fp64 *x, const tuple_fp64 *y)
    {
        if (x->v > y->v || (x->v == y->v && x->k < y->k))
        {
            z->k = x->k ;
            z->v = x->v ;
        }
        else
        {
            z->k = y->k ;
            z->v = y->v ;
        }
    }
    #define MAX_FP64 (a string containing the max_fp64 function above)

    // create the types and operators:
    GrB_Scalar Theta ;                  // unused, but cannot be NULL
    GrB_Scalar_new (&Theta, GrB_BOOL) ;
    GrB_Scalar_setElement_BOOL (Theta, 0) ;
    GxB_IndexBinaryOp Iop ;
    GrB_BinaryOp Bop, MonOp ;
    GrB_Type Tuple ;
    GxB_Type_new (&Tuple, sizeof (tuple_fp64), "tuple_fp64", FP64_K) ;
    GxB_IndexBinaryOp_new (&Iop, make_fp64, Tuple, GrB_FP64, GrB_BOOL, GrB_BOOL,
        "make_fp64", MAKE_FP64)) ;
    GxB_BinaryOp_new_IndexOp (&Bop, Iop, Theta) ;
    tuple_fp64 id ;
    memset (&id, 0, sizeof (tuple_fp64)) ;
    id.k = INT64_MAX ;
    id.v = (double) (-INFINITY) ;
    GxB_BinaryOp_new (&MonOp, max_fp64, Tuple, Tuple, Tuple, "max_fp64", MAX_FP64) ;
    GrB_Monoid MonOp ;
    GrB_Semiring Semiring ;
    GrB_Monoid_new_UDT (&Monoid, MonOp, &id) ;
    GrB_Semiring_new (&Semiring, Monoid, Bop) ;

    // compute the argmax of each row of a GrB_FP64 matrix A:
    // y = zeros (ncols,1) ;
    GrB_Vector y ;
    GrB_Matrix_new (&y, GrB_BOOL, ncols, 1)) ;
    GrB_Matrix_assign_BOOL (y, NULL, NULL, 0, GrB_ALL, ncols, GrB_ALL, 1, NULL)) ;
    // c = A*y using the argmax semiring
    GrB_Vector_new (&c, Tuple, nrows, 1)) ;
    GrB_mxv (c, NULL, NULL, Semiring, A, y, NULL) ; \end{verbatim}}

\newpage
%-------------------------------------------------------------------------------
\subsubsection{{\sf GxB\_IndexBinaryOp\_wait:} wait for an index-binary operator}
%-------------------------------------------------------------------------------
\label{idxbinop_wait}

\begin{mdframed}[userdefinedwidth=6in]
{\footnotesize
\begin{verbatim}
GrB_Info GxB_IndexBinaryOp_wait
(
    GxB_IndexBinaryOp op,
    int mode                    // GrB_COMPLETE or GrB_MATERIALIZE
) ;
\end{verbatim}
}\end{mdframed}

After creating a user-defined index-binary operator, a GraphBLAS library may choose
to exploit non-blocking mode to delay its creation.  Currently,
SuiteSparse:GraphBLAS currently does nothing except to ensure that the
\verb'op' is valid.

% \newpage
%-------------------------------------------------------------------------------
\subsubsection{{\sf GxB\_IndexBinaryOp\_free:} free a user-defined index-binary operator}
%-------------------------------------------------------------------------------
\label{idxbinop_free}

\begin{mdframed}[userdefinedwidth=6in]
{\footnotesize
\begin{verbatim}
GrB_Info GrB_free               // free a user-created index-binary operator
(
    GxB_IndexBinaryOp *op       // handle of IndexBinaryOp to free
) ;
\end{verbatim}
}\end{mdframed}

\verb'GxB_IndexBinaryOp_free' frees a user-defined index-binary operator.  Either usage:

    {\small
    \begin{verbatim}
    GxB_IndexBinaryOp_free (&op) ;
    GrB_free (&op) ; \end{verbatim}}

\noindent
frees the \verb'op' and sets \verb'op' to \verb'NULL'.  It safely
does nothing if passed a \verb'NULL' handle, or if \verb'op == NULL' on
input.  No built-in index-binary operators exist, but if they did,
the method does nothing at all if passed a built-in index-binary operator.

\newpage
%-------------------------------------------------------------------------------
\subsubsection{{\sf GxB\_BinaryOp\_new\_IndexOp:} create a index-based binary operator}
%-------------------------------------------------------------------------------
\label{binop_new_idxop}

\begin{mdframed}[userdefinedwidth=6in]
{\footnotesize
\begin{verbatim}
GrB_Info GxB_BinaryOp_new_IndexOp
(
    GrB_BinaryOp *binop,            // handle of binary op to create
    GxB_IndexBinaryOp idxbinop,     // based on this index binary op
    GrB_Scalar theta                // theta value to bind to the new binary op
) ;
\end{verbatim}
}\end{mdframed}

The \verb'GxB_IndexBinaryOp' cannot be directly used in any GraphBLAS operation
such as \verb'GrB_mxm'.  Instead, it must be used to create a new index-based
\verb'GrB_BinaryOp'.  The resulting binary operator can then be used to as the
multiplicative operator in a new user-defined semiring, or as the primary
binary operator of the element-wise operations (\verb'eWiseAdd',
\verb'eWiseUnion', \verb'eWiseMult', or \verb'kronecker').

The resulting binary operator cannot be used as the \verb'accum' operator in
any GraphBLAS operation.  It also cannot be used in other places where a binary
operator appears, including \verb'GrB_*_build', \verb'GrB_apply',
\verb'GrB_reduce' and \verb'GrB_*_sort'.

The \verb'GxB_BinaryOp_new_IndexOp' method creates this index-based binary
operator.  It takes two input parameters:  an index-binary operator, and a
scalar \verb'Theta'.  The value of \verb'Theta' is copied into this new binary
operator, and the value cannot be changed.  To change \verb'Theta', the binary
operator must be freed, and any semiring that would like to use the new value
of \verb'Theta' must also be recreated.

An example of its use is given in Section~\ref{idxbinop_new_named}.




\newpage
%===============================================================================
\subsection{GraphBLAS monoids: {\sf GrB\_Monoid}} %=============================
%===============================================================================
\label{monoid}

A {\em monoid} is defined on a single domain (that is, a single type), $T$.  It
consists of an associative binary operator $z=f(x,y)$ whose three operands $x$,
$y$, and $z$ are all in this same domain $T$ (that is $T \times T \rightarrow
T$).  The operator must also have an identity element, or ``zero'' in this
domain, such that $f(x,0)=f(0,x)=x$.  Recall that an associative operator
$f(x,y)$ is one for which the condition $f(a, f(b,c)) = f(f (a,b),c)$ always
holds.  That is, operator can be applied in any order and the results remain
the same.  If used in a semiring, the operator must also be commutative.

The 77 predefined monoids are listed in the table below, which
includes nearly all monoids that can be constructed from built-in binary
operators.  A few additional monoids can be defined with \verb'GrB_Monoid_new'
using built-in operators, such as bitwise monoids for signed integers.
Recall that $T$ denotes any built-in type (including boolean, integer, floating
point real, and complex), $R$ denotes any non-complex type (including bool),
$I$ denotes any integer type, and $Z$ denotes any complex type.  Let $S$ denote
the 10 non-boolean real types.  Let $U$ denote all unsigned integer types.

The table lists the GraphBLAS monoid, its type, expression, identity
value, and {\em terminal} value (if any).  For these built-in monoids, the
terminal values are the {\em annihilators} of the function, which is the value
$z$ so that $z=f(z,y)$ regardless of the value of $y$.  For example
$\min(-\infty,y) = -\infty$ for any $y$.  For integer domains, $+\infty$ and
$-\infty$ are the largest and smallest integer in their range.  With unsigned
integers, the smallest value is zero, and thus \verb'GrB_MIN_MONOID_UINT8' has an
identity of 255 and a terminal value of 0.

When computing with a monoid, the computation can terminate early if the
terminal value arises.  No further work is needed since the result will not
change.  This value is called the terminal value instead of the annihilator,
since a user-defined operator can be created with a terminal value that is not
an annihilator.  See Section~\ref{monoid_terminal_new} for an example.

The \verb'GxB_ANY_*' monoid can terminate as soon as it finds any value at all.

\vspace{0.2in}
\noindent
{\footnotesize
\begin{tabular}{lllll}
\hline
GraphBLAS             & types (domains)            & expression      & identity  & terminal \\
operator              &                            & $z=f(x,y)$      &           & \\
\hline
% numeric SxS -> S
\verb'GrB_PLUS_MONOID_'$S$   & $S \times S \rightarrow S$ & $z = x+y$       & 0         & none \\
\verb'GrB_TIMES_MONOID_'$S$  & $S \times S \rightarrow S$ & $z = xy$        & 1         & 0 or none (see note) \\
\verb'GrB_MIN_MONOID_'$S$    & $S \times S \rightarrow S$ & $z = \min(x,y)$ & $+\infty$ & $-\infty$ \\
\verb'GrB_MAX_MONOID_'$S$    & $S \times S \rightarrow S$ & $z = \max(x,y)$ & $-\infty$ & $+\infty$ \\
\hline
% complex ZxZ -> Z
\verb'GxB_PLUS_'$Z$\verb'_MONOID'   & $Z \times Z \rightarrow Z$ & $z = x+y$       & 0         & none \\
\verb'GxB_TIMES_'$Z$\verb'_MONOID'  & $Z \times Z \rightarrow Z$ & $z = xy$        & 1         & none \\
\hline
% any TxT -> T
\verb'GxB_ANY_'$T$\verb'_MONOID'   & $T \times T \rightarrow T$ & $z = x$ or $y$  & any       & any        \\
\hline
% bool x bool -> bool
\verb'GrB_LOR_MONOID'        & \verb'bool' $\times$ \verb'bool' $\rightarrow$ \verb'bool' & $z = x \vee    y $ & false & true  \\
\verb'GrB_LAND_MONOID'       & \verb'bool' $\times$ \verb'bool' $\rightarrow$ \verb'bool' & $z = x \wedge  y $ & true  & false \\
\verb'GrB_LXOR_MONOID'       & \verb'bool' $\times$ \verb'bool' $\rightarrow$ \verb'bool' & $z = x \veebar y $ & false & none \\
\verb'GrB_LXNOR_MONOID'      & \verb'bool' $\times$ \verb'bool' $\rightarrow$ \verb'bool' & $z =(x ==      y)$ & true  & none \\
\hline
% bitwise: UxU -> U
\verb'GxB_BOR_'$U$\verb'_MONOID'    & $U$ $\times$ $U$ $\rightarrow$ $U$ & \verb'z=x|y'    & all bits zero & all bits one  \\
\verb'GxB_BAND_'$U$\verb'_MONOID'   & $U$ $\times$ $U$ $\rightarrow$ $U$ & \verb'z=x&y'    & all bits one  & all bits zero \\
\verb'GxB_BXOR_'$U$\verb'_MONOID'   & $U$ $\times$ $U$ $\rightarrow$ $U$ & \verb'z=x^y'    & all bits zero & none \\
\verb'GxB_BXNOR_'$U$\verb'_MONOID'  & $U$ $\times$ $U$ $\rightarrow$ $U$ & \verb'z=~(x^y)' & all bits one  & none \\
\hline
\end{tabular}
}
\vspace{0.2in}

% 40: (min,max,+,*) x (int8,16,32,64, uint8,16,32,64, fp32, fp64)
The C API Specification includes 44 predefined monoids, with the naming
convention \verb'GrB_op_MONOID_type'.  Forty monoids are available for the four
operators \verb'MIN', \verb'MAX', \verb'PLUS', and \verb'TIMES', each with the
10 non-boolean real types.  Four boolean monoids are predefined:
\verb'GrB_LOR_MONOID_BOOL', \verb'GrB_LAND_MONOID_BOOL',
\verb'GrB_LXOR_MONOID_BOOL', and \verb'GrB_LXNOR_MONOID_BOOL'.

% 13 ANY
%  4 complex (PLUS, TIMES)
% 16 bitwise
% 33 total
These all appear in SuiteSparse:GraphBLAS, which adds 33 additional predefined
\verb'GxB*' monoids, with the naming convention \verb'GxB_op_type_MONOID'.  The
\verb'ANY' operator can be used for all 13 types (including complex).  The
\verb'PLUS' and \verb'TIMES' operators are provided for both complex types, for
4 additional complex monoids.  Sixteen monoids are predefined for four bitwise
operators (\verb'BOR', \verb'BAND', \verb'BXOR', and \verb'BNXOR'), each with
four unsigned integer types (\verb'UINT8', \verb'UINT16', \verb'UINT32', and
\verb'UINT64').

{\bf NOTE:}
The \verb'GrB_TIMES_FP*' operators do not have a terminal value of zero, since
they comply with the IEEE 754 standard, and \verb'0*NaN' is not zero, but
\verb'NaN'.  Technically, their terminal value is \verb'NaN', but this value is
rare in practice and thus the terminal condition is not worth checking.

The next sections define the following methods for the \verb'GrB_Monoid'
object:

\vspace{0.2in}
{\footnotesize
\begin{tabular}{lll}
\hline
GraphBLAS function   & purpose                                      & Section \\
\hline
\verb'GrB_Monoid_new'       & create a user-defined monoid                  & \ref{monoid_new} \\
\verb'GrB_Monoid_wait'      & wait for a user-defined monoid                & \ref{monoid_wait} \\
\verb'GxB_Monoid_terminal_new'  & create a monoid that has a terminal value & \ref{monoid_terminal_new} \\
\verb'GrB_Monoid_free'      & free a monoid                                 & \ref{monoid_free} \\
\verb'GrB_get'  & get properties of a monoid       & \ref{get_set_monoid} \\
\verb'GrB_set'  & set the monoid name              & \ref{get_set_monoid} \\
\hline
\end{tabular}
}
\vspace{0.2in}

%-------------------------------------------------------------------------------
\subsubsection{{\sf GrB\_Monoid\_new:} create a monoid}
%-------------------------------------------------------------------------------
\label{monoid_new}

\begin{mdframed}[userdefinedwidth=6in]
{\footnotesize
\begin{verbatim}
GrB_Info GrB_Monoid_new             // create a monoid
(
    GrB_Monoid *monoid,             // handle of monoid to create
    GrB_BinaryOp op,                // binary operator of the monoid
    <type> identity                 // identity value of the monoid
) ;
\end{verbatim}
} \end{mdframed}

\verb'GrB_Monoid_new' creates a monoid.  The operator, \verb'op', must be an
associative binary operator, either built-in or user-defined.

In the definition above, \verb'<type>' is a place-holder for the specific type
of the monoid.  For built-in types, it is the C type corresponding to the
built-in type (see Section~\ref{type}), such as \verb'bool', \verb'int32_t',
\verb'float', or \verb'double'.  In this case, \verb'identity' is a
scalar value of the particular type, not a pointer.  For
user-defined types, \verb'<type>' is \verb'void *', and thus \verb'identity' is
a not a scalar itself but a \verb'void *' pointer to a memory location
containing the identity value of the user-defined operator, \verb'op'.

If \verb'op' is a built-in operator with a known identity value, then the
\verb'identity' parameter is ignored, and its known identity value is used
instead.
%
The \verb'op' cannot be a binary operator
created by \verb'GxB_BinaryOp_new_IndexOp'.

%-------------------------------------------------------------------------------
\subsubsection{{\sf GrB\_Monoid\_wait:} wait for a monoid}
%-------------------------------------------------------------------------------
\label{monoid_wait}

\begin{mdframed}[userdefinedwidth=6in]
{\footnotesize
\begin{verbatim}
GrB_Info GrB_wait               // wait for a user-defined monoid
(
    GrB_Monoid monoid,          // monoid to wait for
    int mode                    // GrB_COMPLETE or GrB_MATERIALIZE
) ;
\end{verbatim}
}\end{mdframed}

After creating a user-defined monoid, a GraphBLAS library may choose to exploit
non-blocking mode to delay its creation.  Currently, SuiteSparse:GraphBLAS
currently does nothing except to ensure that the \verb'monoid' is valid.

\newpage
%-------------------------------------------------------------------------------
\subsubsection{{\sf GxB\_Monoid\_terminal\_new:} create a monoid with terminal}
%-------------------------------------------------------------------------------
\label{monoid_terminal_new}

\begin{mdframed}[userdefinedwidth=6in]
{\footnotesize
\begin{verbatim}
GrB_Info GxB_Monoid_terminal_new    // create a monoid that has a terminal value
(
    GrB_Monoid *monoid,             // handle of monoid to create
    GrB_BinaryOp op,                // binary operator of the monoid
    <type> identity,                // identity value of the monoid
    <type> terminal                 // terminal value of the monoid
) ;
\end{verbatim}
} \end{mdframed}

\verb'GxB_Monoid_terminal_new' is identical to \verb'GrB_Monoid_new', except
that it allows for the specification of a {\em terminal value}.  The
\verb'<type>' of the terminal value is the same as the \verb'identity'
parameter; see Section~\ref{monoid_new} for details.

The terminal value of a monoid is the value $z$ for which $z=f(z,y)$ for any
$y$, where $z=f(x,y)$ is the binary operator of the monoid.  This is also
called the {\em annihilator}, but the term {\em terminal value} is used here.
This is because all annihilators are terminal values, but a terminal value need
not be an annihilator, as described in the \verb'MIN' example below.

If the terminal value is encountered during computation, the rest of the
computations can be skipped.  This can greatly improve the performance of
\verb'GrB_reduce', and matrix multiply in specific cases (when a dot product
method is used).  For example, using \verb'GrB_reduce' to compute the sum of
all entries in a \verb'GrB_FP32' matrix with $e$ entries takes $O(e)$ time,
since a monoid based on \verb'GrB_PLUS_FP32' has no terminal value.  By
contrast, a reduction using \verb'GrB_LOR' on a \verb'GrB_BOOL' matrix can take
as little as $O(1)$ time, if a \verb'true' value is found in the matrix very
early.

Monoids based on the built-in \verb'GrB_MIN_*' and \verb'GrB_MAX_*' operators
(for any type), the boolean \verb'GrB_LOR', and the boolean \verb'GrB_LAND'
operators all have terminal values.  For example, the identity value of
\verb'GrB_LOR' is \verb'false', and its terminal value is \verb'true'.  When
computing a reduction of a set of boolean values to a single value, once a
\verb'true' is seen, the computation can exit early since the result is now
known.

If \verb'op' is a built-in operator with known identity and terminal values,
then the \verb'identity' and \verb'terminal' parameters are ignored, and its
known identity and terminal values are used instead.

There may be cases in which the user application needs to use a non-standard
terminal value for a built-in operator.  For example, suppose the matrix has
type \verb'GrB_FP32', but all values in the matrix are known to be
non-negative.  The annihilator value of \verb'MIN' is \verb'-INFINITY', but
this will never be seen.  However, the computation could terminate when
finding the value zero.  This is an example of using a terminal value that is
not actually an annihilator, but it functions like one since the monoid will
operate strictly on non-negative values.

In this case, a monoid created with \verb'GrB_MIN_FP32' will not terminate
early, because the identity and terminal inputs are ignored when using
\verb'GrB_Monoid_new' with a built-in operator as its input.
To create a monoid that can terminate early, create a user-defined operator
that computes the same thing as \verb'GrB_MIN_FP32', and then create a monoid
based on this user-defined operator with a terminal value of zero and an
identity of \verb'+INFINITY'.
%
The \verb'op' cannot be a binary operator
created by \verb'GxB_BinaryOp_new_IndexOp'.

% \newpage
%-------------------------------------------------------------------------------
\subsubsection{{\sf GrB\_Monoid\_free:} free a monoid}
%-------------------------------------------------------------------------------
\label{monoid_free}

\begin{mdframed}[userdefinedwidth=6in]
{\footnotesize
\begin{verbatim}
GrB_Info GrB_free                   // free a user-created monoid
(
    GrB_Monoid *monoid              // handle of monoid to free
) ;
\end{verbatim}
} \end{mdframed}

\verb'GrB_Monoid_frees' frees a monoid.  Either usage:

    {\small
    \begin{verbatim}
    GrB_Monoid_free (&monoid) ;
    GrB_free (&monoid) ; \end{verbatim}}

\noindent
frees the \verb'monoid' and sets \verb'monoid' to \verb'NULL'.  It safely does
nothing if passed a \verb'NULL' handle, or if \verb'monoid == NULL' on input.
It does nothing at all if passed a built-in monoid.




\newpage
%===============================================================================
\subsection{GraphBLAS semirings: {\sf GrB\_Semiring}} %=========================
%===============================================================================
\label{semiring}

A {\em semiring} defines all the operators required to define the
multiplication of two sparse matrices in GraphBLAS, ${\bf C=AB}$.  The ``add''
operator is a commutative and associative monoid, and the binary ``multiply''
operator defines a function $z=fmult(x,y)$ where the type of $z$ matches the
exactly with the monoid type.  SuiteSparse:GraphBLAS includes 1,473 predefined
built-in semirings.  The next sections define the following methods for the
\verb'GrB_Semiring' object:

\vspace{0.2in}
{\footnotesize
\begin{tabular}{lll}
\hline
GraphBLAS function   & purpose                                      & Section \\
\hline
\verb'GrB_Semiring_new'       & create a user-defined semiring           & \ref{semiring_new} \\
\verb'GrB_Semiring_wait'      & wait for a user-defined semiring         & \ref{semiring_wait} \\
\verb'GrB_Semiring_free'      & free a semiring                          & \ref{semiring_free} \\
\verb'GrB_get'  & get properties of a semiring       & \ref{get_set_semiring} \\
\verb'GrB_set'  & set the semiring name              & \ref{get_set_semiring} \\
\hline
\end{tabular}
}

% \newpage
%-------------------------------------------------------------------------------
\subsubsection{{\sf GrB\_Semiring\_new:} create a semiring}
%-------------------------------------------------------------------------------
\label{semiring_new}

\begin{mdframed}[userdefinedwidth=6in]
{\footnotesize
\begin{verbatim}
GrB_Info GrB_Semiring_new           // create a semiring
(
    GrB_Semiring *semiring,         // handle of semiring to create
    GrB_Monoid add,                 // add monoid of the semiring
    GrB_BinaryOp multiply           // multiply operator of the semiring
) ;
\end{verbatim}
} \end{mdframed}

\verb'GrB_Semiring_new' creates a new semiring, with \verb'add' being the
additive monoid and \verb'multiply' being the binary ``multiply'' operator.  In
addition to the standard error cases, the function returns
\verb'GrB_DOMAIN_MISMATCH' if the output (\verb'ztype') domain of
\verb'multiply' does not match the domain of the \verb'add' monoid.

The v2.0 C API Specification for GraphBLAS includes 124 predefined semirings,
with names of the form \verb'GrB_add_mult_SEMIRING_type', where \verb'add' is
the operator of the additive monoid, \verb'mult' is the multiply operator, and
\verb'type' is the type of the input $x$ to the multiply operator, $f(x,y)$.
The name of the domain for the additive monoid does not appear in the name,
since it always matches the type of the output of the \verb'mult' operator.
Twelve kinds of \verb'GrB*' semirings are available for all 10 real, non-boolean types:
    \verb'PLUS_TIMES', \verb'PLUS_MIN',
    \verb'MIN_PLUS', \verb'MIN_TIMES', \verb'MIN_FIRST', \verb'MIN_SECOND', \verb'MIN_MAX',
    \verb'MAX_PLUS', \verb'MAX_TIMES', \verb'MAX_FIRST', \verb'MAX_SECOND', and \verb'MAX_MIN'.
Four semirings are for boolean types only:
    \verb'LOR_LAND', \verb'LAND_LOR', \verb'LXOR_LAND', and \verb'LXNOR_LOR'.

SuiteSparse:GraphBLAS pre-defines 1,553 semirings from built-in types and
operators, listed below.  The naming convention is \verb'GxB_add_mult_type'.
The 124 \verb'GrB*' semirings are a subset of the list below, included with two
names: \verb'GrB*' and \verb'GxB*'.  If the \verb'GrB*' name is provided, its
use is preferred, for portability to other GraphBLAS implementations.

\vspace{-0.05in}
\begin{itemize}
\item 1000 semirings with a multiplier $T \times T \rightarrow T$ where $T$ is
    any of the 10 non-Boolean, real types, from the complete cross product of:

    \vspace{-0.05in}
    \begin{itemize}
    \item 5 monoids (\verb'MIN', \verb'MAX', \verb'PLUS', \verb'TIMES', \verb'ANY')
    \item 20 multiply operators
    (\verb'FIRST', \verb'SECOND', \verb'PAIR' (same as \verb'ONEB'),
    \verb'MIN', \verb'MAX',
    \verb'PLUS', \verb'MINUS', \verb'RMINUS', \verb'TIMES', \verb'DIV', \verb'RDIV',
    \verb'ISEQ', \verb'ISNE', \verb'ISGT',
    \verb'ISLT', \verb'ISGE', \verb'ISLE',
    \verb'LOR', \verb'LAND', \verb'LXOR').
    \item 10 non-Boolean types, $T$
    \end{itemize}

\item 300 semirings with a comparator $T \times T \rightarrow$
    \verb'bool', where $T$ is non-Boolean and real, from the complete cross product of:

    \vspace{-0.05in}
    \begin{itemize}
    \item 5 Boolean monoids
    (\verb'LAND', \verb'LOR', \verb'LXOR', \verb'EQ', \verb'ANY')
    \item 6 multiply operators
    (\verb'EQ', \verb'NE', \verb'GT', \verb'LT', \verb'GE', \verb'LE')
    \item 10 non-Boolean types, $T$
    \end{itemize}

\item 55 semirings with purely Boolean types, \verb'bool' $\times$ \verb'bool'
    $\rightarrow$ \verb'bool', from the complete cross product of:

    \vspace{-0.05in}
    \begin{itemize}
    \item 5 Boolean monoids
    (\verb'LAND', \verb'LOR', \verb'LXOR', \verb'EQ', \verb'ANY')
    \item 11 multiply operators
    (\verb'FIRST', \verb'SECOND', \verb'PAIR' (same as \verb'ONEB'),
    \verb'LOR', \verb'LAND', \verb'LXOR',
    \verb'EQ', \verb'GT', \verb'LT', \verb'GE', \verb'LE')
    \end{itemize}

\item 54 complex semirings, $Z \times Z \rightarrow Z$ where $Z$ is
    \verb'GxB_FC32' (single precision complex) or
    \verb'GxB_FC64' (double precision complex):

    \vspace{-0.05in}
    \begin{itemize}
    \item 3 complex monoids (\verb'PLUS', \verb'TIMES', \verb'ANY')
    \item 9 complex multiply operators
        (\verb'FIRST', \verb'SECOND', \verb'PAIR' (same as \verb'ONEB'),
        \verb'PLUS', \verb'MINUS',
            \verb'TIMES', \verb'DIV', \verb'RDIV', \verb'RMINUS')
    \item 2 complex types, $Z$
    \end{itemize}

\item 64 bitwise semirings, $U \times U \rightarrow U$ where $U$ is
    an unsigned integer.

    \vspace{-0.05in}
    \begin{itemize}
    \item 4 bitwise monoids (\verb'BOR', \verb'BAND', \verb'BXOR', \verb'BXNOR')
    \item 4 bitwise multiply operators (the same list)
    \item 4 unsigned integer types
    \end{itemize}

\item 80 index-based semirings, $X \times X \rightarrow N$ where $N$ is
    \verb'INT32' or \verb'INT64':

    \vspace{-0.05in}
    \begin{itemize}
    \item 5 monoids (\verb'MIN', \verb'MAX', \verb'PLUS', \verb'TIMES', \verb'ANY')
    \item 8 index-based operators
        (\verb'FIRSTI', \verb'FIRSTI1', \verb'FIRSTJ', \verb'FIRSTJ1',
        \verb'SECONDI', \verb'SECONDI1', \verb'SECONDJ', \verb'SECONDJ1')
    \item 2 integer types (\verb'INT32', \verb'INT64')
    \end{itemize}

\end{itemize}
%
The \verb'multiply' operator can be any a binary operator, including one
created by \verb'GxB_BinaryOp_new_IndexOp'.

%-------------------------------------------------------------------------------
\subsubsection{{\sf GrB\_Semiring\_wait:} wait for a semiring}
%-------------------------------------------------------------------------------
\label{semiring_wait}

\begin{mdframed}[userdefinedwidth=6in]
{\footnotesize
\begin{verbatim}
GrB_Info GrB_wait               // wait for a user-defined semiring
(
    GrB_Semiring semiring,      // semiring to wait for
    int mode                    // GrB_COMPLETE or GrB_MATERIALIZE
) ;
\end{verbatim}
}\end{mdframed}

After creating a user-defined semiring, a GraphBLAS library may choose to
exploit non-blocking mode to delay its creation.  Currently,
SuiteSparse:GraphBLAS currently does nothing except to ensure that the
\verb'semiring' is valid.

%-------------------------------------------------------------------------------
\subsubsection{{\sf GrB\_Semiring\_free:} free a semiring}
%-------------------------------------------------------------------------------
\label{semiring_free}

\begin{mdframed}[userdefinedwidth=6in]
{\footnotesize
\begin{verbatim}
GrB_Info GrB_free                   // free a user-created semiring
(
    GrB_Semiring *semiring          // handle of semiring to free
) ;
\end{verbatim}
} \end{mdframed}

\verb'GrB_Semiring_free' frees a semiring.  Either usage:

    {\small
    \begin{verbatim}
    GrB_Semiring_free (&semiring) ;
    GrB_free (&semiring) ; \end{verbatim}}

\noindent
frees the \verb'semiring' and sets \verb'semiring' to \verb'NULL'.  It safely
does nothing if passed a \verb'NULL' handle, or if \verb'semiring == NULL' on
input.  It does nothing at all if passed a built-in semiring.




\newpage
%===============================================================================
\subsection{GraphBLAS scalars: {\sf GrB\_Scalar}} %=============================
%===============================================================================
\label{scalar}

This section describes a set of methods that create, modify, query,
and destroy a GraphBLAS scalar, \verb'GrB_Scalar':

\vspace{0.2in}
{\footnotesize
\begin{tabular}{lll}
\hline
GraphBLAS function   & purpose                                      & Section \\
\hline
\verb'GrB_Scalar_new'            & create a scalar                      & \ref{scalar_new} \\
\verb'GrB_Scalar_wait'           & wait for a scalar                    & \ref{scalar_wait} \\
\verb'GrB_Scalar_dup'            & copy a scalar                        & \ref{scalar_dup} \\
\verb'GrB_Scalar_clear'          & clear a scalar of its entry          & \ref{scalar_clear} \\
\verb'GrB_Scalar_nvals'          & return number of entries in a scalar & \ref{scalar_nvals}  \\
\verb'GrB_Scalar_setElement'     & set the single entry of a scalar     & \ref{scalar_setElement} \\
\verb'GrB_Scalar_extractElement' & get the single entry from a scalar   & \ref{scalar_extractElement} \\
\verb'GxB_Scalar_memoryUsage'    & memory used by a scalar              & \ref{scalar_memusage} \\
\verb'GxB_Scalar_type'           & type of a scalar                     & \ref{scalar_type} \\
\verb'GrB_Scalar_free'           & free a scalar                        & \ref{scalar_free} \\
\hline
\hline
\verb'GrB_get'  & get properties of a scalar       & \ref{get_set_scalar} \\
\verb'GrB_set'  & set properties of a scalar       & \ref{get_set_scalar} \\
\hline
\end{tabular}
}

%-------------------------------------------------------------------------------
\subsubsection{{\sf GrB\_Scalar\_new:} create a scalar}
%-------------------------------------------------------------------------------
\label{scalar_new}

\begin{mdframed}[userdefinedwidth=6in]
{\footnotesize
\begin{verbatim}
GrB_Info GrB_Scalar_new     // create a new GrB_Scalar with no entry
(
    GrB_Scalar *s,          // handle of GrB_Scalar to create
    GrB_Type type           // type of GrB_Scalar to create
) ;
\end{verbatim}
} \end{mdframed}

\verb'GrB_Scalar_new' creates a new scalar with no
entry in it, of the given type.  This is analogous to MATLAB/Octave statement
\verb's = sparse(0)', except that GraphBLAS can create scalars any
type.  The pattern of the new scalar is empty.

%-------------------------------------------------------------------------------
\subsubsection{{\sf GrB\_Scalar\_wait:} wait for a scalar}
%-------------------------------------------------------------------------------
\label{scalar_wait}

\begin{mdframed}[userdefinedwidth=6in]
{\footnotesize
\begin{verbatim}
GrB_Info GrB_wait               // wait for a scalar
(
    GrB_Scalar s,               // scalar to wait for
    int mode                    // GrB_COMPLETE or GrB_MATERIALIZE
) ;
\end{verbatim}
}\end{mdframed}

In non-blocking mode, the computations for a \verb'GrB_Scalar' may be delayed.
In this case, the scalar is not yet safe to use by multiple independent user
threads.  A user application may force completion of a scalar \verb's' via
\verb'GrB_Scalar_wait(s,mode)'.
With a \verb'mode' of \verb'GrB_MATERIALIZE',
all pending computations are finished, and afterwards different user threads may
simultaneously call GraphBLAS operations that use the scalar \verb's' as an
input parameter.
See Section~\ref{omp_parallelism}
if GraphBLAS is compiled without OpenMP.

%-------------------------------------------------------------------------------
\subsubsection{{\sf GrB\_Scalar\_dup:} copy a scalar}
%-------------------------------------------------------------------------------
\label{scalar_dup}

\begin{mdframed}[userdefinedwidth=6in]
{\footnotesize
\begin{verbatim}
GrB_Info GrB_Scalar_dup     // make an exact copy of a GrB_Scalar
(
    GrB_Scalar *s,          // handle of output GrB_Scalar to create
    const GrB_Scalar t      // input GrB_Scalar to copy
) ;
\end{verbatim}
} \end{mdframed}

\verb'GrB_Scalar_dup' makes a deep copy of a scalar.
In GraphBLAS, it is possible, and valid, to write the following:

    {\footnotesize
    \begin{verbatim}
    GrB_Scalar t, s ;
    GrB_Scalar_new (&t, GrB_FP64) ;
    s = t ;                         // s is a shallow copy of t  \end{verbatim}}

Then \verb's' and \verb't' can be used interchangeably.  However, only a pointer
reference is made, and modifying one of them modifies both, and freeing one of
them leaves the other as a dangling handle that should not be used.
If two different scalars are needed, then this should be used instead:

    {\footnotesize
    \begin{verbatim}
    GrB_Scalar t, s ;
    GrB_Scalar_new (&t, GrB_FP64) ;
    GrB_Scalar_dup (&s, t) ;        // like s = t, but making a deep copy \end{verbatim}}

Then \verb's' and \verb't' are two different scalars that currently have
the same value, but they do not depend on each other.  Modifying one has no
effect on the other.
The \verb'GrB_NAME' is copied into the new scalar.

%-------------------------------------------------------------------------------
\subsubsection{{\sf GrB\_Scalar\_clear:} clear a scalar of its entry}
%-------------------------------------------------------------------------------
\label{scalar_clear}

\begin{mdframed}[userdefinedwidth=6in]
{\footnotesize
\begin{verbatim}
GrB_Info GrB_Scalar_clear   // clear a GrB_Scalar of its entry
(                           // type remains unchanged.
    GrB_Scalar s            // GrB_Scalar to clear
) ;
\end{verbatim}
} \end{mdframed}

\verb'GrB_Scalar_clear' clears the entry from a scalar.  The pattern of
\verb's' is empty, just as if it were created fresh with \verb'GrB_Scalar_new'.
Analogous with \verb's = sparse (0)' in MATLAB/Octave.  The type of \verb's' does not
change.  Any pending updates to the scalar are discarded.

%-------------------------------------------------------------------------------
\subsubsection{{\sf GrB\_Scalar\_nvals:} return the number of entries in a scalar}
%-------------------------------------------------------------------------------
\label{scalar_nvals}

\begin{mdframed}[userdefinedwidth=6in]
{\footnotesize
\begin{verbatim}
GrB_Info GrB_Scalar_nvals   // get the number of entries in a GrB_Scalar
(
    GrB_Index *nvals,       // GrB_Scalar has nvals entries (0 or 1)
    const GrB_Scalar s      // GrB_Scalar to query
) ;
\end{verbatim}
} \end{mdframed}

\verb'GrB_Scalar_nvals' returns the number of entries in a scalar, which
is either 0 or 1.  Roughly analogous to \verb'nvals = nnz(s)' in MATLAB/Octave,
except that the implicit value in GraphBLAS need not be zero and \verb'nnz'
(short for ``number of nonzeros'') in MATLAB is better described as ``number of
entries'' in GraphBLAS.

%-------------------------------------------------------------------------------
\subsubsection{{\sf GrB\_Scalar\_setElement:} set the single entry of a scalar}
%-------------------------------------------------------------------------------
\label{scalar_setElement}

\begin{mdframed}[userdefinedwidth=6in]
{\footnotesize
\begin{verbatim}
GrB_Info GrB_Scalar_setElement          // s = x
(
    GrB_Scalar s,                       // GrB_Scalar to modify
    <type> x                            // user scalar to assign to s
) ;
\end{verbatim} } \end{mdframed}

\verb'GrB_Scalar_setElement' sets the single entry in a scalar, like
\verb's = sparse(x)' in MATLAB notation.  For further details of this function,
see \verb'GrB_Matrix_setElement' in Section~\ref{matrix_setElement}.
If an error occurs, \verb'GrB_error(&err,s)' returns details about the error.
The scalar \verb'x' can be any non-opaque C scalar corresponding to
a built-in type, or \verb'void *' for a user-defined type.  It cannot be
a \verb'GrB_Scalar'.

\newpage
%-------------------------------------------------------------------------------
\subsubsection{{\sf GrB\_Scalar\_extractElement:} get the single entry from a scalar}
%-------------------------------------------------------------------------------
\label{scalar_extractElement}

\begin{mdframed}[userdefinedwidth=6in]
{\footnotesize
\begin{verbatim}
GrB_Info GrB_Scalar_extractElement  // x = s
(
    <type> *x,                      // user scalar extracted
    const GrB_Scalar s              // GrB_Sclar to extract an entry from
) ;
\end{verbatim} } \end{mdframed}

\verb'GrB_Scalar_extractElement' extracts the single entry from a sparse
scalar, like \verb'x = full(s)' in MATLAB.  Further details of this method are
discussed in Section~\ref{matrix_extractElement}, which discusses
\verb'GrB_Matrix_extractElement'.  {\bf NOTE: }  if no entry is present in the
scalar \verb's', then \verb'x' is not modified, and the return value of
\verb'GrB_Scalar_extractElement' is \verb'GrB_NO_VALUE'.

%-------------------------------------------------------------------------------
\subsubsection{{\sf GxB\_Scalar\_memoryUsage:} memory used by a scalar}
%-------------------------------------------------------------------------------
\label{scalar_memusage}

\begin{mdframed}[userdefinedwidth=6in]
{\footnotesize
\begin{verbatim}
GrB_Info GxB_Scalar_memoryUsage  // return # of bytes used for a scalar
(
    size_t *size,           // # of bytes used by the scalar s
    const GrB_Scalar s      // GrB_Scalar to query
) ;
\end{verbatim} } \end{mdframed}

Returns the memory space required for a scalar, in bytes.
By default, any read-only components are not included in the total memory.
This can be changed with via \verb'GrB_set'; see Section~\ref{get_set_global}.

%-------------------------------------------------------------------------------
\subsubsection{{\sf GxB\_Scalar\_type:} type of a scalar}
%-------------------------------------------------------------------------------
\label{scalar_type}

\begin{mdframed}[userdefinedwidth=6in]
{\footnotesize
\begin{verbatim}
GrB_Info GxB_Scalar_type    // get the type of a GrB_Scalar
(
    GrB_Type *type,         // returns the type of the GrB_Scalar
    const GrB_Scalar s      // GrB_Scalar to query
) ;
\end{verbatim} } \end{mdframed}

Returns the type of a scalar.  See \verb'GxB_Matrix_type' for details
(Section~\ref{matrix_type}).

\newpage
%-------------------------------------------------------------------------------
\subsubsection{{\sf GrB\_Scalar\_free:} free a scalar}
%-------------------------------------------------------------------------------
\label{scalar_free}

\begin{mdframed}[userdefinedwidth=6in]
{\footnotesize
\begin{verbatim}
GrB_Info GrB_free           // free a GrB_Scalar
(
    GrB_Scalar *s           // handle of GrB_Scalar to free
) ;
\end{verbatim}
} \end{mdframed}

\verb'GrB_Scalar_free' frees a scalar.  Either usage:

    {\small
    \begin{verbatim}
    GrB_Scalar_free (&s) ;
    GrB_free (&s) ; \end{verbatim}}

\noindent
frees the scalar \verb's' and sets \verb's' to \verb'NULL'.  It safely
does nothing if passed a \verb'NULL' handle, or if \verb's == NULL' on input.
Any pending updates to the scalar are abandoned.




\newpage
%===============================================================================
\subsection{GraphBLAS vectors: {\sf GrB\_Vector}} %=============================
%===============================================================================
\label{vector}

This section describes a set of methods that create, modify, query,
and destroy a GraphBLAS sparse vector, \verb'GrB_Vector':

\vspace{0.2in}
\noindent
{\footnotesize
\begin{tabular}{lll}
\hline
GraphBLAS function   & purpose                                      & Section \\
\hline
\verb'GrB_Vector_new'            & create a vector                  & \ref{vector_new} \\
\verb'GrB_Vector_wait'           & wait for a vector                & \ref{vector_wait} \\
\verb'GrB_Vector_dup'            & copy a vector                    & \ref{vector_dup} \\
\verb'GrB_Vector_clear'          & clear a vector of all entries    & \ref{vector_clear} \\
\verb'GrB_Vector_size'           & size of a vector                 & \ref{vector_size} \\
\verb'GrB_Vector_nvals'          & number of entries in a vector    & \ref{vector_nvals} \\
\verb'GrB_Vector_build'          & build a vector from tuples       & \ref{vector_build} \\
\verb'GxB_Vector_build_Vector'   & build a vector from tuples       & \ref{vector_build_Vector} \\
\verb'GxB_Vector_build_Scalar'   & build a vector from tuples       & \ref{vector_build_Scalar} \\
\verb'GxB_Vector_build_Scalar_Vector' & build a vector from tuples  & \ref{vector_build_Scalar_Vector} \\
\verb'GrB_Vector_setElement'     & add an entry to a vector         & \ref{vector_setElement} \\
\verb'GrB_Vector_extractElement' & get an entry from a vector       & \ref{vector_extractElement} \\
\verb'GxB_Vector_isStoredElement'& check if entry present in vector & \ref{vector_isStoredElement} \\
\verb'GrB_Vector_removeElement'  & remove an entry from a vector    & \ref{vector_removeElement} \\
\verb'GrB_Vector_extractTuples'  & get all entries from a vector    & \ref{vector_extractTuples} \\
\verb'GxB_Vector_extractTuples_Vector'  & get all entries from a vector    & \ref{vector_extractTuples_Vector} \\
\verb'GrB_Vector_resize'         & resize a vector                  & \ref{vector_resize} \\
\verb'GxB_Vector_diag'           & extract a diagonal from a matrix & \ref{vector_diag} \\
\verb'GxB_Vector_memoryUsage'    & memory used by a vector          & \ref{vector_memusage} \\
\verb'GxB_Vector_type'           & type of the matrix               & \ref{vector_type} \\
\verb'GrB_Vector_free'           & free a vector                    & \ref{vector_free} \\
\hline
\hline
% NOTE: GrB_Vector_serialize / deserialize does not appear in the 2.0 C API.
% \verb'GrB_Vector_serializeSize'  & return size of serialized vector & \ref{vector_serialize_size} \\
% \verb'GrB_Vector_serialize'      & serialize a vector               & \ref{vector_serialize} \\
\verb'GxB_Vector_serialize'      & serialize a vector               & \ref{vector_serialize_GxB} \\
% \verb'GrB_Vector_deserialize'    & deserialize a vector             & \ref{vector_deserialize} \\
\verb'GxB_Vector_deserialize'    & deserialize a vector             & \ref{vector_deserialize_GxB} \\
\hline
\hline
\verb'GxB_Vector_sort'          & sort a vector & \ref{vector_sort} \\
\hline
\hline
\verb'GrB_get'  & get properties of a vector       & \ref{get_set_vector} \\
\verb'GrB_set'  & set properties of a vector       & \ref{get_set_vector} \\
\end{tabular}
}

\vspace{0.2in}
Refer to
Section~\ref{serialize_deserialize} for serialization/deserialization methods
and to
Section~\ref{sorting_methods} for sorting methods.

\newpage
%-------------------------------------------------------------------------------
\subsubsection{{\sf GrB\_Vector\_new:}           create a vector}
%-------------------------------------------------------------------------------
\label{vector_new}

\begin{mdframed}[userdefinedwidth=6in]
{\footnotesize
\begin{verbatim}
GrB_Info GrB_Vector_new     // create a new vector with no entries
(
    GrB_Vector *v,          // handle of vector to create
    GrB_Type type,          // type of vector to create
    GrB_Index n             // vector dimension is n-by-1
) ;
\end{verbatim}
} \end{mdframed}

\verb'GrB_Vector_new' creates a new \verb'n'-by-\verb'1' sparse vector with no
entries in it, of the given type.  This is analogous to MATLAB/Octave statement
\verb'v = sparse (n,1)', except that GraphBLAS can create sparse vectors any
type.  The pattern of the new vector is empty.

\begin{alert}
{\bf SPEC:} \verb'n' may be zero, as an extension to the specification.
\end{alert}

%-------------------------------------------------------------------------------
\subsubsection{{\sf GrB\_Vector\_wait:} wait for a vector}
%-------------------------------------------------------------------------------
\label{vector_wait}

\begin{mdframed}[userdefinedwidth=6in]
{\footnotesize
\begin{verbatim}
GrB_Info GrB_wait               // wait for a vector
(
    GrB_Vector w,               // vector to wait for
    int mode                    // GrB_COMPLETE or GrB_MATERIALIZE
) ;
\end{verbatim}
}\end{mdframed}

In non-blocking mode, the computations for a \verb'GrB_Vector' may be delayed.
In this case, the vector is not yet safe to use by multiple independent user
threads.  A user application may force completion of a vector \verb'w' via
\verb'GrB_Vector_wait(w,mode)'.
With a \verb'mode' of \verb'GrB_MATERIALIZE',
all pending computations are finished, and afterwards different user threads may
simultaneously call GraphBLAS operations that use the vector \verb'w' as an
input parameter.
See Section~\ref{omp_parallelism}
if GraphBLAS is compiled without OpenMP.

\newpage
%-------------------------------------------------------------------------------
\subsubsection{{\sf GrB\_Vector\_dup:}           copy a vector}
%-------------------------------------------------------------------------------
\label{vector_dup}

\begin{mdframed}[userdefinedwidth=6in]
{\footnotesize
\begin{verbatim}
GrB_Info GrB_Vector_dup     // make an exact copy of a vector
(
    GrB_Vector *w,          // handle of output vector to create
    const GrB_Vector u      // input vector to copy
) ;
\end{verbatim}
} \end{mdframed}

\verb'GrB_Vector_dup' makes a deep copy of a sparse vector.
In GraphBLAS, it is possible, and valid, to write the following:

    {\footnotesize
    \begin{verbatim}
    GrB_Vector u, w ;
    GrB_Vector_new (&u, GrB_FP64, n) ;
    w = u ;                         // w is a shallow copy of u  \end{verbatim}}

Then \verb'w' and \verb'u' can be used interchangeably.  However, only a pointer
reference is made, and modifying one of them modifies both, and freeing one of
them leaves the other as a dangling handle that should not be used.
If two different vectors are needed, then this should be used instead:

    {\footnotesize
    \begin{verbatim}
    GrB_Vector u, w ;
    GrB_Vector_new (&u, GrB_FP64, n) ;
    GrB_Vector_dup (&w, u) ;        // like w = u, but making a deep copy \end{verbatim}}

Then \verb'w' and \verb'u' are two different vectors that currently have the
same set of values, but they do not depend on each other.  Modifying one has
no effect on the other.
The \verb'GrB_NAME' is copied into the new vector.

%-------------------------------------------------------------------------------
\subsubsection{{\sf GrB\_Vector\_clear:}         clear a vector of all entries}
%-------------------------------------------------------------------------------
\label{vector_clear}

\begin{mdframed}[userdefinedwidth=6in]
{\footnotesize
\begin{verbatim}
GrB_Info GrB_Vector_clear   // clear a vector of all entries;
(                           // type and dimension remain unchanged.
    GrB_Vector v            // vector to clear
) ;
\end{verbatim}
} \end{mdframed}

\verb'GrB_Vector_clear' clears all entries from a vector.  All values
\verb'v(i)' are now equal to the implicit value, depending on what semiring
ring is used to perform computations on the vector.  The pattern of \verb'v' is
empty, just as if it were created fresh with \verb'GrB_Vector_new'.  Analogous
with \verb'v (:) = sparse(0)' in MATLAB.  The type and dimension of \verb'v' do
not change.  Any pending updates to the vector are discarded.

\newpage
%-------------------------------------------------------------------------------
\subsubsection{{\sf GrB\_Vector\_size:}          return the size of a vector}
%-------------------------------------------------------------------------------
\label{vector_size}

\begin{mdframed}[userdefinedwidth=6in]
{\footnotesize
\begin{verbatim}
GrB_Info GrB_Vector_size    // get the dimension of a vector
(
    GrB_Index *n,           // vector dimension is n-by-1
    const GrB_Vector v      // vector to query
) ;
\end{verbatim}
} \end{mdframed}

\verb'GrB_Vector_size' returns the size of a vector (the number of rows).
Analogous to \verb'n = length(v)' or \verb'n = size(v,1)' in MATLAB.

%-------------------------------------------------------------------------------
\subsubsection{{\sf GrB\_Vector\_nvals:}         return the number of entries in a vector}
%-------------------------------------------------------------------------------
\label{vector_nvals}

\begin{mdframed}[userdefinedwidth=6in]
{\footnotesize
\begin{verbatim}
GrB_Info GrB_Vector_nvals   // get the number of entries in a vector
(
    GrB_Index *nvals,       // vector has nvals entries
    const GrB_Vector v      // vector to query
) ;
\end{verbatim}
} \end{mdframed}

\verb'GrB_Vector_nvals' returns the number of entries in a vector.  Roughly
analogous to \verb'nvals = nnz(v)' in MATLAB, except that the implicit value in
GraphBLAS need not be zero and \verb'nnz' (short for ``number of nonzeros'') in
MATLAB is better described as ``number of entries'' in GraphBLAS.

%-------------------------------------------------------------------------------
\subsubsection{{\sf GrB\_Vector\_build:}         build a vector from a set of tuples}
%-------------------------------------------------------------------------------
\label{vector_build}

\begin{mdframed}[userdefinedwidth=6in]
{\footnotesize
\begin{verbatim}
GrB_Info GrB_Vector_build           // build a vector from (I,X) tuples
(
    GrB_Vector w,                   // vector to build
    const GrB_Index *I,             // array of row indices of tuples
    const <type> *X,                // array of values of tuples
    GrB_Index nvals,                // number of tuples
    const GrB_BinaryOp dup          // binary function to assemble duplicates
) ;
\end{verbatim}
} \end{mdframed}

\verb'GrB_Vector_build' constructs a sparse vector \verb'w' from a set of
tuples, \verb'I' and \verb'X', each of length \verb'nvals'.  The vector
\verb'w' must have already been initialized with \verb'GrB_Vector_new', and it
must have no entries in it before calling \verb'GrB_Vector_build'.
This function is just like \verb'GrB_Matrix_build' (see
Section~\ref{matrix_build}), except that it builds a sparse vector instead of a
sparse matrix.  For a description of what \verb'GrB_Vector_build' does, refer
to \verb'GrB_Matrix_build'.  For a vector, the list of column indices \verb'J'
in \verb'GrB_Matrix_build' is implicitly a vector of length \verb'nvals' all
equal to zero.  Otherwise the methods are identical.

If \verb'dup' is \verb'NULL', any duplicates result in an error.
If \verb'dup' is the special binary operator \verb'GxB_IGNORE_DUP', then
any duplicates are ignored.  If duplicates appear, the last one in the
list of tuples is taken and the prior ones ignored.  This is not an error.
%
The \verb'dup' operator cannot be a binary operator
created by \verb'GxB_BinaryOp_new_IndexOp'.

\begin{alert}
{\bf SPEC:} Results are defined even if \verb'dup' is non-associative and/or
non-commutative.
\end{alert}

%-------------------------------------------------------------------------------
\subsubsection{{\sf GrB\_Vector\_build\_Vector:} build a vector from a set of tuples}
%-------------------------------------------------------------------------------
\label{vector_build_Vector}

\begin{mdframed}[userdefinedwidth=6in]
{\footnotesize
\begin{verbatim}
GrB_Info GrB_Vector_build       // build a vector from (I,X) tuples
(
    GrB_Vector w,               // vector to build
    const GrB_Vector I_vector,  // row indices
    const GrB_Vector X_vector,  // values
    const GrB_BinaryOp dup,     // binary function to assemble duplicates
    const GrB_Descriptor desc
) ;
\end{verbatim}
} \end{mdframed}

\verb'GxB_Vector_build_Vector' is identical to \verb'GrB_Vector_build', except
that the inputs \verb'I' and \verb'X' are \verb'GrB_Vector' objects, each with
\verb'nvals' entries.  The interpretation of \verb'I_vector' and
\verb'X_vector' are controlled by descriptor settings \verb'GxB_ROWINDEX_LIST'
and \verb'GxB_VALUE_LIST', respectively.  The method can use either the indices
or values of each of the input vectors; the default is to use the values.  See
Section~\ref{ijxvector} for details.

%-------------------------------------------------------------------------------
\subsubsection{{\sf GxB\_Vector\_build\_Scalar:} build a vector from a set of tuples}
%-------------------------------------------------------------------------------
\label{vector_build_Scalar}

\begin{mdframed}[userdefinedwidth=6in]
{\footnotesize
\begin{verbatim}
GrB_Info GrB_Vector_build       // build a vector from (I,scalar) tuples
(
    GrB_Vector w,                   // vector to build
    const GrB_Index *I,             // array of row indices of tuples
    GrB_Scalar scalar,              // value for all tuples
    GrB_Index nvals                 // number of tuples
) ;
\end{verbatim} } \end{mdframed}

\verb'GxB_Vector_build_Scalar' constructs a sparse vector \verb'w' from a set
of tuples defined by the index array \verb'I' of length \verb'nvals', and a
scalar.  The scalar is the value of all of the tuples.  Unlike
\verb'GrB_Vector_build', there is no \verb'dup' operator to handle duplicate
entries.  Instead, any duplicates are silently ignored (if the number of
duplicates is desired, simply compare the input \verb'nvals' with the value
returned by \verb'GrB_Vector_nvals' after the vector is constructed).  All
entries in the sparsity pattern of \verb'w' are identical, and equal to the
input scalar value.

%-------------------------------------------------------------------------------
\subsubsection{{\sf GxB\_Vector\_build\_Scalar\_Vector:} build a vector from a set of tuples}
%-------------------------------------------------------------------------------
\label{vector_build_Scalar_Vector}

\begin{mdframed}[userdefinedwidth=6in]
{\footnotesize
\begin{verbatim}
GrB_Info GrB_Vector_build       // build a vector from (I,scalar) tuples
(
    GrB_Vector w,               // vector to build
    const GrB_Vector I_vector,  // row indices
    const GrB_Scalar scalar,    // value for all tuples
    const GrB_Descriptor desc
) ;
\end{verbatim} } \end{mdframed}

\verb'GxB_Vector_build_Scalar_Vector' is identical to
\verb'GxB_Vector_build_Scalar', except that the inputs \verb'I' and \verb'X'
are \verb'GrB_Vector' objects, each with \verb'nvals' entries.  The
interpretation of \verb'I_vector' is controlled by the descriptor setting
\verb'GxB_ROWINDEX_LIST'.  The method
can use either the indices or values of the \verb'I_input' vector; the default
is to use the values.  See Section~\ref{ijxvector} for details.

%-------------------------------------------------------------------------------
\subsubsection{{\sf GrB\_Vector\_setElement:}    add an entry to a vector}
%-------------------------------------------------------------------------------
\label{vector_setElement}

\begin{mdframed}[userdefinedwidth=6in]
{\footnotesize
\begin{verbatim}
GrB_Info GrB_Vector_setElement          // w(i) = x
(
    GrB_Vector w,                       // vector to modify
    <type> x,                           // scalar to assign to w(i)
    GrB_Index i                         // index
) ;
\end{verbatim} } \end{mdframed}

\verb'GrB_Vector_setElement' sets a single entry in a vector, \verb'w(i) = x'.
The operation is exactly like setting a single entry in an \verb'n'-by-1
matrix, \verb'A(i,0) = x', where the column index for a vector is implicitly
\verb'j=0'.  For further details of this function, see
\verb'GrB_Matrix_setElement' in Section~\ref{matrix_setElement}.
If an error occurs, \verb'GrB_error(&err,w)' returns details about the error.

\newpage
%-------------------------------------------------------------------------------
\subsubsection{{\sf GrB\_Vector\_extractElement:} get an entry from a vector}
%-------------------------------------------------------------------------------
\label{vector_extractElement}

\begin{mdframed}[userdefinedwidth=6in]
{\footnotesize
\begin{verbatim}
GrB_Info GrB_Vector_extractElement  // x = v(i)
(
    <type> *x,                  // scalar extracted (non-opaque, C scalar)
    const GrB_Vector v,         // vector to extract an entry from
    GrB_Index i                 // index
) ;

GrB_Info GrB_Vector_extractElement  // x = v(i)
(
    GrB_Scalar x,               // GrB_Scalar extracted
    const GrB_Vector v,         // vector to extract an entry from
    GrB_Index i                 // index
) ;
\end{verbatim} } \end{mdframed}

\verb'GrB_Vector_extractElement' extracts a single entry from a vector,
\verb'x = v(i)'.  The method is identical to extracting a single entry
\verb'x = A(i,0)' from an \verb'n'-by-1 matrix; see
Section~\ref{matrix_extractElement}.

%-------------------------------------------------------------------------------
\subsubsection{{\sf GxB\_Vector\_isStoredElement:} check if entry present in vector}
%-------------------------------------------------------------------------------
\label{vector_isStoredElement}

\begin{mdframed}[userdefinedwidth=6in]
{\footnotesize
\begin{verbatim}
GrB_Info GxB_Vector_isStoredElement
(
    const GrB_Vector v,         // check presence of entry v(i)
    GrB_Index i                 // index
) ;
\end{verbatim} } \end{mdframed}

\verb'GxB_Vector_isStoredElement' checks if a single entry \verb'v(i)'
is present, returning \verb'GrB_SUCCESS' if the entry is present or
\verb'GrB_NO_VALUE' otherwise.  The value of \verb'v(i)' is not returned.
See also Section~\ref{matrix_isStoredElement}.

%-------------------------------------------------------------------------------
\subsubsection{{\sf GrB\_Vector\_removeElement:} remove an entry from a vector}
%-------------------------------------------------------------------------------
\label{vector_removeElement}

\begin{mdframed}[userdefinedwidth=6in]
{\footnotesize
\begin{verbatim}
GrB_Info GrB_Vector_removeElement
(
    GrB_Vector w,                   // vector to remove an entry from
    GrB_Index i                     // index
) ;
\end{verbatim} } \end{mdframed}

\verb'GrB_Vector_removeElement' removes a single entry \verb'w(i)' from a vector.
If no entry is present at \verb'w(i)', then the vector is not modified.
If an error occurs, \verb'GrB_error(&err,w)' returns details about the error.

%-------------------------------------------------------------------------------
\subsubsection{{\sf GrB\_Vector\_extractTuples:} get all entries from a vector}
%-------------------------------------------------------------------------------
\label{vector_extractTuples}

\begin{mdframed}[userdefinedwidth=6in]
{\footnotesize
\begin{verbatim}
GrB_Info GrB_Vector_extractTuples           // [I,~,X] = find (v)
(
    GrB_Index *I,               // array for returning row indices of tuples
    <type> *X,                  // array for returning values of tuples
    GrB_Index *nvals,           // I, X size on input; # tuples on output
    const GrB_Vector v          // vector to extract tuples from
) ;
\end{verbatim} } \end{mdframed}

\verb'GrB_Vector_extractTuples' extracts all tuples from a sparse vector,
analogous to \verb'[I,~,X] = find(v)' in MATLAB/Octave.  This function is
identical to its \verb'GrB_Matrix_extractTuples' counterpart, except that the
array of column indices \verb'J' does not appear in this function.  Refer to
Section~\ref{matrix_extractTuples} where further details of this function are
described.

%-------------------------------------------------------------------------------
\subsubsection{{\sf GxB\_Vector\_extractTuples\_Vector:} get all entries from a vector}
%-------------------------------------------------------------------------------
\label{vector_extractTuples_Vector}

\begin{mdframed}[userdefinedwidth=6in]
{\footnotesize
\begin{verbatim}
GrB_Info GrB_Vector_extractTuples           // [I,~,X] = find (v)
(
    GrB_Vector I_vector,    // row indices
    GrB_Vector X_vector,    // values
    const GrB_Vector V,     // vectors to extract tuples from
    const GrB_Descriptor desc   // currently unused; for future expansion
) ;

\end{verbatim} } \end{mdframed}

\verb'GxB_Vector_extractTuples_Vector' is identical to
\verb'GrB_Vector_extractTuples' except that its two outputs are
\verb'GrB_Vector' objects.  The vectors \verb'I_vector' and \verb'X_vector'
objects must exist on input.  On output, any prior content is erased and
their type, dimensions, and values are revised to contain dense vectors of
length \verb'nvals'.

\newpage
%-------------------------------------------------------------------------------
\subsubsection{{\sf GrB\_Vector\_resize:}          resize a vector}
%-------------------------------------------------------------------------------
\label{vector_resize}

\begin{mdframed}[userdefinedwidth=6in]
{\footnotesize
\begin{verbatim}
GrB_Info GrB_Vector_resize      // change the size of a vector
(
    GrB_Vector u,               // vector to modify
    GrB_Index nrows_new         // new number of rows in vector
) ;
\end{verbatim} } \end{mdframed}

\verb'GrB_Vector_resize' changes the size of a vector.  If the dimension
decreases, entries that fall outside the resized vector are deleted.

%-------------------------------------------------------------------------------
\subsubsection{{\sf GxB\_Vector\_diag:} extract a diagonal from a matrix}
%-------------------------------------------------------------------------------
\label{vector_diag}

\begin{mdframed}[userdefinedwidth=6in]
{\footnotesize
\begin{verbatim}
GrB_Info GxB_Vector_diag    // extract a diagonal from a matrix
(
    GrB_Vector v,                   // output vector
    const GrB_Matrix A,             // input matrix
    int64_t k,
    const GrB_Descriptor desc       // unused, except threading control
) ;
\end{verbatim} } \end{mdframed}


\verb'GxB_Vector_diag' extracts a vector \verb'v' from an input matrix
\verb'A', which may be rectangular.  If \verb'k' = 0, the main diagonal of
\verb'A' is extracted; \verb'k' $> 0$ denotes diagonals above the main diagonal
of \verb'A', and \verb'k' $< 0$ denotes diagonals below the main diagonal of
\verb'A'.  Let \verb'A' have dimension $m$-by-$n$.  If \verb'k' is in the range
0 to $n-1$, then \verb'v' has length $\min(m,n-k)$.  If \verb'k' is negative
and in the range -1 to $-m+1$, then \verb'v' has length $\min(m+k,n)$.  If
\verb'k' is outside these ranges, \verb'v' has length 0 (this is not an error).
This function computes the same thing as the MATLAB/Octave statement
\verb'v=diag(A,k)' when \verb'A' is a matrix, except that
\verb'GxB_Vector_diag' can also do typecasting.

The vector \verb'v' must already exist on input, and
\verb'GrB_Vector_size (&len,v)' must return \verb'len' = 0 if \verb'k' $\ge n$
or \verb'k' $\le -m$, \verb'len' $=\min(m,n-k)$ if \verb'k' is in the range 0
to $n-1$, and \verb'len' $=\min(m+k,n)$ if \verb'k' is in the range -1 to
$-m+1$.  Any existing entries in \verb'v' are discarded.  The type of \verb'v'
is preserved, so that if the type of \verb'A' and \verb'v' differ, the entries
are typecasted into the type of \verb'v'.  Any settings made to \verb'v' by
\verb'GrB_set' (bitmap switch and sparsity control) are
unchanged.

\newpage

%-------------------------------------------------------------------------------
\subsubsection{{\sf GxB\_Vector\_memoryUsage:} memory used by a vector}
%-------------------------------------------------------------------------------
\label{vector_memusage}

\begin{mdframed}[userdefinedwidth=6in]
{\footnotesize
\begin{verbatim}
GrB_Info GxB_Vector_memoryUsage  // return # of bytes used for a vector
(
    size_t *size,           // # of bytes used by the vector v
    const GrB_Vector v      // vector to query
) ;
\end{verbatim} } \end{mdframed}

Returns the memory space required for a vector, in bytes.
By default, any read-only components are not included in the total memory.
This can be changed with via \verb'GrB_set'; see Section~\ref{get_set_global}.

%-------------------------------------------------------------------------------
\subsubsection{{\sf GxB\_Vector\_type:} type of a vector}
%-------------------------------------------------------------------------------
\label{vector_type}

\begin{mdframed}[userdefinedwidth=6in]
{\footnotesize
\begin{verbatim}
GrB_Info GxB_Vector_type    // get the type of a vector
(
    GrB_Type *type,         // returns the type of the vector
    const GrB_Vector v      // vector to query
) ;
\end{verbatim} } \end{mdframed}

Returns the type of a vector.  See \verb'GxB_Matrix_type' for details
(Section~\ref{matrix_type}).

%-------------------------------------------------------------------------------
\subsubsection{{\sf GrB\_Vector\_free:}          free a vector}
%-------------------------------------------------------------------------------
\label{vector_free}

\begin{mdframed}[userdefinedwidth=6in]
{\footnotesize
\begin{verbatim}
GrB_Info GrB_free           // free a vector
(
    GrB_Vector *v           // handle of vector to free
) ;
\end{verbatim}
} \end{mdframed}

\verb'GrB_Vector_free' frees a vector.  Either usage:

    {\small
    \begin{verbatim}
    GrB_Vector_free (&v) ;
    GrB_free (&v) ; \end{verbatim}}

\noindent
frees the vector \verb'v' and sets \verb'v' to \verb'NULL'.  It safely does
nothing if passed a \verb'NULL' handle, or if \verb'v == NULL' on input.  Any
pending updates to the vector are abandoned.





\newpage
%===============================================================================
\subsection{GraphBLAS matrices: {\sf GrB\_Matrix}} %============================
%===============================================================================
\label{matrix}

This section describes a set of methods that create, modify, query,
and destroy a GraphBLAS sparse matrix, \verb'GrB_Matrix':

\vspace{0.2in}
\noindent
{\footnotesize
\begin{tabular}{lll}
\hline
GraphBLAS function   & purpose                                      & Section \\
\hline
\verb'GrB_Matrix_new'           & create a matrix                       & \ref{matrix_new} \\
\verb'GrB_Matrix_wait'          & wait for a matrix                     & \ref{matrix_wait} \\
\verb'GrB_Matrix_dup'           & copy a matrix                         & \ref{matrix_dup} \\
\verb'GrB_Matrix_clear'         & clear a matrix of all entries         & \ref{matrix_clear} \\
\verb'GrB_Matrix_nrows'         & number of rows of a matrix            & \ref{matrix_nrows} \\
\verb'GrB_Matrix_ncols'         & number of columns of a matrix         & \ref{matrix_ncols} \\
\verb'GrB_Matrix_nvals'         & number of entries in a matrix         & \ref{matrix_nvals} \\
\verb'GrB_Matrix_build'         & build a matrix from tuples            & \ref{matrix_build} \\
\verb'GxB_Matrix_build_Vector'         & build a matrix from tuples     & \ref{matrix_build_Vector} \\
\verb'GxB_Matrix_build_Scalar'  & build a matrix from tuples            & \ref{matrix_build_Scalar} \\
\verb'GxB_Matrix_build_Scalar_Vector'  & build a matrix from tuples     & \ref{matrix_build_Scalar_Vector} \\
\verb'GrB_Matrix_setElement'    & add an entry to a matrix              & \ref{matrix_setElement} \\
\verb'GrB_Matrix_extractElement'& get an entry from a matrix            & \ref{matrix_extractElement} \\
\verb'GxB_Matrix_isStoredElement'& check if entry present in matrix     & \ref{matrix_isStoredElement} \\
\verb'GrB_Matrix_removeElement' & remove an entry from a matrix         & \ref{matrix_removeElement} \\
\verb'GrB_Matrix_extractTuples' & get all entries from a matrix         & \ref{matrix_extractTuples} \\
\verb'GxB_Matrix_extractTuples_Vector' & get all entries from a matrix  & \ref{matrix_extractTuples_Vector} \\
\verb'GrB_Matrix_resize'        & resize a matrix                       & \ref{matrix_resize} \\
\verb'GxB_Matrix_reshape'       & reshape a matrix                      & \ref{matrix_reshape} \\
\verb'GxB_Matrix_reshapeDup'    & reshape a matrix                      & \ref{matrix_reshapedup} \\
\verb'GxB_Matrix_concat'        & concatenate matrices                  & \ref{matrix_concat} \\
\verb'GxB_Matrix_split'         & split a matrix into matrices          & \ref{matrix_split} \\
\verb'GrB_Matrix_diag'          & diagonal matrix from vector           & \ref{matrix_diag} \\
\verb'GxB_Matrix_diag'          & diagonal matrix from vector           & \ref{matrix_diag_GxB} \\
\verb'GxB_Matrix_memoryUsage'   & memory used by a matrix               & \ref{matrix_memusage} \\
\verb'GxB_Matrix_type'          & type of the matrix                    & \ref{matrix_type} \\
\verb'GrB_Matrix_free'          & free a matrix                         & \ref{matrix_free} \\
\hline
\hline
\verb'GrB_Matrix_serializeSize' & return size of serialized matrix & \ref{matrix_serialize_size} \\
\verb'GrB_Matrix_serialize'     & serialize a matrix               & \ref{matrix_serialize} \\
\verb'GxB_Matrix_serialize'     & serialize a matrix               & \ref{matrix_serialize_GxB} \\
\verb'GrB_Matrix_deserialize'   & deserialize a matrix             & \ref{matrix_deserialize} \\
\verb'GxB_Matrix_deserialize'   & deserialize a matrix             & \ref{matrix_deserialize_GxB} \\
\hline
\hline
\verb'GrB_get'  & get properties of a matrix       & \ref{get_set_matrix} \\
\verb'GrB_set'  & set properties of a matrix       & \ref{get_set_matrix} \\
\hline
\end{tabular}
}

\vspace{0.2in}
\noindent
{\footnotesize
\begin{tabular}{lll}
\hline
GraphBLAS function   & purpose                                      & Section \\
\hline
\verb'GrB_Matrix_import'        & import in various formats & \ref{GrB_matrix_import} \\
\verb'GrB_Matrix_export'        & export in various formats & \ref{GrB_matrix_export} \\
\verb'GrB_Matrix_exportSize'    & array sizes for export & \ref{export_size} \\
\verb'GrB_Matrix_exportHint'    & hint best export format & \ref{export_hint} \\
\hline
\verb'GxB_Matrix_sort'          & sort a matrix & \ref{matrix_sort} \\
\hline
\end{tabular}
}

\vspace{0.2in}
Refer to
Section~\ref{serialize_deserialize} for serialization/deserialization methods,
Section~\ref{GrB_import_export} for \verb'GrB' import/export methods,
Section~\ref{sorting_methods} for sorting methods,
and Section~\ref{get_set_matrix} for get/set methods.

%-------------------------------------------------------------------------------
\subsubsection{{\sf GrB\_Matrix\_new:}          create a matrix}
%-------------------------------------------------------------------------------
\label{matrix_new}

\begin{mdframed}[userdefinedwidth=6in]
{\footnotesize
\begin{verbatim}
GrB_Info GrB_Matrix_new     // create a new matrix with no entries
(
    GrB_Matrix *A,          // handle of matrix to create
    GrB_Type type,          // type of matrix to create
    GrB_Index nrows,        // matrix dimension is nrows-by-ncols
    GrB_Index ncols
) ;
\end{verbatim} } \end{mdframed}

\verb'GrB_Matrix_new' creates a new \verb'nrows'-by-\verb'ncols' sparse matrix
with no entries in it, of the given type.  This is analogous to the MATLAB
statement \verb'A = sparse (nrows, ncols)', except that GraphBLAS can create
sparse matrices of any type.

By default, matrices of size \verb'nrows-by-1' are held by column, regardless
of the global setting controlled by \verb'GrB_set (GrB_GLOBAL, ...,' \newline
\verb'GrB_STORAGE_ORIENTATION_HINT)', for any value of \verb'nrows'.  Matrices
of size \verb'1-by-ncols' with \verb'ncols' not equal to 1 are held by row,
regardless of this global setting.  The global setting only affects matrices
with both \verb'm > 1' and \verb'n > 1'.  Empty matrices (\verb'0-by-0') are
also controlled by the global setting.

Once a matrix is created, its format (by-row or by-column) can be arbitrarily
changed with \verb'GrB_set (A, fmt, GrB_STORAGE_ORIENTATION_HINT)'
with \verb'fmt' equal to \verb'GrB_COLMAJOR' or \verb'GrB_ROWMAJOR'.

\begin{alert}
{\bf SPEC:} \verb'nrows' and/or \verb'ncols' may be zero.
as an extension to the specification.
\end{alert}

\newpage
%-------------------------------------------------------------------------------
\subsubsection{{\sf GrB\_Matrix\_wait:} wait for a matrix}
%-------------------------------------------------------------------------------
\label{matrix_wait}

\begin{mdframed}[userdefinedwidth=6in]
{\footnotesize
\begin{verbatim}
GrB_Info GrB_wait               // wait for a matrix
(
    GrB_Matrix C,               // matrix to wait for
    int mode                    // GrB_COMPLETE or GrB_MATERIALIZE
) ;
\end{verbatim}
}\end{mdframed}

In non-blocking mode, the computations for a \verb'GrB_Matrix' may be delayed.
In this case, the matrix is not yet safe to use by multiple independent user
threads.  A user application may force completion of a matrix \verb'C' via
\verb'GrB_Matrix_wait(C,mode)'.
With a \verb'mode' of \verb'GrB_MATERIALIZE',
all pending computations are finished, and afterwards different user threads may
simultaneously call GraphBLAS operations that use the matrix \verb'C' as an
input parameter.
See Section~\ref{omp_parallelism}
if GraphBLAS is compiled without OpenMP.

%-------------------------------------------------------------------------------
\subsubsection{{\sf GrB\_Matrix\_dup:}          copy a matrix}
%-------------------------------------------------------------------------------
\label{matrix_dup}

\begin{mdframed}[userdefinedwidth=6in]
{\footnotesize
\begin{verbatim}
GrB_Info GrB_Matrix_dup     // make an exact copy of a matrix
(
    GrB_Matrix *C,          // handle of output matrix to create
    const GrB_Matrix A      // input matrix to copy
) ;
\end{verbatim} } \end{mdframed}

\verb'GrB_Matrix_dup' makes a deep copy of a sparse matrix.
In GraphBLAS, it is possible, and valid, to write the following:

    {\footnotesize
    \begin{verbatim}
    GrB_Matrix A, C ;
    GrB_Matrix_new (&A, GrB_FP64, n) ;
    C = A ;                         // C is a shallow copy of A  \end{verbatim}}

Then \verb'C' and \verb'A' can be used interchangeably.  However, only a
pointer reference is made, and modifying one of them modifies both, and freeing
one of them leaves the other as a dangling handle that should not be used.  If
two different matrices are needed, then this should be used instead:

    {\footnotesize
    \begin{verbatim}
    GrB_Matrix A, C ;
    GrB_Matrix_new (&A, GrB_FP64, n) ;
    GrB_Matrix_dup (&C, A) ;        // like C = A, but making a deep copy \end{verbatim}}

Then \verb'C' and \verb'A' are two different matrices that currently have the
same set of values, but they do not depend on each other.  Modifying one has
no effect on the other.
The \verb'GrB_NAME' is copied into the new matrix.

%-------------------------------------------------------------------------------
\subsubsection{{\sf GrB\_Matrix\_clear:}        clear a matrix of all entries}
%-------------------------------------------------------------------------------
\label{matrix_clear}

\begin{mdframed}[userdefinedwidth=6in]
{\footnotesize
\begin{verbatim}
GrB_Info GrB_Matrix_clear   // clear a matrix of all entries;
(                           // type and dimensions remain unchanged
    GrB_Matrix A            // matrix to clear
) ;
\end{verbatim} } \end{mdframed}

\verb'GrB_Matrix_clear' clears all entries from a matrix.  All values
\verb'A(i,j)' are now equal to the implicit value, depending on what semiring
ring is used to perform computations on the matrix.  The pattern of \verb'A' is
empty, just as if it were created fresh with \verb'GrB_Matrix_new'.  Analogous
with \verb'A (:,:) = 0' in MATLAB.  The type and dimensions of \verb'A' do not
change.  Any pending updates to the matrix are discarded.

%-------------------------------------------------------------------------------
\subsubsection{{\sf GrB\_Matrix\_nrows:}        return the number of rows of a matrix}
%-------------------------------------------------------------------------------
\label{matrix_nrows}

\begin{mdframed}[userdefinedwidth=6in]
{\footnotesize
\begin{verbatim}
GrB_Info GrB_Matrix_nrows   // get the number of rows of a matrix
(
    GrB_Index *nrows,       // matrix has nrows rows
    const GrB_Matrix A      // matrix to query
) ;
\end{verbatim} } \end{mdframed}

\verb'GrB_Matrix_nrows' returns the number of rows of a matrix
(\verb'nrows=size(A,1)' in MATLAB).

%-------------------------------------------------------------------------------
\subsubsection{{\sf GrB\_Matrix\_ncols:}        return the number of columns of a matrix}
%-------------------------------------------------------------------------------
\label{matrix_ncols}

\begin{mdframed}[userdefinedwidth=6in]
{\footnotesize
\begin{verbatim}
GrB_Info GrB_Matrix_ncols   // get the number of columns of a matrix
(
    GrB_Index *ncols,       // matrix has ncols columns
    const GrB_Matrix A      // matrix to query
) ;
\end{verbatim}
} \end{mdframed}

\verb'GrB_Matrix_ncols' returns the number of columns of a matrix
(\verb'ncols=size(A,2)' in MATLAB).

\newpage
%-------------------------------------------------------------------------------
\subsubsection{{\sf GrB\_Matrix\_nvals:}        return the number of entries in a matrix}
%-------------------------------------------------------------------------------
\label{matrix_nvals}

\begin{mdframed}[userdefinedwidth=6in]
{\footnotesize
\begin{verbatim}
GrB_Info GrB_Matrix_nvals   // get the number of entries in a matrix
(
    GrB_Index *nvals,       // matrix has nvals entries
    const GrB_Matrix A      // matrix to query
) ;
\end{verbatim} } \end{mdframed}

\verb'GrB_Matrix_nvals' returns the number of entries in a matrix.  Roughly
analogous to \verb'nvals = nnz(A)' in MATLAB, except that the implicit value in
GraphBLAS need not be zero and \verb'nnz' (short for ``number of nonzeros'') in
MATLAB is better described as ``number of entries'' in GraphBLAS.

%-------------------------------------------------------------------------------
\subsubsection{{\sf GrB\_Matrix\_build:} build a matrix from a set of tuples}
%-------------------------------------------------------------------------------
\label{matrix_build}

\begin{mdframed}[userdefinedwidth=6in]
{\footnotesize
\begin{verbatim}
GrB_Info GrB_Matrix_build           // build a matrix from (I,J,X) tuples
(
    GrB_Matrix C,                   // matrix to build
    const GrB_Index *I,             // array of row indices of tuples
    const GrB_Index *J,             // array of column indices of tuples
    const <type> *X,                // array of values of tuples
    GrB_Index nvals,                // number of tuples
    const GrB_BinaryOp dup          // binary function to assemble duplicates
) ;
\end{verbatim} } \end{mdframed}

\verb'GrB_Matrix_build' constructs a sparse matrix \verb'C' from a set of
tuples, \verb'I', \verb'J', and \verb'X', each of length \verb'nvals'.  The
matrix \verb'C' must have already been initialized with \verb'GrB_Matrix_new',
and it must have no entries in it before calling \verb'GrB_Matrix_build'.  Thus
the dimensions and type of \verb'C' are not changed by this function, but are
inherited from the prior call to \verb'GrB_Matrix_new' or
\verb'GrB_matrix_dup'.

An error is returned (\verb'GrB_INDEX_OUT_OF_BOUNDS') if any row index in
\verb'I' is greater than or equal to the number of rows of \verb'C', or if any
column index in \verb'J' is greater than or equal to the number of columns of
\verb'C'

Any duplicate entries with identical indices are assembled using the binary
\verb'dup' operator provided on input.  All three types (\verb'x', \verb'y',
\verb'z' for \verb'z=dup(x,y)') must be identical.  The types of \verb'dup',
\verb'C' and \verb'X' must all be compatible.  See Section~\ref{typecasting}
regarding typecasting and compatibility.  The values in \verb'X' are
typecasted, if needed, into the type of \verb'dup'.  Duplicates are then
assembled into a matrix \verb'T' of the same type as \verb'dup', using
\verb'T(i,j) = dup (T (i,j), X (k))'.  After \verb'T' is constructed, it is
typecasted into the result \verb'C'.  That is, typecasting does not occur at
the same time as the assembly of duplicates.

If \verb'dup' is \verb'NULL', any duplicates result in an error.
If \verb'dup' is the special binary operator \verb'GxB_IGNORE_DUP', then
any duplicates are ignored.  If duplicates appear, the last one in the
list of tuples is taken and the prior ones ignored.  This is not an error.

\begin{alert}
{\bf SPEC:} As an extension to the specification, results are defined even if
\verb'dup' is non-associative and/or non-commutative.
\end{alert}

The GraphBLAS API requires \verb'dup' to be associative so
that entries can be assembled in any order, and states that the result is
undefined if \verb'dup' is not associative.  However, SuiteSparse:GraphBLAS
guarantees a well-defined order of assembly.  Entries in the tuples
\verb'[I,J,X]' are first sorted in increasing order of row and column index,
with ties broken by the position of the tuple in the \verb'[I,J,X]' list.  If
duplicates appear, they are assembled in the order they appear in the
\verb'[I,J,X]' input.  That is, if the same indices \verb'i' and \verb'j'
appear in positions \verb'k1', \verb'k2', \verb'k3', and \verb'k4' in
\verb'[I,J,X]', where \verb'k1 < k2 < k3 < k4', then the following operations
will occur in order:

    {\footnotesize
    \begin{verbatim}
    T (i,j) = X (k1) ;
    T (i,j) = dup (T (i,j), X (k2)) ;
    T (i,j) = dup (T (i,j), X (k3)) ;
    T (i,j) = dup (T (i,j), X (k4)) ; \end{verbatim}}

This is a well-defined order but the user should not depend upon it when using
other GraphBLAS implementations since the GraphBLAS API does not
require this ordering.

However, SuiteSparse:GraphBLAS guarantees this ordering, even when it compute
the result in parallel.  With this well-defined order, several operators become
very useful.  In particular, the \verb'SECOND' operator results in the last
tuple overwriting the earlier ones.  The \verb'FIRST' operator means the value
of the first tuple is used and the others are discarded.

The acronym \verb'dup' is used here for the name of binary function used for
assembling duplicates, but this should not be confused with the \verb'_dup'
suffix in the name of the function \verb'GrB_Matrix_dup'.  The latter function
does not apply any operator at all, nor any typecasting, but simply makes a
pure deep copy of a matrix.

The parameter \verb'X' is a pointer to any C equivalent built-in type, or a
\verb'void *' pointer.  The \verb'GrB_Matrix_build' function uses the
\verb'_Generic' feature of C11 to detect the type of pointer passed as the
parameter \verb'X'.  If \verb'X' is a pointer to a built-in type, then the
function can do the right typecasting.  If \verb'X' is a \verb'void *' pointer,
then it can only assume \verb'X' to be a pointer to a user-defined type that is
the same user-defined type of \verb'C' and \verb'dup'.  This function has no
way of checking this condition that the \verb'void * X' pointer points to an
array of the correct user-defined type, so behavior is undefined if the user
breaks this condition.

The \verb'GrB_Matrix_build' method is analogous to \verb'C = sparse (I,J,X)' in
MATLAB, with several important extensions that go beyond that which MATLAB can
do.  In particular, the MATLAB \verb'sparse' function only provides one option
for assembling duplicates (summation), and it can only build double, double
complex, and logical sparse matrices.
%
The \verb'dup' operator cannot be a binary operator
created by \verb'GxB_BinaryOp_new_IndexOp'.

%-------------------------------------------------------------------------------
\subsubsection{{\sf GxB\_Matrix\_build\_Vector:} build a matrix from a set of tuples}
%-------------------------------------------------------------------------------
\label{matrix_build_Vector}

\begin{mdframed}[userdefinedwidth=6in]
{\footnotesize
\begin{verbatim}
GrB_Info GrB_Matrix_build           // build a matrix from (I,J,X) tuples
(
    GrB_Matrix C,               // matrix to build
    const GrB_Vector I_vector,  // row indices
    const GrB_Vector J_vector,  // col indices
    const GrB_Vector X_vector,  // values
    const GrB_BinaryOp dup,     // binary function to assemble duplicates
    const GrB_Descriptor desc
) ;
\end{verbatim} } \end{mdframed}

\verb'GxB_Matrix_build_Vector' is identical to \verb'GrB_Matrix_build', except
that the inputs \verb'I', \verb'J', and \verb'X' are \verb'GrB_Vector' objects,
each with \verb'nvals' entries.  The interpretation of \verb'I_vector',
\verb'J_vector', and \verb'X_vector' are controlled by descriptor settings
\verb'GxB_ROWINDEX_LIST', \verb'GxB_COLINDEX_LIST', and \verb'GxB_VALUE_LIST',
respectively.  The method can use either the indices or values of each of the
input vectors; the default is to use the values.  See Section~\ref{ijxvector} for
details.

\newpage
%-------------------------------------------------------------------------------
\subsubsection{{\sf GxB\_Matrix\_build\_Scalar:} build a matrix from a set of tuples}
%-------------------------------------------------------------------------------
\label{matrix_build_Scalar}

\begin{mdframed}[userdefinedwidth=6in]
{\footnotesize
\begin{verbatim}
GrB_Info GrB_Matrix_build           // build a matrix from (I,J,scalar) tuples
(
    GrB_Matrix C,                   // matrix to build
    const GrB_Index *I,             // array of row indices of tuples
    const GrB_Index *J,             // array of column indices of tuples
    GrB_Scalar scalar,              // value for all tuples
    GrB_Index nvals                 // number of tuples
) ;
\end{verbatim} } \end{mdframed}

\verb'GxB_Matrix_build_Scalar' constructs a sparse matrix \verb'C' from a set
of tuples defined the index arrays \verb'I' and \verb'J' of length
\verb'nvals', and a scalar.  The scalar is the value of all of the tuples.
Unlike \verb'GrB_Matrix_build', there is no \verb'dup' operator to handle
duplicate entries.  Instead, any duplicates are silently ignored (if the number
of duplicates is desired, simply compare the input \verb'nvals' with the value
returned by \verb'GrB_Vector_nvals' after the matrix is constructed).  All
entries in the sparsity pattern of \verb'C' are identical, and equal to the
input scalar value.

%-------------------------------------------------------------------------------
\subsubsection{{\sf GxB\_Matrix\_build\_Scalar\_Vector:} build a matrix from a set of tuples}
%-------------------------------------------------------------------------------
\label{matrix_build_Scalar_Vector}

\begin{mdframed}[userdefinedwidth=6in]
{\footnotesize
\begin{verbatim}
GrB_Info GrB_Matrix_build       // build a matrix from (I,J,scalar) tuples
(
    GrB_Matrix C,               // matrix to build
    const GrB_Vector I_vector,  // row indices
    const GrB_Vector J_vector,  // col indices
    GrB_Scalar scalar,          // value for all tuples
    const GrB_Descriptor desc
) ;
\end{verbatim} } \end{mdframed}

\verb'GxB_Matrix_build_Scalar_Vector' is identical to
\verb'GxB_Matrix_build_Scalar', except that the inputs \verb'I', \verb'J', and
\verb'X' are \verb'GrB_Vector' objects, each with \verb'nvals' entries.  The
interpretation of \verb'I_vector' and \verb'J_vector' are controlled by
descriptor settings \verb'GxB_ROWINDEX_LIST' and \verb'GxB_VALUE_LIST',
respectively.  The method can use either the indices or values of the
\verb'I_input' and \verb'J_vector' vectors; the default is to use the values.
See Section~\ref{ijxvector} for details.

\newpage
%-------------------------------------------------------------------------------
\subsubsection{{\sf GrB\_Matrix\_setElement:}   add an entry to a matrix}
%-------------------------------------------------------------------------------
\label{matrix_setElement}

\begin{mdframed}[userdefinedwidth=6in]
{\footnotesize
\begin{verbatim}
GrB_Info GrB_Matrix_setElement          // C (i,j) = x
(
    GrB_Matrix C,                       // matrix to modify
    <type> x,                           // scalar to assign to C(i,j)
    GrB_Index i,                        // row index
    GrB_Index j                         // column index
) ;
\end{verbatim} } \end{mdframed}

\verb'GrB_Matrix_setElement' sets a single entry in a matrix, \verb'C(i,j)=x'.
If the entry is already present in the pattern of \verb'C', it is overwritten
with the new value.  If the entry is not present, it is added to \verb'C'.  In
either case, no entry is ever deleted by this function.  Passing in a value of
\verb'x=0' simply creates an explicit entry at position \verb'(i,j)' whose
value is zero, even if the implicit value is assumed to be zero.

An error is returned (\verb'GrB_INVALID_INDEX') if the row index \verb'i' is
greater than or equal to the number of rows of \verb'C', or if the column index
\verb'j' is greater than or equal to the number of columns of \verb'C'.  Note
that this error code differs from the same kind of condition in
\verb'GrB_Matrix_build', which returns \verb'GrB_INDEX_OUT_OF_BOUNDS'.  This is
because \verb'GrB_INVALID_INDEX' is an API error, and is caught immediately
even in non-blocking mode, whereas \verb'GrB_INDEX_OUT_OF_BOUNDS' is an
execution error whose detection may wait until the computation completes
sometime later.

The scalar \verb'x' is typecasted into the type of \verb'C'.  Any value can be
passed to this function and its type will be detected, via the \verb'_Generic'
feature of C11.  For a user-defined type, \verb'x' is a \verb'void *'
pointer that points to a memory space holding a single entry of this
user-defined type.  This user-defined type must exactly match the user-defined
type of \verb'C' since no typecasting is done between user-defined types.
%
If \verb'x' is a \verb'GrB_Scalar' and contains no entry, then the
entry \verb'C(i,j)' is removed (if it exists).  The action taken is
identical to \verb'GrB_Matrix_removeElement(C,i,j)' in this case.

{\bf Performance considerations:} % BLOCKING: setElement, *assign
SuiteSparse:GraphBLAS exploits the non-blocking mode to greatly improve the
performance of this method.  Refer to the example shown in
Section~\ref{overview}.  If the entry exists in the pattern already, it is
updated right away and the work is not left pending.  Otherwise, it is placed
in a list of pending updates, and the later on the updates are done all at
once, using the same algorithm used for \verb'GrB_Matrix_build'.  In other
words, \verb'setElement' in SuiteSparse:GraphBLAS builds its own internal list
of tuples \verb'[I,J,X]', and then calls \verb'GrB_Matrix_build' whenever the
matrix is needed in another computation, or whenever \verb'GrB_Matrix_wait' is
called.

As a result, if calls to \verb'setElement' are mixed with calls to most other
methods and operations (even \verb'extractElement') then the pending updates
are assembled right away, which will be slow.  Performance will be good if many
\verb'setElement' updates are left pending, and performance will be poor if the
updates are assembled frequently.

A few methods and operations can be intermixed with \verb'setElement', in
particular, some forms of the \verb'GrB_assign' and \verb'GxB_subassign'
operations are compatible with the pending updates from \verb'setElement'.
Section~\ref{compare_assign} gives more details on which \verb'GxB_subassign'
and \verb'GrB_assign' operations can be interleaved with calls to
\verb'setElement' without forcing updates to be assembled.  Other methods that
do not access the existing entries may also be done without forcing the updates
to be assembled, namely \verb'GrB_Matrix_clear' (which erases all pending
updates), \verb'GrB_Matrix_free', \verb'GrB_Matrix_ncols',
\verb'GrB_Matrix_nrows', \verb'GrB_get', and of course
\verb'GrB_Matrix_setElement' itself.  All other methods and operations cause
the updates to be assembled.  Future versions of SuiteSparse:GraphBLAS may
extend this list.

See Section~\ref{random} for an example of how to use
\verb'GrB_Matrix_setElement'.
If an error occurs, \verb'GrB_error(&err,C)' returns details about the error.

%-------------------------------------------------------------------------------
\subsubsection{{\sf GrB\_Matrix\_extractElement:} get an entry from a matrix}
%-------------------------------------------------------------------------------
\label{matrix_extractElement}

\begin{mdframed}[userdefinedwidth=6in]
{\footnotesize
\begin{verbatim}
GrB_Info GrB_Matrix_extractElement      // x = A(i,j)
(
    <type> *x,                  // extracted scalar (non-opaque C scalar)
    const GrB_Matrix A,         // matrix to extract a scalar from
    GrB_Index i,                // row index
    GrB_Index j                 // column index
) ;
GrB_Info GrB_Matrix_extractElement      // x = A(i,j)
(
    GrB_Scalar x,               // extracted GrB_Scalar
    const GrB_Matrix A,         // matrix to extract a scalar from
    GrB_Index i,                // row index
    GrB_Index j                 // column index
) ;
\end{verbatim} } \end{mdframed}

\verb'GrB_Matrix_extractElement' extracts a single entry from a matrix
\verb'x=A(i,j)'.
An error is returned (\verb'GrB_INVALID_INDEX') if the row index \verb'i' is
greater than or equal to the number of rows of \verb'C', or if column index
\verb'j' is greater than or equal to the number of columns of \verb'C'.
If the entry is present, \verb'x=A(i,j)' is performed and the scalar \verb'x'
is returned with this value.  The method returns \verb'GrB_SUCCESS'.
If no entry is present at \verb'A(i,j)', and \verb'x' is a non-opaque C scalar,
then \verb'x' is not modified, and the return value of
\verb'GrB_Matrix_extractElement' is \verb'GrB_NO_VALUE'.  If \verb'x' is a
\verb'GrB_Scalar', then \verb'x' is returned as an empty scalar with no entry,
and \verb'GrB_SUCCESS' is returned.

The function knows the type of the pointer \verb'x', so it can do typecasting
as needed, from the type of \verb'A' into the type of \verb'x'.  User-defined
types cannot be typecasted, so if \verb'A' has a user-defined type then
\verb'x' must be a \verb'void *' pointer that points to a memory space the same
size as a single scalar of the type of \verb'A'.

Currently, this method causes all pending updates from
\verb'GrB_setElement', \verb'GrB_assign', or \verb'GxB_subassign' to be
assembled, so its use can have performance implications.  Calls to this
function should not be arbitrarily intermixed with calls to these other two
functions.  Everything will work correctly and results will be predictable, it
will just be slow.

%-------------------------------------------------------------------------------
\subsubsection{{\sf GxB\_Matrix\_isStoredElement:} check if entry present in matrix}
%-------------------------------------------------------------------------------
\label{matrix_isStoredElement}

\begin{mdframed}[userdefinedwidth=6in]
{\footnotesize
\begin{verbatim}
GrB_Info GxB_Matrix_isStoredElement
(
    const GrB_Matrix A,         // check for A(i,j)
    GrB_Index i,                // row index
    GrB_Index j                 // column index
) ;
\end{verbatim} } \end{mdframed}

\verb'GxB_Matrix_isStoredElement' check if the single entry \verb'A(i,j)' is
present in the matrix \verb'A'.  It returns \verb'GrB_SUCCESS' if the entry is
present, or \verb'GrB_NO_VALUE' otherwise.  The value of \verb'A(i,j)' is not
returned. It is otherwise identical to \verb'GrB_Matrix_extractElement'.

\newpage
%-------------------------------------------------------------------------------
\subsubsection{{\sf GrB\_Matrix\_removeElement:} remove an entry from a matrix}
%-------------------------------------------------------------------------------
\label{matrix_removeElement}

\begin{mdframed}[userdefinedwidth=6in]
{\footnotesize
\begin{verbatim}
GrB_Info GrB_Matrix_removeElement
(
    GrB_Matrix C,                   // matrix to remove an entry from
    GrB_Index i,                    // row index
    GrB_Index j                     // column index
) ;
\end{verbatim} } \end{mdframed}

\verb'GrB_Matrix_removeElement' removes a single entry \verb'A(i,j)' from a
matrix.  If no entry is present at \verb'A(i,j)', then the matrix is not
modified.  If an error occurs, \verb'GrB_error(&err,A)' returns details about
the error.

%-------------------------------------------------------------------------------
\subsubsection{{\sf GrB\_Matrix\_extractTuples:} get all entries from a matrix}
%-------------------------------------------------------------------------------
\label{matrix_extractTuples}

\begin{mdframed}[userdefinedwidth=6in]
{\footnotesize
\begin{verbatim}
GrB_Info GrB_Matrix_extractTuples           // [I,J,X] = find (A)
(
    GrB_Index *I,               // array for returning row indices of tuples
    GrB_Index *J,               // array for returning col indices of tuples
    <type> *X,                  // array for returning values of tuples
    GrB_Index *nvals,           // I,J,X size on input; # tuples on output
    const GrB_Matrix A          // matrix to extract tuples from
) ;
\end{verbatim} } \end{mdframed}

\verb'GrB_Matrix_extractTuples' extracts all the entries from the matrix
\verb'A', returning them as a list of tuples, analogous to
\verb'[I,J,X]=find(A)' in MATLAB.  Entries in the tuples \verb'[I,J,X]' are
unique.  No pair of row and column indices \verb'(i,j)' appears more than once.

The GraphBLAS API states the tuples can be returned in any order.  If
\verb'GrB_wait' is called first, then SuiteSparse:GraphBLAS chooses to
always return them in sorted order, depending on whether the matrix is stored
by row or by column.  Otherwise, the indices can be returned in any order.

The number of tuples in the matrix \verb'A' is given by
\verb'GrB_Matrix_nvals(&anvals,A)'.  If \verb'anvals' is larger than the size
of the arrays (\verb'nvals' in the parameter list), an error
\verb'GrB_INSUFFICIENT_SIZE' is returned, and no tuples are extracted.  If
\verb'nvals' is larger than \verb'anvals', then only the first \verb'anvals'
entries in the arrays \verb'I' \verb'J', and \verb'X' are modified, containing
all the tuples of \verb'A', and the rest of \verb'I' \verb'J', and \verb'X' are
left unchanged.  On output, \verb'nvals' contains the number of tuples
extracted.

\begin{alert}
{\bf SPEC:} As an extension to the specification, the arrays \verb'I',
\verb'J', and/or \verb'X' may be passed in as \verb'NULL' pointers.
\verb'GrB_Matrix_extractTuples' does not return a component specified as
\verb'NULL'.  This is not an error condition.
\end{alert}

%-------------------------------------------------------------------------------
\subsubsection{{\sf GxB\_Matrix\_extractTuples\_Vector:} get all entries from a matrix}
%-------------------------------------------------------------------------------
\label{matrix_extractTuples_Vector}

\begin{mdframed}[userdefinedwidth=6in]
{\footnotesize
\begin{verbatim}
GrB_Info GrB_Matrix_extractTuples   // [I,J,X] = find (A)
(
    GrB_Vector I_vector,    // row indices
    GrB_Vector J_vector,    // col indices
    GrB_Vector X_vector,    // values
    const GrB_Matrix A,     // matrix to extract tuples from
    const GrB_Descriptor desc   // currently unused; for future expansion
) ;
\end{verbatim} } \end{mdframed}

\verb'GxB_Matrix_extractTuples_Vector' is identical to
\verb'GrB_Matrix_extractTuples' except that its three outputs are
\verb'GrB_Vector' objects.  The vectors \verb'I_vector', \verb'J_vector', and
\verb'X_vector' objects must exist on input.  On output, any prior content is
erased and their type, dimensions, and values are revised to contain dense
vectors of length \verb'nvals'.

%-------------------------------------------------------------------------------
\subsubsection{{\sf GrB\_Matrix\_resize:}          resize a matrix}
%-------------------------------------------------------------------------------
\label{matrix_resize}

\begin{mdframed}[userdefinedwidth=6in]
{\footnotesize
\begin{verbatim}
GrB_Info GrB_Matrix_resize      // change the size of a matrix
(
    GrB_Matrix A,               // matrix to modify
    const GrB_Index nrows_new,  // new number of rows in matrix
    const GrB_Index ncols_new   // new number of columns in matrix
) ;
\end{verbatim} } \end{mdframed}

\verb'GrB_Matrix_resize' changes the size of a matrix.  If the dimensions
decrease, entries that fall outside the resized matrix are deleted.  Unlike
\verb'GxB_Matrix_reshape*' (see Sections \ref{matrix_reshape} and
\ref{matrix_reshapedup}), entries remain in their same position after resizing
the matrix.

\newpage
%-------------------------------------------------------------------------------
\subsubsection{{\sf GxB\_Matrix\_reshape:} reshape a matrix}
%-------------------------------------------------------------------------------
\label{matrix_reshape}

\begin{mdframed}[userdefinedwidth=6in]
{\footnotesize
\begin{verbatim}
GrB_Info GxB_Matrix_reshape     // reshape a GrB_Matrix in place
(
    // input/output:
    GrB_Matrix C,               // input/output matrix, reshaped in place
    // input:
    bool by_col,                // true if reshape by column, false if by row
    GrB_Index nrows_new,        // new number of rows of C
    GrB_Index ncols_new,        // new number of columns of C
    const GrB_Descriptor desc
) ;
\end{verbatim} } \end{mdframed}

\verb'GxB_Matrix_reshape' changes the size of a matrix \verb'C', taking entries
from the input matrix either column-wise or row-wise.  If matrix \verb'C' on
input is \verb'nrows'-by-\verb'ncols', and the requested dimensions of
\verb'C' on output are \verb'nrows_new'-by-\verb'nrows_cols', then
the condition \verb'nrows*ncols == nrows_new*nrows_cols' must hold.
The matrix \verb'C' is modified in-place, as both an input and output for
this method.  To create a new matrix, use \verb'GxB_Matrix_reshapeDup'
instead (Section \ref{matrix_reshapedup}).

For example, if \verb'C' is 3-by-4 on input, and is reshaped column-wise to
have dimensions 2-by-6:

\begin{verbatim}
        C on input      C on output (by_col true)
        00 01 02 03     00 20 11 02 22 13
        10 11 12 13     10 01 21 12 03 23
        20 21 22 23
\end{verbatim}

If the same \verb'C' on input is reshaped row-wise to dimensions 2-by-6:

\begin{verbatim}
        C on input      C on output (by_col false)
        00 01 02 03     00 01 02 03 10 11
        10 11 12 13     12 13 20 21 22 23
        20 21 22 23
\end{verbatim}

NOTE: because an intermediate linear index must be computed for each entry,
\verb'GxB_Matrix_reshape' cannot be used on matrices for which
\verb'nrows*ncols' exceeds $2^{60}$.

\newpage
%-------------------------------------------------------------------------------
\subsubsection{{\sf GxB\_Matrix\_reshapeDup:} reshape a matrix}
%-------------------------------------------------------------------------------
\label{matrix_reshapedup}

\begin{mdframed}[userdefinedwidth=6in]
{\footnotesize
\begin{verbatim}
GrB_Info GxB_Matrix_reshapeDup // reshape a GrB_Matrix into another GrB_Matrix
(
    // output:
    GrB_Matrix *C,              // newly created output matrix, not in place
    // input:
    GrB_Matrix A,               // input matrix, not modified
    bool by_col,                // true if reshape by column, false if by row
    GrB_Index nrows_new,        // number of rows of C
    GrB_Index ncols_new,        // number of columns of C
    const GrB_Descriptor desc
) ;
\end{verbatim} } \end{mdframed}

\verb'GxB_Matrix_reshapeDup' is identical to \verb'GxB_Matrix_reshape' (see
Section \ref{matrix_reshape}), except that creates a new output matrix
\verb'C' that is reshaped from the input matrix \verb'A'.

%-------------------------------------------------------------------------------
\subsubsection{{\sf GxB\_Matrix\_concat:} concatenate matrices   }
%-------------------------------------------------------------------------------
\label{matrix_concat}

\begin{mdframed}[userdefinedwidth=6in]
{\footnotesize
\begin{verbatim}
GrB_Info GxB_Matrix_concat          // concatenate a 2D array of matrices
(
    GrB_Matrix C,                   // input/output matrix for results
    const GrB_Matrix *Tiles,        // 2D row-major array of size m-by-n
    const GrB_Index m,
    const GrB_Index n,
    const GrB_Descriptor desc       // unused, except threading control
) ;
\end{verbatim} } \end{mdframed}

\verb'GxB_Matrix_concat' concatenates an array of matrices (\verb'Tiles') into
a single \verb'GrB_Matrix' \verb'C'.

\verb'Tiles' is an \verb'm'-by-\verb'n' dense array of matrices held in
row-major format, where \verb'Tiles [i*n+j]' is the $(i,j)$th tile, and where
\verb'm' $> 0$ and \verb'n' $> 0$ must hold.  Let $A_{i,j}$ denote the
$(i,j)$th tile.  The matrix \verb'C' is constructed by concatenating these
tiles together, as:

\[
C =
\left[
\begin{array}{ccccc}
          A_{0,0}   & A_{0,1}   & A_{0,2}   & \cdots & A_{0,n-1}   \\
          A_{1,0}   & A_{1,1}   & A_{1,2}   & \cdots & A_{1,n-1}   \\
          \cdots    &                                              \\
          A_{m-1,0} & A_{m-1,1} & A_{m-1,2} & \cdots & A_{m-1,n-1}
\end{array}
\right]
\]

On input, the matrix \verb'C' must already exist.  Any existing entries in
\verb'C' are discarded.  \verb'C' must have dimensions \verb'nrows' by
\verb'ncols' where \verb'nrows' is the sum of the number of rows in the
matrices $A_{i,0}$ for all $i$, and \verb'ncols' is the sum of the number of
columns in the matrices $A_{0,j}$ for all $j$.  All matrices in any given tile
row $i$ must have the same number of rows (that is, and all matrices in any
given tile column $j$ must have the same number of columns).

The type of \verb'C' is unchanged, and all matrices $A_{i,j}$ are typecasted
into the type of \verb'C'.  Any settings made to \verb'C' by
\verb'GrB_set' (format by row or by column, bitmap switch, hyper
switch, and sparsity control) are unchanged.

%-------------------------------------------------------------------------------
\subsubsection{{\sf GxB\_Matrix\_split:} split a matrix   }
%-------------------------------------------------------------------------------
\label{matrix_split}

\begin{mdframed}[userdefinedwidth=6in]
{\footnotesize
\begin{verbatim}
GrB_Info GxB_Matrix_split           // split a matrix into 2D array of matrices
(
    GrB_Matrix *Tiles,              // 2D row-major array of size m-by-n
    const GrB_Index m,
    const GrB_Index n,
    const GrB_Index *Tile_nrows,    // array of size m
    const GrB_Index *Tile_ncols,    // array of size n
    const GrB_Matrix A,             // input matrix to split
    const GrB_Descriptor desc       // unused, except threading control
) ;
\end{verbatim} } \end{mdframed}

\verb'GxB_Matrix_split' does the opposite of \verb'GxB_Matrix_concat'.  It
splits a single input matrix \verb'A' into a 2D array of tiles.  On input, the
\verb'Tiles' array must be a non-\verb'NULL' pointer to a previously allocated
array of size at least \verb'm*n' where both \verb'm' and \verb'n' must be
greater than zero.  The \verb'Tiles_nrows' array has size \verb'm', and
\verb'Tiles_ncols' has size \verb'n'.  The $(i,j)$th tile has dimension
\verb'Tiles_nrows[i]'-by-\verb'Tiles_ncols[j]'.  The sum of
\verb'Tiles_nrows [0:m-1]' must equal the number of rows of \verb'A', and the
sum of \verb'Tiles_ncols [0:n-1]' must equal the number of columns of \verb'A'.
The type of each tile is the same as the type of \verb'A'; no typecasting is
done.

%-------------------------------------------------------------------------------
\subsubsection{{\sf GrB\_Matrix\_diag:} construct a diagonal matrix}
%-------------------------------------------------------------------------------
\label{matrix_diag}

\begin{mdframed}[userdefinedwidth=6in]
{\footnotesize
\begin{verbatim}
GrB_Info GrB_Matrix_diag    // construct a diagonal matrix from a vector
(
    GrB_Matrix *C,                  // output matrix
    const GrB_Vector v,             // input vector
    int64_t k
) ;
\end{verbatim} } \end{mdframed}

\verb'GrB_Matrix_diag' constructs a matrix from a vector.  Let $n$ be the
length of the \verb'v' vector, from \verb'GrB_Vector_size (&n, v)'.  If
\verb'k' = 0, then \verb'C' is an $n$-by-$n$ diagonal matrix with the entries
from \verb'v' along the main diagonal of \verb'C', with \verb'C(i,i)=v(i)'.  If
\verb'k' is nonzero, \verb'C' is square with dimension $n+|k|$.  If \verb'k' is
positive, it denotes diagonals above the main diagonal, with
\verb'C(i,i+k)=v(i)'.
If \verb'k' is negative, it denotes diagonals below the main diagonal of
\verb'C', with \verb'C(i-k,i)=v(i)'.  This behavior is identical to the MATLAB
statement \verb'C=diag(v,k)', where \verb'v' is a vector.

The output matrix \verb'C' is a newly-constructed square matrix with the
same type as the input vector \verb'v'.  No typecasting is performed.

%-------------------------------------------------------------------------------
\subsubsection{{\sf GxB\_Matrix\_diag:} build a diagonal matrix}
%-------------------------------------------------------------------------------
\label{matrix_diag_GxB}

\begin{mdframed}[userdefinedwidth=6in]
{\footnotesize
\begin{verbatim}
GrB_Info GxB_Matrix_diag    // build a diagonal matrix from a vector
(
    GrB_Matrix C,                   // output matrix
    const GrB_Vector v,             // input vector
    int64_t k,
    const GrB_Descriptor desc       // unused, except threading control
) ;
\end{verbatim} } \end{mdframed}

Identical to \verb'GrB_Matrix_diag', except for the extra parameter
(a \verb'descriptor' to provide control over the number of threads used),
and this method is not a constructor.

The matrix \verb'C' must already exist on input, of the correct size.  It must
be square of dimension $n+|k|$ where the vector \verb'v' has length $n$.  Any
existing entries in \verb'C' are discarded.  The type of \verb'C' is preserved,
so that if the type of \verb'C' and \verb'v' differ, the entries are typecasted
into the type of \verb'C'.  Any settings made to \verb'C' by
\verb'GrB_set' (format by row or by column, bitmap switch, hyper
switch, and sparsity control) are unchanged.

%-------------------------------------------------------------------------------
\subsubsection{{\sf GxB\_Matrix\_memoryUsage:} memory used by a matrix}
%-------------------------------------------------------------------------------
\label{matrix_memusage}

\begin{mdframed}[userdefinedwidth=6in]
{\footnotesize
\begin{verbatim}
GrB_Info GxB_Matrix_memoryUsage  // return # of bytes used for a matrix
(
    size_t *size,           // # of bytes used by the matrix A
    const GrB_Matrix A      // matrix to query
) ;
\end{verbatim} } \end{mdframed}

Returns the memory space required for a matrix, in bytes.
By default, any read-only components are not included in the total memory.
This can be changed with via \verb'GrB_set'; see Section~\ref{get_set_global}.

\newpage
%-------------------------------------------------------------------------------
\subsubsection{{\sf GxB\_Matrix\_type:} type of a matrix}
%-------------------------------------------------------------------------------
\label{matrix_type}

\begin{mdframed}[userdefinedwidth=6in]
{\footnotesize
\begin{verbatim}
GrB_Info GxB_Matrix_type    // get the type of a matrix
(
    GrB_Type *type,         // returns the type of the matrix
    const GrB_Matrix A      // matrix to query
) ;
\end{verbatim} } \end{mdframed}

Returns the type of a matrix.  The \verb'type' parameter is not allocated.
Calling \verb'GxB_Matrix_type' is identical to making a shallow pointer copy
of the type used to create a matrix.  In particular, suppose a matrix is
created, and a copy of its type is saved at the same time:

{\footnotesize
\begin{verbatim}
    GrB_Matrix_new (&A, atype, m, n) ;
    GrB_Type save_type = atype ;
\end{verbatim}}

Sometime later, while the matrix \verb'A' and its type \verb'atype' have not
been freed, the following two code fragments are identical:

{\footnotesize
\begin{verbatim}
    // using GxB_Matrix_type:
    GrB_Type atype2 ;
    GxB_Matrix_type (&atype2, A) ;
    assert (atype2 == save_type) ;

    // without GxB_Matrix_type:
    GrB_Type atype2 = save_type ;
\end{verbatim}}

As a result, freeing \verb'atype2' would be the same as freeing the original
\verb'atype'.

%-------------------------------------------------------------------------------
\subsubsection{{\sf GrB\_Matrix\_free:} free a matrix}
%-------------------------------------------------------------------------------
\label{matrix_free}

\begin{mdframed}[userdefinedwidth=6in]
{\footnotesize
\begin{verbatim}
GrB_Info GrB_free           // free a matrix
(
    GrB_Matrix *A           // handle of matrix to free
) ;
\end{verbatim} } \end{mdframed}

\verb'GrB_Matrix_free' frees a matrix.  Either usage:

    {\small
    \begin{verbatim}
    GrB_Matrix_free (&A) ;
    GrB_free (&A) ; \end{verbatim}}

\noindent
frees the matrix \verb'A' and sets \verb'A' to \verb'NULL'.  It safely does
nothing if passed a \verb'NULL' handle, or if \verb'A == NULL' on input.  Any
pending updates to the matrix are abandoned.




\newpage
%===============================================================================
\subsection{Serialize/deserialize methods}
%===============================================================================
\label{serialize_deserialize}

{\em Serialization} takes an opaque GraphBLAS object (a vector or matrix) and
encodes it in a single non-opaque array of bytes, the {\em blob}.  The blob can
only be deserialized by the same library that created it (SuiteSparse:GraphBLAS
in this case).  The array of bytes can be written to a file, sent to another
process over an MPI channel, or operated on in any other way that moves the
bytes around.  The contents of the array cannot be interpreted except by
deserialization back into a vector or matrix, by the same library (and
sometimes the same version) that created the blob.

All versions of SuiteSparse:GraphBLAS that implement
serialization/deserialization use essentially the same format for the blob, so
the library versions are compatible with each other.  Version v9.0.0 adds the
\verb'GrB_NAME' and \verb'GrB_EL_TYPE_STRING' to the blob in an upward
compatible manner, so that older versions of SuiteSparse:GraphBLAS can read the blobs
created by v9.0.0; they simply ignore those components.

SuiteSparse:GraphBLAS v10 adds
32/64-bit integers, and can read the blobs created by any prior version of
GraphBLAS (they are deserialized with all 64-bit integers however).  If an older
version of SuiteSparse:GraphBLAS (v9 or earlier) attempts to deserialize a blob
containing a matrix with 32-bit integers, it will safely report that the blob
is invalid and refuse to deserialize it.  If SuiteSparse:GraphBLAS v10 creates a
serialized blob with all-64-bit integers, then it can be read correctly by
SuiteSparse:GraphBLAS v9, and likely also by earlier versions of the library.

There are two forms of serialization: \verb'GrB*serialize' and
\verb'GxB*serialize'.  For the \verb'GrB' form, the blob must first be
allocated by the user application, and it must be large enough to hold the
serialized matrix or vector.  By contrast \verb'GxB*serialize' allocates
the blob itself.

By default, ZSTD (level 1) compression is used for serialization, but other
options can be selected via the descriptor:
\verb'GrB_set (desc, method,' \verb'GxB_COMPRESSION)', where \verb'method' is an
integer selected from the following options:

\vspace{0.2in}
{\footnotesize
\begin{tabular}{ll}
\hline
method                           &  description \\
\hline
\verb'GxB_COMPRESSION_NONE'      &  no compression \\
\verb'GxB_COMPRESSION_DEFAULT'   &  ZSTD, with default level 1 \\
\verb'GxB_COMPRESSION_LZ4'       &  LZ4 \\
\verb'GxB_COMPRESSION_LZ4HC'     &  LZ4HC, with default level 9 \\
\verb'GxB_COMPRESSION_ZSTD'      &  ZSTD, with default level 1 \\
\hline
\end{tabular} }
\vspace{0.2in}

The LZ4HC method can be modified by adding a level of zero to 9, with 9 being
the default.  Higher levels lead to a more compact blob, at the cost of extra
computational time. This level is simply added to the method, so to compress a
vector with LZ4HC with level 6, use:

    {\footnotesize
    \begin{verbatim}
    GrB_set (desc, GxB_COMPRESSION_LZ4HC + 6, GxB_COMPRESSION) ; \end{verbatim}}

The ZSTD method can be specified as level 1 to 19, with 1 being the default.
To compress with ZSTD at level 6, use:

    {\footnotesize
    \begin{verbatim}
    GrB_set (desc, GxB_COMPRESSION_ZSTD + 6, GxB_COMPRESSION) ; \end{verbatim}}

Deserialization of untrusted data is a common security problem; see
\url{https://cwe.mitre.org/data/definitions/502.html}. The deserialization
methods in SuiteSparse:GraphBLAS do a few basic checks so that no out-of-bounds
access occurs during deserialization, but the output matrix or vector itself
may still be corrupted.  If the data is untrusted, use \verb'GxB_*_fprint' with
the print level set to \verb'GxB_SILENT' to
check the matrix or vector after deserializing it:

{\footnotesize
\begin{verbatim}
    info = GxB_Vector_fprint (w, "w deserialized", GxB_SILENT, NULL) ;
    if (info != GrB_SUCCESS) GrB_free (&w) ;
    info = GxB_Matrix_fprint (A, "A deserialized", GxB_SILENT, NULL) ;
    if (info != GrB_SUCCESS) GrB_free (&A) ; \end{verbatim}}

The following methods are described in this Section:

\vspace{0.2in}
\noindent
{\footnotesize
\begin{tabular}{lll}
\hline
GraphBLAS function   & purpose                                      & Section \\
\hline
% \verb'GrB_Vector_serializeSize'  & return size of serialized vector & \ref{vector_serialize_size} \\
% \verb'GrB_Vector_serialize'      & serialize a vector               & \ref{vector_serialize} \\
\verb'GxB_Vector_serialize'      & serialize a vector               & \ref{vector_serialize_GxB} \\
% \verb'GrB_Vector_deserialize'    & deserialize a vector             & \ref{vector_deserialize} \\
\verb'GxB_Vector_deserialize'    & deserialize a vector             & \ref{vector_deserialize_GxB} \\
\hline
\verb'GrB_Matrix_serializeSize' & return size of serialized matrix & \ref{matrix_serialize_size} \\
\verb'GrB_Matrix_serialize'     & serialize a matrix               & \ref{matrix_serialize} \\
\verb'GxB_Matrix_serialize'     & serialize a matrix               & \ref{matrix_serialize_GxB} \\
\verb'GrB_Matrix_deserialize'   & deserialize a matrix             & \ref{matrix_deserialize} \\
\verb'GxB_Matrix_deserialize'   & deserialize a matrix             & \ref{matrix_deserialize_GxB} \\
\hline
\verb'GrB_get' & get blob properties & \ref{get_set_blob} \\
\hline
\end{tabular}
}

%-------------------------------------------------------------------------------
% \subsubsection{{\sf GrB\_Vector\_serializeSize:}  return size of serialized vector}
%-------------------------------------------------------------------------------
% \label{vector_serialize_size}

% \begin{mdframed}[userdefinedwidth=6in]
% {\footnotesize
% \begin{verbatim}
% GrB_Info GrB_Vector_serializeSize   // estimate the size of a blob
% (
%    // output:
%    GrB_Index *blob_size_handle,    // upper bound on the required size of the
%                                    // blob on output.
%    // input:
%    GrB_Vector u                    // vector to serialize
%) ;
%\end{verbatim}
%} \end{mdframed}
%
% \verb'GrB_Vector_serializeSize' returns an upper bound on the size of the blob
% needed to serialize a \verb'GrB_Vector' using \verb'GrB_Vector_serialize'.
% After the vector is serialized, the actual size used is returned, and the blob
% may be \verb'realloc''d to that size if desired.
% This method is not required for \verb'GxB_Vector_serialize'.

% \newpage
%-------------------------------------------------------------------------------
% \subsubsection{{\sf GrB\_Vector\_serialize:}      serialize a vector}
%-------------------------------------------------------------------------------
% \label{vector_serialize}

% \begin{mdframed}[userdefinedwidth=6in]
% {\footnotesize
% \begin{verbatim}
% GrB_Info GrB_Vector_serialize       // serialize a GrB_Vector to a blob
% (
%    // output:
%    void *blob,                     // the blob, already allocated in input
%    // input/output:
%    GrB_Index *blob_size_handle,    // size of the blob on input.  On output,
%                                    // the # of bytes used in the blob.
%    // input:
%    GrB_Vector u                    // vector to serialize
% ) ;
% \end{verbatim}
% } \end{mdframed}
%
% \verb'GrB_Vector_serialize' serializes a vector into a single array of bytes
% (the blob), which must be already allocated by the user application.
% On input, \verb'&blob_size' is the size of the allocated blob in bytes.
% On output, it is reduced to the numbed of bytes actually used to serialize
% the vector.  After calling \verb'GrB_Vector_serialize', the blob may be
% \verb'realloc''d to this revised size if desired (this is optional).
% ZSTD (level 1) compression is used to construct a compact blob.

\newpage
%-------------------------------------------------------------------------------
\subsubsection{{\sf GxB\_Vector\_serialize:}      serialize a vector}
%-------------------------------------------------------------------------------
\label{vector_serialize_GxB}

\begin{mdframed}[userdefinedwidth=6in]
{\footnotesize
\begin{verbatim}
GrB_Info GxB_Vector_serialize       // serialize a GrB_Vector to a blob
(
    // output:
    void **blob_handle,             // the blob, allocated on output
    GrB_Index *blob_size_handle,    // size of the blob on output
    // input:
    GrB_Vector u,                   // vector to serialize
    const GrB_Descriptor desc       // descriptor to select compression method
) ;
\end{verbatim}
} \end{mdframed}

\verb'GxB_Vector_serialize' serializes a vector into a single array of bytes
(the blob), which is \verb'malloc''ed and filled with the serialized vector.
By default, ZSTD (level 1) compression is used, but other options can be
selected via the descriptor.  Serializing a vector is identical to serializing
a matrix; see Section \ref{matrix_serialize_GxB} for more information.

%-------------------------------------------------------------------------------
% \subsubsection{{\sf GrB\_Vector\_deserialize:}    deserialize a vector}
%-------------------------------------------------------------------------------
% \label{vector_deserialize}

% \begin{mdframed}[userdefinedwidth=6in]
% {\footnotesize
% \begin{verbatim}
% GrB_Info GrB_Vector_deserialize     // deserialize blob into a GrB_Vector
% (
%     // output:
%     GrB_Vector *w,      // output vector created from the blob
%     // input:
%     GrB_Type type,      // type of the vector w.  Required if the blob holds a
%                         // vector of user-defined type.  May be NULL if blob
%                         // holds a built-in type; otherwise must match the
%                         // type of w.
%     const void *blob,       // the blob
%     GrB_Index blob_size     // size of the blob
% ) ;
% \end{verbatim}
% } \end{mdframed}
%
% This method creates a vector \verb'w' by deserializing the contents of the
% blob, constructed by either \verb'GrB_Vector_serialize' or
% \verb'GxB_Vector_serialize'.

%-------------------------------------------------------------------------------
\subsubsection{{\sf GxB\_Vector\_deserialize:}    deserialize a vector}
%-------------------------------------------------------------------------------
\label{vector_deserialize_GxB}

\begin{mdframed}[userdefinedwidth=6in]
{\footnotesize
\begin{verbatim}
GrB_Info GxB_Vector_deserialize     // deserialize blob into a GrB_Vector
(
    // output:
    GrB_Vector *w,      // output vector created from the blob
    // input:
    GrB_Type type,      // type of the vector w.  See GxB_Matrix_deserialize.
    const void *blob,       // the blob
    GrB_Index blob_size,    // size of the blob
    const GrB_Descriptor desc
) ;
\end{verbatim}
} \end{mdframed}

This method creates a vector \verb'w' by deserializing the contents of the
blob, constructed by
% either \verb'GrB_Vector_serialize' or
\verb'GxB_Vector_serialize'.
Deserializing a vector is identical to deserializing a matrix;
see Section \ref{matrix_deserialize_GxB} for more information.

The blob is allocated with the \verb'malloc' function passed to
\verb'GxB_init', or the C11 \verb'malloc' if \verb'GrB_init' was used
to initialize GraphBLAS.  The blob must be freed by the matching \verb'free'
method, either the \verb'free' function passed to \verb'GxB_init' or
the C11 \verb'free' if \verb'GrB_init' was used.

\newpage
%-------------------------------------------------------------------------------
\subsubsection{{\sf GrB\_Matrix\_serializeSize:}  return size of serialized matrix}
%-------------------------------------------------------------------------------
\label{matrix_serialize_size}

\begin{mdframed}[userdefinedwidth=6in]
{\footnotesize
\begin{verbatim}
GrB_Info GrB_Matrix_serializeSize   // estimate the size of a blob
(
    // output:
    GrB_Index *blob_size_handle,    // upper bound on the required size of the
                                    // blob on output.
    // input:
    GrB_Matrix A                    // matrix to serialize
) ;
\end{verbatim}
} \end{mdframed}

\verb'GrB_Matrix_serializeSize' returns an upper bound on the size of the blob
needed to serialize a \verb'GrB_Matrix' with \verb'GrB_Matrix_serialize'.
After the matrix is serialized, the actual size used is returned, and the blob
may be \verb'realloc''d to that size if desired.
This method is not required for \verb'GxB_Matrix_serialize'.

%-------------------------------------------------------------------------------
\subsubsection{{\sf GrB\_Matrix\_serialize:}      serialize a matrix}
%-------------------------------------------------------------------------------
\label{matrix_serialize}

\begin{mdframed}[userdefinedwidth=6in]
{\footnotesize
\begin{verbatim}
GrB_Info GrB_Matrix_serialize       // serialize a GrB_Matrix to a blob
(
    // output:
    void *blob,                     // the blob, already allocated in input
    // input/output:
    GrB_Index *blob_size_handle,    // size of the blob on input.  On output,
                                    // the # of bytes used in the blob.
    // input:
    GrB_Matrix A                    // matrix to serialize
) ;
\end{verbatim}
} \end{mdframed}

\verb'GrB_Matrix_serialize' serializes a matrix into a single array of bytes
(the blob), which must be already allocated by the user application.
On input, \verb'&blob_size' is the size of the allocated blob in bytes.
On output, it is reduced to the numbed of bytes actually used to serialize
the matrix.  After calling \verb'GrB_Matrix_serialize', the blob may be
\verb'realloc''d to this revised size if desired (this is optional).
ZSTD (level 1) compression is used to construct a compact blob.

\newpage
%-------------------------------------------------------------------------------
\subsubsection{{\sf GxB\_Matrix\_serialize:}      serialize a matrix}
%-------------------------------------------------------------------------------
\label{matrix_serialize_GxB}

\begin{mdframed}[userdefinedwidth=6in]
{\footnotesize
\begin{verbatim}
GrB_Info GxB_Matrix_serialize       // serialize a GrB_Matrix to a blob
(
    // output:
    void **blob_handle,             // the blob, allocated on output
    GrB_Index *blob_size_handle,    // size of the blob on output
    // input:
    GrB_Matrix A,                   // matrix to serialize
    const GrB_Descriptor desc       // descriptor to select compression method
) ;
\end{verbatim}
} \end{mdframed}

\verb'GxB_Matrix_serialize' is identical to \verb'GrB_Matrix_serialize', except
that it does not require a pre-allocated blob.  Instead, it allocates the blob
internally, and fills it with the serialized matrix.  By default, ZSTD (level 1)
compression is used, but other options can be selected via the descriptor.

The blob is allocated with the \verb'malloc' function passed to
\verb'GxB_init', or the C11 \verb'malloc' if \verb'GrB_init' was used
to initialize GraphBLAS.  The blob must be freed by the matching \verb'free'
method, either the \verb'free' function passed to \verb'GxB_init' or
the C11 \verb'free' if \verb'GrB_init' was used.

%-------------------------------------------------------------------------------
\subsubsection{{\sf GrB\_Matrix\_deserialize:}    deserialize a matrix}
%-------------------------------------------------------------------------------
\label{matrix_deserialize}

\begin{mdframed}[userdefinedwidth=6in]
{\footnotesize
\begin{verbatim}
GrB_Info GrB_Matrix_deserialize     // deserialize blob into a GrB_Matrix
(
    // output:
    GrB_Matrix *C,      // output matrix created from the blob
    // input:
    GrB_Type type,      // type of the matrix C.  Required if the blob holds a
                        // matrix of user-defined type.  May be NULL if blob
                        // holds a built-in type; otherwise must match the
                        // type of C.
    const void *blob,       // the blob
    GrB_Index blob_size     // size of the blob
) ;
\end{verbatim}
} \end{mdframed}

This method creates a matrix \verb'A' by deserializing the contents of the
blob, constructed by either \verb'GrB_Matrix_serialize' or
\verb'GxB_Matrix_serialize'.

% extended in the v2.1 C API (type may be NULL):
The \verb'type' may be \verb'NULL' if the blob holds a serialized matrix with a
built-in type.  In this case, the type is determined automatically.  For
user-defined types, the \verb'type' must match the type of the matrix in the
blob.  The \verb'GrB_get' method can be used to query the blob for the name of
this type.

%-------------------------------------------------------------------------------
\subsubsection{{\sf GxB\_Matrix\_deserialize:}    deserialize a matrix}
%-------------------------------------------------------------------------------
\label{matrix_deserialize_GxB}

\begin{mdframed}[userdefinedwidth=6in]
{\footnotesize
\begin{verbatim}
GrB_Info GxB_Matrix_deserialize     // deserialize blob into a GrB_Matrix
(
    // output:
    GrB_Matrix *C,      // output matrix created from the blob
    // input:
    GrB_Type type,      // type of the matrix C.  Required if the blob holds a
                        // matrix of user-defined type.  May be NULL if blob
                        // holds a built-in type; otherwise must match the
                        // type of C.
    const void *blob,       // the blob
    GrB_Index blob_size,    // size of the blob
    const GrB_Descriptor desc
) ;
\end{verbatim}
} \end{mdframed}

Identical to \verb'GrB_Matrix_deserialize'.


\newpage
%-------------------------------------------------------------------------------
\subsection{The GxB\_Container object and its methods}
%-------------------------------------------------------------------------------
\label{container}

A new set of \verb'load/unload' methods are introduced in GraphBLAS v10 to move
data between a \verb'GrB_Matrix' or \verb'GrB_Vector' and a new
\verb'GxB_Container' object.  This object is non-opaque but contains opaque
objects.  Its primary components are five dense \verb'GrB_Vectors' that hold
the contents of the matrix/vector.  The data in these dense vectors can then be
loaded/unloaded via \verb'GxB_Vector_load' and \verb'GxB_Vector_unload'.

Moving data from a \verb'GrB_Matrix' into user-visible C arrays is a two-step
process.  The data is first moved into a \verb'GxB_Container' using \newline
\verb'GxB_unload_Matrix_into_Container', and then from the Container into C
arrays with \verb'GxB_Vector_unload'.  Moving data in the opposite direction is
also a two-step process: first load the C array into a \verb'GrB_Vector'
component of a \verb'GxB_Container' with \verb'GxB_Vector_load', and then from
the \verb'GxB_Container' into a \verb'GrB_Matrix' using the
\verb'GxB_load_Matrix_from_Container' method.

The following methods are available.  The first two do not use the
Container object, but instead move data to/from a dense \verb'GrB_Vector':

\begin{itemize}

\item \verb'GxB_Vector_load':  this method moves data in O(1) time from a
    user-visible C array into a \verb'GrB_Vector'.  The vector length and type
    are revised to match the new data from the C array.  Ownership is normally
    transferred to the \verb'GrB_Vector', but this can be revised with a
    \verb'handling' parameter.  The C array is passed in as a \verb'void *'
    pointer, and its type is indicated by a \verb'GrB_Type' parameter.  See
    Section~\ref{vector_load} for details.

\item \verb'GxB_Vector_unload': this method moves data in O(1) time from a
    \verb'GrB_Vector' into a user-visible C array.  The length of the
    \verb'GrB_Vector' is reduced to zero, to denote that it no longer holds any
    content.  The vector must be dense; it must have the same number of entries
    as its size (that is \verb'GrB_Vector_nvals' and \verb'GrB_Vector_size'
    must return the same value).  The C array is returned as a \verb'void *'
    pointer, and its type is indicated by a \verb'GrB_Type' parameter.  See
    Section~\ref{vector_unload} for details.

\end{itemize}

The next six methods rely on the \verb'GxB_Container' object:

\begin{itemize}
\item \verb'GxB_Container_new': creates a container
    (see Section~\ref{container_new}).

\item \verb'GxB_Container_free': frees a container
    (see Section~\ref{container_free}).

\item \verb'GxB_load_Matrix_from_Container': moves all of the data from a
    \verb'GxB_Container' into a \verb'GrB_Matrix' in O(1) time
    (see Section~\ref{load_matrix_from_container}).

\item \verb'GxB_load_Vector_from_Container': moves all of the data from a
    \verb'GxB_Container' into a \verb'GrB_Vector' in O(1) time
    (see Section~\ref{load_vector_from_container}).

\item \verb'GxB_unload_Matrix_into_Container': moves all of the data from
    a \verb'GrB_Matrix' into a \verb'GxB_Container' in O(1) time
    (see Section~\ref{unload_matrix_into_container}).

\item \verb'GxB_unload_Vector_into_Container': moves all of the data from
    a \verb'GrB_Vector' into a \verb'GxB_Container' in O(1) time
    (see Section~\ref{unload_vector_into_container}).

\end{itemize}

%-------------------------------------------------------------------------------
\subsubsection{{\sf GxB\_Vector\_load:} load data into a vector}
%-------------------------------------------------------------------------------
\label{vector_load}

\begin{mdframed}[userdefinedwidth=6in]
{\footnotesize
\begin{verbatim}
GrB_Info GxB_Vector_load
(
    // input/output:
    GrB_Vector V,           // vector to load from the C array X
    void **X,               // numerical array to load into V
    // input:
    GrB_Type type,          // type of X
    uint64_t n,             // # of entries in X
    uint64_t X_size,        // size of X in bytes (at least n*(sizeof the type))
    int handling,           // GrB_DEFAULT (0): transfer ownership to GraphBLAS
                            // GxB_IS_READONLY: X treated as read-only;
                            //      ownership kept by the user application
    const GrB_Descriptor desc   // currently unused; for future expansion
) ;
\end{verbatim}
} \end{mdframed}

\verb'GxB_Vector_load' loads data from a C array into a \verb'GrB_Vector' in
O(1) time.

On input, the \verb'GrB_Vector V' must already exist, but its content (type,
size, and entries) are ignored.  On output, any prior content of \verb'V' is
freed, and its data is replaced with the C array \verb'X' of length \verb'n'
entries, whose type is given by the \verb'type' parameter.  The size of
\verb'V' becomes \verb'n', and its type is changed to match the \verb'type'
parameter.

After this method returns, \verb'GrB_Vector_size' and \verb'GrB_Vector_nvals' 
will both return \verb'n'.  That is, the vector \verb'V' is a dense vector.
It is held in the \verb'GxB_FULL' data format, in \verb'GrB_COLMAJOR'
orientation.

The size in bytes of \verb'X' is also provided on input as \verb'X_size'; this
must be at least \verb'n' times the size of the given \verb'type', but it can
be larger.  This size is maintained and returned to the user application by
\verb'GxB_Vector_unload'.  The error code \verb'GrB_INVALID_VALUE' is returned
if \verb'X_size' is too small.

The \verb'handling' parameter provides control over the ownership of the array
\verb'X'.  By default, ownership of \verb'X' is handed over to the
\verb'GrB_Vector V'.  In this case, the vector \verb'V' acts as a normal
GraphBLAS vector.  It can be modified or freed as usual.  Freeing \verb'V' with
\verb'GrB_Vector_free' will also free \verb'X'.  The array \verb'X' is returned
as \verb'NULL' to denote this change of ownership.

If the \verb'handling' parameter is \verb'GxB_IS_READONLY', then \verb'X' is
still owned by the user application.  It remains non-\verb'NULL' when this
method returns.  The resulting vector \verb'V' can be used as an input to any
GraphBLAS method, but it cannot be modified (except that it can be freed).
If a call is made that attempts to modify a matrix with read-only components,
an error is returned (\verb'GxB_OUTPUT_IS_READONLY').
Freeing the vector \verb'V' does not modify \verb'X', however.  It simply
frees the rest of the object \verb'V'.

Note that this method does not rely on the \verb'GxB_Container' object.
Instead, it loads a C array \verb'X' directly into a dense \verb'GrB_Vector V'.

%-------------------------------------------------------------------------------
\subsubsection{{\sf GxB\_Vector\_unload:} unload data from a vector}
%-------------------------------------------------------------------------------
\label{vector_unload}

\begin{mdframed}[userdefinedwidth=6in]
{\footnotesize
\begin{verbatim}
GrB_Info GxB_Vector_unload
(
    // input/output:
    GrB_Vector V,           // vector to unload
    void **X,               // numerical array to unload from V
    // output:
    GrB_Type *type,         // type of X
    uint64_t *n,            // # of entries in X
    uint64_t *X_size,       // size of X in bytes (at least n*(sizeof the type))
    int *handling,          // see GxB_Vector_load
    const GrB_Descriptor desc   // currently unused; for future expansion
) ;
\end{verbatim}
} \end{mdframed}

\verb'GxB_Vector_unload' unloads data from \verb'GrB_Vector' into a C array in
O(1) time (unless the vector has pending work that must be finished first).

On input, the vector \verb'V' must have all possible entries present (that is,
\verb'GrB_Vector_nvals' and \verb'GrB_Vector_size' must return the same value).
The vector can be in any internal data format; it does not have to be in the
\verb'GxB_FULL' format on input, but it must be in \verb'GrB_COLMAJOR'
orientation.  If any entries are missing, the unload is not performed and
\verb'GrB_INVALID_OBJECT' is returned.

On output, the size of \verb'V' is reduced to zero, and it holds no entries but
its type is unchanged.  The array \verb'X' is returned to the caller with the
entries from the vector.  The type of \verb'X' is given by the \verb'type'
parameter.  The number of entries in \verb'V' is returned as \verb'n'.
The size of \verb'X' in bytes is returned as \verb'X_size'.

The \verb'handling' parameter informs the user application of the ownership of
the array \verb'X'.  If it was created by GraphBLAS, or passed into GraphBLAS
via \verb'GxB_Vector_load' with \verb'handling' set to \verb'GrB_DEFAULT', then
this is returned to the user as handling set to \verb'GrB_DEFAULT'.  This
denotes that ownership of the array \verb'X' has been transfered from GraphBLAS
to the user application.  The user application is now responsible for freeing
the array \verb'X'.

If the \verb'handling' parameter is returned as \verb'GxB_IS_READONLY', then
this array \verb'X' was originally passed to GraphBLAS via
\verb'GxB_Vector_load' with a \verb'handling' parameter of
\verb'GxB_IS_READONLY'.  The ownership of the array \verb'X' is not changed; it
remains owned by the user application.  The user application is still
responsible for freeing the array \verb'X', but caution must be observed so
that it is not freed twice.

Note that this method does not rely on the \verb'GxB_Container' object.
Instead, it unloads a dense \verb'GrB_Vector' \verb'V' directly into a
C array \verb'X'.

%-------------------------------------------------------------------------------
\subsubsection{{\sf GxB\_Container\_new:} create a container}
%-------------------------------------------------------------------------------
\label{container_new}

\begin{mdframed}[userdefinedwidth=6in]
{\footnotesize
\begin{verbatim}
GrB_Info GxB_Container_new (GxB_Container *Container) ;
\end{verbatim}
} \end{mdframed}

\verb'GxB_Container_new' creates a new container.  It has the following
definition (extra space for future expansion excluded for clarity):

\begin{mdframed}[userdefinedwidth=6in]
{\footnotesize
\begin{verbatim}
struct GxB_Container_struct
{
    uint64_t nrows, ncols ;
    int64_t nrows_nonempty, ncols_nonempty ;
    uint64_t nvals ;
    int32_t format ;      // GxB_HYPERSPARSE, GxB_SPARSE, GxB_BITMAP, or GxB_FULL
    int32_t orientation ; // GrB_ROWMAJOR or GrB_COLMAJOR
    GrB_Vector p, h, b, i, x ;  // dense vector components
    GrB_Matrix Y ;
    bool iso, jumbled ;
} ;
typedef struct GxB_Container_struct *GxB_Container ; \end{verbatim}
} \end{mdframed}

The \verb'Container' holds all of the data from a single GraphBLAS matrix or
vector, with any data type and any data format.  It contains extra space for
future data formats (not shown above).  Each scalar component is described
below:

\begin{itemize}
\item \verb'nrows': the number of rows of a matrix, or the size of a vector.
\item \verb'ncols': the number of columns of a matrix, or 1 for a vector. 
\item \verb'nrows_nonempty':  if known, this value must give the exact number
    of non-empty rows of a matrix or vector (that is, the number of
    rows have at least one entry).  If not known, this value must be
    set to -1.
\item \verb'ncols_nonempty':  if known, this value must give the exact number
    of non-empty columns of a matrix or vector (that is, the number of
    columns have at least one entry).  If not known, this value must be
    set to -1.
\item \verb'nvals': the number of entries in the matrix or vector.
\item \verb'format': this defines the data format of a matrix or vector.
    Currently, GraphBLAS supports four formats, described in
    Section~\ref{formats}, each with two orientations.  A \verb'GrB_Vector'
    cannot be held in \verb'GxB_HYPERSPARSE' format.
\item \verb'orientation':  whether the matrix is held by row or by column.
    This is always \verb'GrB_COLMAJOR' for a container holding data for a
    \verb'GrB_Vector', and for data holding an $n$-by-1 \verb'GrB_Matrix'
    with a single column.
\item \verb'iso': if true, all of the entries in the matrix have the same
    value, and only a single value is stored in the \verb'x' component of
    the Container.
\item \verb'jumbled': if true, the indices in any given row (if in row-major
    orientation) or column (if column-major) may appear out of order.
    Otherwise, they must appear in ascending order.
    A jumbled matrix or vector must not have any read-only components. 
\end{itemize}

The Container holds five dense \verb'GrB_Vector' objects that hold the primary
data for the matrix or vector, and a single \verb'GrB_Matrix' \verb'Y' that
holds an optional optimization structure called the hyper-hash.  These
components are fully described in Section~\ref{formats}.

\newpage
%-------------------------------------------------------------------------------
\subsubsection{{\sf GxB\_Container\_free:} free a container}
%-------------------------------------------------------------------------------
\label{container_free}

\begin{mdframed}[userdefinedwidth=6in]
{\footnotesize
\begin{verbatim}
GrB_Info GrB_free (GxB_Container *Container) ;
\end{verbatim}
} \end{mdframed}

\verb'GxB_Container_free' frees a container.  It also frees all its opaque
components.  Any read-only array inside these opaque objects is not freed.

%-------------------------------------------------------------------------------
\subsubsection{{\sf GxB\_load\_Matrix\_from\_Container:} load a matrix from a container}
%-------------------------------------------------------------------------------
\label{load_matrix_from_container}

\begin{mdframed}[userdefinedwidth=6in]
{\footnotesize
\begin{verbatim}
GrB_Info GxB_load_Matrix_from_Container     // GrB_Matrix <- GxB_Container
(
    GrB_Matrix A,               // matrix to load from the Container.  On input,
                                // A is a matrix of any size or type; on output
                                // any prior size, type, or contents is freed
                                // and overwritten with the Container.
    GxB_Container Container,    // Container with contents to load into A
    const GrB_Descriptor desc   // currently unused
) ;
\end{verbatim}
} \end{mdframed}

\verb'GxB_load_Matrix_from_Container' moves all of the data from a
\verb'GxB_Container' into a \verb'GrB_Matrix' in O(1) time.

%-------------------------------------------------------------------------------
\subsubsection{{\sf GxB\_load\_Vector\_from\_Container:} load a vector from a container}
%-------------------------------------------------------------------------------
\label{load_vector_from_container}

\begin{mdframed}[userdefinedwidth=6in]
{\footnotesize
\begin{verbatim}

GrB_Info GxB_load_Vector_from_Container     // GrB_Vector <- GxB_Container
(
    GrB_Vector V,               // GrB_Vector to load from the Container
    GxB_Container Container,    // Container with contents to load into V
    const GrB_Descriptor desc   // currently unused
) ;
\end{verbatim}
} \end{mdframed}

\verb'GxB_load_Vector_from_Container' moves all of the data from a
\verb'GxB_Container' into a \verb'GrB_Vector' in O(1) time.

\newpage
%-------------------------------------------------------------------------------
\subsubsection{{\sf GxB\_unload\_Matrix\_into\_Container:} unload a matrix into a container}
%-------------------------------------------------------------------------------
\label{unload_matrix_into_container}

\begin{mdframed}[userdefinedwidth=6in]
{\footnotesize
\begin{verbatim}
GrB_Info GxB_unload_Matrix_into_Container   // GrB_Matrix -> GxB_Container
(
    GrB_Matrix A,               // matrix to unload into the Container
    GxB_Container Container,    // Container to hold the contents of A
    const GrB_Descriptor desc   // currently unused
) ;
\end{verbatim}
} \end{mdframed}

\verb'GxB_unload_Matrix_into_Container': moves all of the data from
a \verb'GrB_Matrix' into a \verb'GxB_Container' in O(1) time.

%-------------------------------------------------------------------------------
\subsubsection{{\sf GxB\_unload\_Vector\_into\_Container:} unload a vector into a container}
%-------------------------------------------------------------------------------
\label{unload_vector_into_container}

\begin{mdframed}[userdefinedwidth=6in]
{\footnotesize
\begin{verbatim}
GrB_Info GxB_unload_Vector_into_Container   // GrB_Vector -> GxB_Container
(
    GrB_Vector V,               // vector to unload into the Container
    GxB_Container Container,    // Container to hold the contents of V
    const GrB_Descriptor desc   // currently unused
) ;
\end{verbatim}
} \end{mdframed}

\verb'GxB_unload_Vector_into_Container': moves all of the data from
a \verb'GrB_Vector' into a \verb'GxB_Container' in O(1) time.

%-------------------------------------------------------------------------------
\subsubsection{Container example: unloading/loading an entire matrix into C arrays}
%-------------------------------------------------------------------------------
\label{container_example}

The following example unloads a 
\verb'GrB_Matrix A' of size \verb'nrows'-by-\verb'ncols',
with \verb'nvals' entries, of type \verb'xtype'.  The example takes will take O(1) time,
and the only \verb'mallocs' are in \verb'GxB_Container_new' (which can be reused for
an arbitrary number of load/unload cycles), and the only frees are in
\verb'GxB_Container_free'.

Note that getting C arrays from a \verb'GrB_Matrix' is a 2-step process:
First unload the matrix A into a Container, giving \verb'GrB_Vector' objects
\verb'Container->p',
\verb'Container->i', \verb'Container->x', etc, and then unload those dense vectors into
C arrays.  This may seem tedious but it allows everything to be done in
O(1) time and space (often no new malloc'd space), and it allows support for
arbitrary integers for the \verb'p', \verb'h', and \verb'i' components of a matrix.
It also makes for a simple API overall.

{\footnotesize
\begin{verbatim}
GxB_Container_new (&Container) ;    // requires several O(1)-sized mallocs

// no malloc/free will occur below, until GxB_Container_free.

for (as many times as you like)
{

    GxB_unload_Matrix_into_Container (A, Container, desc) ;
    // A is now 0-by-0 with nvals(A)=0.  Its type is unchanged.

    // All of the following is optional; if any item in the Container is not
    // needed by the user, it can be left as-is, and then it will be put
    // back into A at the end.  (This is done for the Container->Y).

    // to extract numerical values from the Container:
    void *x = NULL ;
    uint64_t nvals = 0, nheld = 0 ;
    GrB_Type xtype = NULL ;
    int x_handling, p_handling, h_handling, i_handling, b_handling ;
    uint64_t x_size, p_size, h_size, i_size, b_size ;
    GxB_Vector_unload (Container->x, &x, &xtype, &nheld, &x_size, &x_handling,
        desc) ;

    // The C array x now has size nheld and contains the values of the original
    // GrB_Matrix A, with type xtype being the original type of the matrix A.
    // The Container->x GrB_Vector still exists but it now has length 0.
    // If the matrix A was iso-valued, nheld == 1.

    // to extract the sparsity pattern from the Container:
    GrB_Type ptype = NULL, htype = NULL, itype = NULL, btype = NULL ;
    void *p = NULL, *h = NULL, *i = NULL, *b = NULL ;
    uint64_t plen = 0, plen1 = 0, nheld = 0 ;

    switch (Container->format)
    {
        case GxB_HYPERSPARSE :
            // The Container->Y matrix can be unloaded here as well,
            // if desired.  Its use is optional.
            GxB_Vector_unload (Container->h, &h, &htype, &plen, &h_size,
                &h_handling, desc) ;
        case GxB_SPARSE :
            GxB_Vector_unload (Container->p, &p, &ptype, &plen1, &p_size,
                &p_handling, desc) ;
            GxB_Vector_unload (Container->i, &i, &itype, &nvals, &i_size,
                &i_handling, desc) ;
            break ;
        case GxB_BITMAP :
            GxB_Vector_unload (Container->b, &b, &btype, &nheld, &b_size,
                &b_handling, desc) ;
            break ;
    }

    // Now the C arrays (p, h, i, b, and x) are all populated and owned by the
    // user application.  They can be modified here, if desired.  Their C type
    // is (void *), and their actual types correspond to ptype, htype, itype,
    // btype, and xtype).

    // to load them back into A, first load them into the Container->[phbix]
    // vectors:
    switch (Container->format)
    {
        case GxB_HYPERSPARSE :
            // The Container->Y matrix can be loaded here as well,
            // if desired.  Its use is optional.
            GxB_Vector_load (Container->h, &h, htype, plen, h_size,
                h_handling, desc) ;
        case GxB_SPARSE :
            GxB_Vector_load (Container->p, &p, ptype, plen1, p_size,
                p_handling, desc) ;
            GxB_Vector_load (Container->i, &i, itype, nvals, i_size,
                i_handling, desc) ;
            break ;
        case GxB_BITMAP :
            GxB_Vector_load (Container->b, &b, btype, nheld, b_size,
                b_handling, desc) ;
            break ;
    }
    GxB_Vector_load (Container->x, &x, xtype, nheld, x_size,
        x_handling, desc) ;

    // Now the C arrays p, h, i, b, and x are all NULL.  They are in the
    // Container->p,h,b,i,x GrB_Vectors.  Load A from the non-opaque Container:

    GxB_load_Matrix_from_Container (A, Container, desc) ;
    // A is now back to its original state.  The Container and its p,h,b,i,x
    // GrB_Vectors exist but its vectors all have length 0.

}

GxB_Container_free (&Container) ;    // does several O(1)-sized free's
\end{verbatim}}

%-------------------------------------------------------------------------------
\subsubsection{Container example: unloading/loading, but not using C arrays}
%-------------------------------------------------------------------------------
\label{container_example2}

Using the container is very simple if the resulting Container \verb'GrB_Vector'
components are used directly by GraphBLAS, with no need for C arrays.  For
example, in a push/relabel maxflow algorithm, there is a need to extract the
tuples from a \verb'GrB_Vector' \verb'Delta', followed by a call to
\verb'GrB_Matrix_build' to create a matrix from that data.  In GraphBLAS v9 and
earlier, extracting the tuples requires a copy.  In v10, it can be done using
the container, without requiring a copy of the contents of \verb'Delta'.

{\footnotesize
\begin{verbatim}
GxB_Container_new (&Container) ;
for (...)
{
    GrB_Vector Delta, J_Vector ;      // computed by GraphBLAS (not shown)
    GrB_Matrix DeltaMatrix ;
    ...
    GxB_unload_Vector_into_Container (Delta, Container, desc) ;
    GxB_Matrix_build_Vector (DeltaMatrix, Container->i, J_vector,
        Container->x, GrB_PLUS_FP32, NULL) :
    GxB_load_Vector_from_Container (Delta, Container, desc) ;
}
GxB_Container_free (&Container) ;
\end{verbatim}}

The contents of the \verb'Delta' vector can be used unloaded in to the
container for use by \verb'GxB_Matrix_build_Vector', in O(1) time, and then
loaded back afterwards, also in O(1) time.  The construction of the
\verb'DeltaMatrix' takes the same time as \verb'GrB_Matrix_build', but the
extra copy that would be required for \verb'GrB_Vector_extractTuples' is
entirely avoided.


\newpage
%-------------------------------------------------------------------------------
\subsection{SuiteSparse:GraphBLAS data formats}
%-------------------------------------------------------------------------------
\label{formats}

SuiteSparse:GraphBLAS uses four distinct data formats: sparse, hypersparse,
bitmap, and full, each in row-major or column-major orientations, for eight
different variants (each of which are listed below).
Each of these eight total variants can be iso-valued, where if
\verb'Container->iso' is true the numerical values are all the same, and
\verb'Container->x' holds a single entry with this value.
Each of the sparse and hypersparse formats can appear in {\em jumbled} form,
where the indices within any given row (if the orientation is row-major)
or column may be out of order.  If \verb'Container->jumbled' is false, then
the indices appear in ascending order.

The \verb'p', \verb'h', and \verb'i' vectors in the Container have an integer
type, either \verb'GrB_UINT32' or \verb'GrB_UINT64'.  These appear below as
just \verb'integer', but the actual corresponding C type (\verb'uint32_t' or
\verb'uint64_t') must be used for each component.

%-------------------------------------------------------------------------------
\subsubsection{Sparse, held by row}
%-------------------------------------------------------------------------------
\label{format_sparse_by_row}

A sparse matrix in CSR format, held by row, has a \verb'Container->format'
value of \verb'GxB_SPARSE' and a \verb'Container->orientation' of
\verb'GrB_ROWMAJOR'.  It requires three arrays:

\begin{itemize}
\item \verb'integer Ap [nrows+1] ;'  The \verb'Ap' array is the row
``pointer'' array.  It does not actual contain pointers, but integer offsets.
More precisely, it is an integer array that defines where the column indices
and values appear in \verb'Aj' and \verb'Ax', for each row.  The number of
entries in row \verb'i' is given by the expression \verb'Ap [i+1] - Ap [i]'.

\item \verb'integer Aj [nvals] ;'  The \verb'Aj' array defines the
column indices of entries in each row.

\item \verb'type Ax [nvals] ;'  The \verb'Ax' array defines the values of
entries in each row.  
\end{itemize}

The content of the three arrays \verb'Ap' \verb'Aj', and \verb'Ax' is very
specific.  This content is not checked, since this function takes only
$O(1)$ time.  Results are undefined if the following specification is not
followed exactly.

The column indices of entries in the ith row of the matrix are held in
\verb'Aj [Ap [i] ... Ap[i+1]]', and the corresponding values are held in the
same positions in \verb'Ax'.  Column indices must be in the range 0 to
\verb'ncols'-1.  If \verb'jumbled' is \verb'false', column indices must appear
in ascending order within each row.  If \verb'jumbled' is \verb'true', column
indices may appear in any order within each row.  No duplicate column indices
may appear in any row.  \verb'Ap [0]' must equal zero, and \verb'Ap [nrows]'
must equal \verb'nvals'.  The \verb'Ap' array must be of size \verb'nrows'+1
(or larger), and the \verb'Aj' and \verb'Ax' arrays must have size at least
\verb'nvals'.

An example of the CSR format is shown below.  Consider the following
matrix with 10 nonzero entries, and suppose the zeros are not stored.

    \begin{equation}
    \label{eqn:Aexample}
    A = \left[
    \begin{array}{cccc}
    4.5 &   0 & 3.2 &   0 \\
    3.1 & 2.9 &  0  & 0.9 \\
     0  & 1.7 & 3.0 &   0 \\
    3.5 & 0.4 &  0  & 1.0 \\
    \end{array}
    \right]
    \end{equation}

The \verb'Ap' array has length 5, since the matrix is 4-by-4.  The first entry
must always zero, and \verb'Ap [5] = 10' is the number of entries.
The content of the arrays is shown below:

{\footnotesize
\begin{verbatim}
    integer Ap [ ] = { 0,        2,             5,        7,            10 } ;
    integer Aj [ ] = { 0,   2,   0,   1,   3,   1,   2,   0,   1,   3   } ;
    double  Ax [ ] = { 4.5, 3.2, 3.1, 2.9, 0.9, 1.7, 3.0, 3.5, 0.4, 1.0 } ; \end{verbatim} }

Spaces have been added to the \verb'Ap' array, just for illustration.  Row zero
is in \verb'Aj [0..1]' (column indices) and \verb'Ax [0..1]' (values), starting
at \verb'Ap [0] = 0' and ending at \verb'Ap [0+1]-1 = 1'.  The list of column
indices of row one is at \verb'Aj [2..4]' and row two is in \verb'Aj [5..6]'.
The last row (three) appears \verb'Aj [7..9]', because \verb'Ap [3] = 7' and
\verb'Ap [4]-1 = 10-1 = 9'.  The corresponding numerical values appear in the
same positions in \verb'Ax'.

To iterate over the rows and entries of this matrix, the following code can be
used (assuming it has type \verb'GrB_FP64'):

    {\footnotesize
    \begin{verbatim}
    integer nvals = Ap [nrows] ;
    for (integer i = 0 ; i < nrows ; i++)
    {
        // get A(i,:)
        for (integer p = Ap [i] ; p < Ap [i+1] ; p++)
        {
            // get A(i,j)
            integer  j = Aj [p] ;           // column index
            double aij = Ax [iso ? 0 : p] ;   // numerical value
        }
    } \end{verbatim}}

In the container, the three arrays \verb'Ap', \verb'Aj' and \verb'Ax'
are held in three \verb'GrB_Vector' objects:
\verb'Container->p',
\verb'Container->i', and
\verb'Container->x', respectively.

%-------------------------------------------------------------------------------
\subsubsection{Sparse, held by column}
%-------------------------------------------------------------------------------
\label{format_sparse_by_col}

This format is the transpose of sparse-by-row.  A sparse matrix in CSC format,
held by column, has a \verb'Container->format' value of \verb'GxB_SPARSE' and a
\verb'Container->orientation' of \verb'GrB_COLMAJOR'.  It requires three
arrays: \verb'Ap', \verb'Ai', and \verb'Ax'.

The column ``pointer'' array \verb'Ap' has size \verb'ncols+1'.  The row
indices of the columns are in \verb'Ai', and if \verb'jumbled' is false,
they must appear in ascending order in
each column.  The corresponding numerical values are held in \verb'Ax'.  The
row indices of column \verb'j' are held in \verb'Ai [Ap [j]...Ap [j+1]-1]',
and the corresponding numerical values are in the same locations in \verb'Ax'.

The same matrix from Equation~\ref{eqn:Aexample} in
the last section (repeated here):

    \begin{equation}
    A = \left[
    \begin{array}{cccc}
    4.5 &   0 & 3.2 &   0 \\
    3.1 & 2.9 &  0  & 0.9 \\
     0  & 1.7 & 3.0 &   0 \\
    3.5 & 0.4 &  0  & 1.0 \\
    \end{array}
    \right]
    \end{equation}

is held in CSC form as follows:

{\footnotesize
\begin{verbatim}
    integer Ap [ ] = { 0,             3,             6,        8,       10 } ;
    integer Ai [ ] = { 0,   1,   3,   1,   2,   3,   0,   2,   1,   3   } ;
    double  Ax [ ] = { 4.5, 3.1, 3.5, 2.9, 1.7, 0.4, 3.2, 3.0, 0.9, 1.0 } ; \end{verbatim} }

That is, the row indices of column 1 (the second column) are in
\verb'Ai [3..5]', and the values in the same place in \verb'Ax',
since \verb'Ap [1] = 3' and \verb'Ap [2]-1 = 5'.

To iterate over the columns and entries of this matrix, the following code can
be used (assuming it has type \verb'GrB_FP64'):

    {\footnotesize
    \begin{verbatim}
    integer nvals = Ap [ncols] ;
    for (integer j = 0 ; j < ncols ; j++)
    {
        // get A(:,j)
        for (integer p = Ap [j] ; p < Ap [j+1] ; p++)
        {
            // get A(i,j)
            integer  i = Ai [p] ;             // row index
            double aij = Ax [iso ? 0 : p] ;   // numerical value
        }
    } \end{verbatim}}

In the container, the three arrays \verb'Ap', \verb'Ai' and \verb'Ax'
are held in three \verb'GrB_Vector' objects:
\verb'Container->p',
\verb'Container->i', and
\verb'Container->x', respectively.

%-------------------------------------------------------------------------------
\subsubsection{Hypersparse, held by row}
%-------------------------------------------------------------------------------
\label{format_hypersparse_by_row}

The hypersparse HyperCSR format is identical to the CSR format, except that the
\verb'Ap' array itself becomes sparse, if the matrix has rows that are
completely empty.  An array \verb'Ah' of size \verb'nvec' provides a list of
rows that appear in the data structure.  For example, consider
Equation~\ref{eqn:Ahyper}, which is a sparser version of the matrix in
Equation~\ref{eqn:Aexample}.  Row 2 and column 1 of this matrix are all empty.

    \begin{equation}
    \label{eqn:Ahyper}
    A = \left[
    \begin{array}{cccc}
    4.5 &   0 & 3.2 &   0 \\
    3.1 &   0 &  0  & 0.9 \\
     0  &   0 &  0  &   0 \\
    3.5 &   0 &  0  & 1.0 \\
    \end{array}
    \right]
    \end{equation}

The conventional CSR format would appear as follows.  Since the third row (row
2) is all zero, accessing \verb'Ai [Ap [2] ... Ap [3]-1]' gives an empty set
(\verb'[2..1]'), and the number of entries in this row is
\verb'Ap [i+1] - Ap [i]' \verb'= Ap [3] - Ap [2] = 0'.

{\footnotesize
\begin{verbatim}
    integer Ap [ ] = { 0,        2,2,      4,       5 } ;
    integer Aj [ ] = { 0,   2,   0,   3,   0    3   }
    double  Ax [ ] = { 4.5, 3.2, 3.1, 0.9, 3.5, 1.0 } ; \end{verbatim} }

A hypersparse CSR format for this same matrix would discard
these duplicate integers in \verb'Ap'.  Doing so requires
another array, \verb'Ah', that keeps track of the rows that appear
in the data structure.

{\footnotesize
\begin{verbatim}
    integer nvec = 3 ;
    integer Ah [ ] = { 0,        1,        3        } ;
    integer Ap [ ] = { 0,        2,        4,       5 } ;
    integer Aj [ ] = { 0,   2,   0,   3,   0    3   }
    double  Ax [ ] = { 4.5, 3.2, 3.1, 0.9, 3.5, 1.0 } ; \end{verbatim} }

Note that the \verb'Aj' and \verb'Ax' arrays are the same in the CSR and
HyperCSR formats.  If \verb'jumbled' is false, the row indices in \verb'Ah'
must appear in ascending order, and no duplicates can appear.  To iterate over
this data structure (assuming it has type \verb'GrB_FP64'):

    {\footnotesize
    \begin{verbatim}
    integer nvals = Ap [nvec] ;
    for (integer k = 0 ; k < nvec ; k++)
    {
        integer i = Ah [k] ;                // row index
        // get A(i,:)
        for (integer p = Ap [k] ; p < Ap [k+1] ; p++)
        {
            // get A(i,j)
            integer  j = Aj [p] ;             // column index
            double aij = Ax [iso ? 0 : p] ;   // numerical value
        }
    } \end{verbatim}}

\vspace{-0.05in}
This is more complex than the CSR format, but it requires at most
$O(e)$ space, where $A$ is $m$-by-$n$ with $e$ = \verb'nvals' entries.  The
CSR format requires $O(m+e)$ space.  If $e << m$, then the size $m+1$
of \verb'Ap' can dominate the memory required.  In the hypersparse form,
\verb'Ap' takes on size \verb'nvec+1', and \verb'Ah' has size \verb'nvec',
where \verb'nvec' is the number of rows that appear in the data structure.
The CSR format can be viewed as a dense array (of size \verb'nrows')
of sparse row vectors.   By contrast, the hypersparse CSR format is a sparse
array (of size \verb'nvec') of sparse row vectors.

In the container, the four arrays \verb'Ap', \verb'Ah', \verb'Aj' and \verb'Ax'
are held in four \verb'GrB_Vector' objects:
\verb'Container->p',
\verb'Container->h',
\verb'Container->i', and \newline
\verb'Container->x', respectively.

In addition, the container may hold an optional optimization structure,
\verb'Container->Y', called the hyper-hash.  This is a \verb'GrB_Matrix' that
holds the inverse of the \verb'Container->h' array (called \verb'Y' because it
looks like an upside-down \verb'h').  If a matrix is being loaded from raw
data, the hyper-hash is not yet constructed, so the \verb'Container->Y' matrix
should be set to \verb'NULL'.  GraphBLAS will compute it when needed.

When a matrix is unload into a container, GraphBLAS will place the hyper-hash
matrix there if it has been computed.  If the matrix is subsequently loaded
from the container, and \verb'Container->h' is unchanged, then leaving the
hyper-hash unmodified will preserve this optional optimization data structure.
If instead \verb'Container->h' is revised, the hyper-hash in
\verb'Container->Y' must be freed (or at least removed from the container) when
the matrix is loaded from the container..

A \verb'GrB_Vector' is never held in hypersparse format.

%-------------------------------------------------------------------------------
\subsubsection{Hypersparse, held by column}
%-------------------------------------------------------------------------------
\label{format_hypersparse_by_col}

The hypersparse-by-column format is the transpose of the hypersparse-by-row format.
The \verb'Container->format' is \verb'GxB_HYPERSPARSE' and the \newline
\verb'Container->orientation' is \verb'GrB_COLMAJOR'.
In the container, the four arrays \verb'Ap', \verb'Ah', \verb'Ai' and \verb'Ax'
are held in four \verb'GrB_Vector' objects:
\verb'Container->p',
\verb'Container->h',
\verb'Container->i', and
\verb'Container->x', respectively.
A \verb'GrB_Vector' is never held in hypersparse format.

%-------------------------------------------------------------------------------
\subsubsection{Bitmap, held by row}
%-------------------------------------------------------------------------------
\label{format_bitmap_by_row}

The \verb'Container->format' is \verb'GxB_BITMAP' and the
\verb'Container->orientation' is \verb'GrB_ROWMAJOR'.
This format requires two arrays, \verb'Ab' and \verb'Ax', each of which are
size \verb'nrows*ncols'.  They correspond to \verb'Container->b' and
\verb'Container->x' in the \verb'GxB_Container' object.  These arrays define
the pattern and values of the matrix \verb'A':

\begin{itemize}
\item \verb'int8_t Ab [nrows*ncols] ;'  The \verb'Ab' array defines which
entries of \verb'A' are present.  If \verb'Ab[i*ncols+j]=1', then the entry
$A(i,j)$ is present, with value \verb'Ax[i*ncols+j]'.  If
\verb'Ab[i*ncols+j]=0', then the entry $A(i,j)$ is not present.  The \verb'Ab'
array must contain only 0s and 1s.  The \verb'nvals' input must exactly match
the number of 1s in the \verb'Ab' array.
\item \verb'type Ax [nrows*ncols] ;'  The \verb'Ax' array defines the values of
entries in the matrix.  If \verb'Ab[p]' is zero, the value of \verb'Ax[p]' is
ignored.  If the matrix is iso-valued, \verb'Ax' has size 1.
\end{itemize}

%-------------------------------------------------------------------------------
\subsubsection{Bitmap, held by column}
%-------------------------------------------------------------------------------
\label{format_bitmap_by_col}

This is the transpose of the bitmap-by-row format.
The \verb'Container->format' is \verb'GxB_BITMAP' and the
\verb'Container->orientation' is \verb'GrB_COLMAJOR'.
The value of the entry $A(i,j)$ is held in \verb'Ax [i+j*nrows]', or
in \verb'Ax[0]' if the matrix is iso-valued.
It is present if \verb'Ab [i+j*nrows]' is 1, and not present if zero.

%-------------------------------------------------------------------------------
\subsubsection{Full, held by row}
%-------------------------------------------------------------------------------
\label{format_full_by_row}

The \verb'Container->format' is \verb'GxB_FULL' and the
\verb'Container->orientation' is \verb'GrB_ROWMAJOR'.  This format is held in a
single \verb'GrB_Vector', \verb'Container->x'.  The $A(i,j)$ entry is in
position \verb'i*ncols+j' in this array, or in position 0 if the matrix is
iso-valued.  All entries are present.

%-------------------------------------------------------------------------------
\subsubsection{Full, held by column}
%-------------------------------------------------------------------------------
\label{format_full_by_col}

This is the transpose of the full-by-row format.
The \verb'Container->format' is \verb'GxB_FULL' and the
\verb'Container->orientation' is \verb'GrB_COLMAJOR'.  This format is held in a
single \verb'GrB_Vector', \verb'Container->x'.  The $A(i,j)$ entry is in
position \verb'i+j*nrows' in this array, or in position 0 if the matrix is
iso-valued.  All entries are present.



\newpage
%===============================================================================
\subsection{GraphBLAS import/export: using copy semantics} %====================
%===============================================================================
\label{GrB_import_export}

The v2.0 C API includes import/export methods for matrices (not vectors) using
a different strategy as compared to the \verb'GxB_Container' methods.  The
\verb'GxB_Container' methods are based on {\em move semantics}, in which
ownership of arrays is passed between SuiteSparse:GraphBLAS and the user
application.  This allows the \verb'GxB_Container' methods to work in $O(1)$
time, and require no additional memory, but it requires that GraphBLAS and the
user application agree on which memory manager to use.  This is done via
\verb'GxB_init'.  This allows GraphBLAS to \verb'malloc' an array that can be
later \verb'free'd by the user application, and visa versa.

The \verb'GrB' import/export methods take a different approach.  The data
is always copied in and out between the opaque GraphBLAS matrix and the
user arrays.  This takes $\Omega(e)$ time, if the matrix has $e$ entries,
and requires more memory.  It has the advantage that it does not require
GraphBLAS and the user application to agree on what memory manager to use,
since no ownership of allocated arrays is changed.

The format for \verb'GrB_Matrix_import' and \verb'GrB_Matrix_export' is
controlled by the following enum:

{\footnotesize
\begin{verbatim}
typedef enum
{
    GrB_CSR_FORMAT = 0,     // CSR format (equiv to GxB_SPARSE with GrB_ROWMAJOR)
    GrB_CSC_FORMAT = 1,     // CSC format (equiv to GxB_SPARSE with GrB_COLMAJOR)
    GrB_COO_FORMAT = 2      // triplet format (like input to GrB*build)
}
GrB_Format ; \end{verbatim}}

\newpage
%-------------------------------------------------------------------------------
\subsubsection{{\sf GrB\_Matrix\_import:}  import a matrix}
%-------------------------------------------------------------------------------
\label{GrB_matrix_import}

\begin{mdframed}[userdefinedwidth=6in]
{\footnotesize
\begin{verbatim}
GrB_Info GrB_Matrix_import  // import a matrix
(
    GrB_Matrix *A,          // handle of matrix to create
    GrB_Type type,          // type of matrix to create
    GrB_Index nrows,        // number of rows of the matrix
    GrB_Index ncols,        // number of columns of the matrix
    const GrB_Index *Ap,    // pointers for CSR, CSC, column indices for COO
    const GrB_Index *Ai,    // row indices for CSR, CSC
    const <type> *Ax,       // values
    GrB_Index Ap_len,       // number of entries in Ap (not # of bytes)
    GrB_Index Ai_len,       // number of entries in Ai (not # of bytes)
    GrB_Index Ax_len,       // number of entries in Ax (not # of bytes)
    int format              // import format (GrB_Format)
) ;
\end{verbatim}
} \end{mdframed}

The \verb'GrB_Matrix_import' method copies from user-provided arrays into an
opaque \verb'GrB_Matrix' and \verb'GrB_Matrix_export' copies data out, from an
opaque \verb'GrB_Matrix' into user-provided arrays.

The suffix \verb'TYPE' in the prototype above is one of \verb'BOOL',
\verb'INT8', \verb'INT16', etc, for built-n types, or \verb'UDT' for
user-defined types.  The type of the \verb'Ax' array must match this type.  No
typecasting is performed.

Unlike the \verb'GxB_Container' methods, memory is not handed off between the
user application and GraphBLAS.   The three arrays \verb'Ap', \verb'Ai'.  and
\verb'Ax' are not modified, and are still owned by the user application when
the method finishes.

\verb'GrB_Matrix_import' does not support the creation of matrices with 32-bit
integer indices.

The matrix can be imported in one of three different formats:

\begin{packed_itemize}
    \item \verb'GrB_CSR_FORMAT': % CSR format (equiv to GxB_SPARSE with GrB_ROWMAJOR)
        Compressed-row format.  \verb'Ap' is an array of size \verb'nrows+1'.
        The arrays \verb'Ai' and \verb'Ax' are of size \verb'nvals = Ap [nrows]',
        and \verb'Ap[0]' must be zero.
        The column indices of entries in the \verb'i'th row appear in
        \verb'Ai[Ap[i]...Ap[i+1]-1]', and the values of those entries appear in
        the same locations in \verb'Ax'.
        The column indices need not be in any particular order.
        See Section~\ref{format_sparse_by_row} for details of the sparse-by-row (CSR) format.

    \item \verb'GrB_CSC_FORMAT': % CSC format (equiv to GxB_SPARSE with GrB_COLMAJOR)
        Compressed-column format.  \verb'Ap' is an array of size \verb'ncols+1'.
        The arrays \verb'Ai' and \verb'Ax' are of size \verb'nvals = Ap [ncols]',
        and \verb'Ap[0]' must be zero.
        The row indices of entries in the \verb'j'th column appear in
        \verb'Ai[Ap[j]...Ap[j+1]-1]', and the values of those entries appear in
        the same locations in \verb'Ax'.
        The row indices need not be in any particular order.
        See Section~\ref{format_sparse_by_col} for details of the sparse-by-column (CSC) format.

    \item \verb'GrB_COO_FORMAT': % triplet format (like input to GrB*build)
        Coordinate format.  This is the same format as the \verb'I', \verb'J',
        \verb'X' inputs to \verb'GrB_Matrix_build'.  The three arrays
        \verb'Ap', \verb'Ai', and \verb'Ax' have the same size.  The \verb'k'th
        tuple has row index \verb'Ai[k]', column index \verb'Ap[k]', and value
        \verb'Ax[k]'.  The tuples can appear any order, but no duplicates are
        permitted.

%   \item \verb'GrB_DENSE_ROW_FORMAT': % FullR format (GxB_FULL with GrB_ROWMAJOR)
%       Dense matrix format, held by row.  Only the \verb'Ax' array is used, of
%       size \verb'nrows*ncols'.
%       It holds the matrix in dense format, in row major order.
%
%   \item \verb'GrB_DENSE_COL_FORMAT': % FullC format (GxB_FULL with GrB_ROWMAJOR)
%       Dense matrix format, held by column.  Only the \verb'Ax' array is used, of
%       size \verb'nrows*ncols'.
%       It holds the matrix in dense format, in column major order.

\end{packed_itemize}

%-------------------------------------------------------------------------------
\subsubsection{{\sf GrB\_Matrix\_export:}  export a matrix}
%-------------------------------------------------------------------------------
\label{GrB_matrix_export}

\begin{mdframed}[userdefinedwidth=6in]
{\footnotesize
\begin{verbatim}
GrB_Info GrB_Matrix_export  // export a matrix
(
    GrB_Index *Ap,          // pointers for CSR, CSC, column indices for COO
    GrB_Index *Ai,          // col indices for CSR/COO, row indices for CSC
    <type> *Ax,             // values (must match the type of A_input)
    GrB_Index *Ap_len,      // number of entries in Ap (not # of bytes)
    GrB_Index *Ai_len,      // number of entries in Ai (not # of bytes)
    GrB_Index *Ax_len,      // number of entries in Ax (not # of bytes)
    int format,             // export format (GrB_Format)
    GrB_Matrix A            // matrix to export
) ;
\end{verbatim}
} \end{mdframed}

\verb'GrB_Matrix_export' copies the contents of a matrix into three
user-provided arrays, using any one of the three different formats
described in Section~\ref{GrB_matrix_import}.  The size of the arrays must be
at least as large as the lengths returned by \verb'GrB_Matrix_exportSize'.  The
matrix \verb'A' is not modified.

On input, the size of the three arrays \verb'Ap', \verb'Ai', and \verb'Ax' is
given by \verb'Ap_len', \verb'Ai_len', and \verb'Ax_len', respectively.  These
values are in terms of the number of entries in these arrays, not the number of
bytes.  On output, these three value are adjusted to report the number of
entries written to the three arrays.

The suffix \verb'TYPE' in the prototype above is one of \verb'BOOL',
\verb'INT8', \verb'INT16', etc, for built-n types, or \verb'UDT' for
user-defined types.  The type of the \verb'Ax' array must match this type.  No
typecasting is performed.

\verb'GrB_Matrix_export' always exports the indices and offsets of the matrix
using 64-bit integer indices, even if they are held internally using 32-bit
integers.

% The \verb'GrB_DENSE_ROW_FORMAT' and \verb'GrB_DENSE_COL_FORMAT' formats can
% only be used if all entries are present in the matrix.  That is,
% \verb'GrB_Matrix_nvals (&nvals,A)' must return \verb'nvals' equal to
% \verb'nrows*ncols'.

\newpage
%-------------------------------------------------------------------------------
\subsubsection{{\sf GrB\_Matrix\_exportSize:} determine size of export}
%-------------------------------------------------------------------------------
\label{export_size}

\begin{mdframed}[userdefinedwidth=6in]
{\footnotesize
\begin{verbatim}
GrB_Info GrB_Matrix_exportSize  // determine sizes of user arrays for export
(
    GrB_Index *Ap_len,      // # of entries required for Ap (not # of bytes)
    GrB_Index *Ai_len,      // # of entries required for Ai (not # of bytes)
    GrB_Index *Ax_len,      // # of entries required for Ax (not # of bytes)
    int format,             // export format (GrB_Format)
    GrB_Matrix A            // matrix to export
) ;
\end{verbatim}
} \end{mdframed}

Returns the required sizes of the arrays \verb'Ap', \verb'Ai', and \verb'Ax'
for exporting a matrix using \verb'GrB_Matrix_export', using the same
\verb'format'.

%-------------------------------------------------------------------------------
\subsubsection{{\sf GrB\_Matrix\_exportHint:} determine best export format}
%-------------------------------------------------------------------------------
\label{export_hint}

\begin{mdframed}[userdefinedwidth=6in]
{\footnotesize
\begin{verbatim}
GrB_Info GrB_Matrix_exportHint  // suggest the best export format
(
    int *format,            // export format (GrB_Format)
    GrB_Matrix A            // matrix to export
) ;
\end{verbatim}
} \end{mdframed}

This method suggests the most efficient format for the export of a given
matrix.  For SuiteSparse:GraphBLAS, the hint depends on the current
format of the \verb'GrB_Matrix':

\begin{packed_itemize}
\item \verb'GxB_SPARSE', \verb'GrB_ROWMAJOR': export as \verb'GrB_CSR_FORMAT'
\item \verb'GxB_SPARSE', \verb'GrB_COLMAJOR': export as \verb'GrB_CSC_FORMAT'
\item \verb'GxB_HYPERSPARSE': export as \verb'GrB_COO_FORMAT'
\item \verb'GxB_BITMAP', \verb'GrB_ROWMAJOR': export as \verb'GrB_CSR_FORMAT'
\item \verb'GxB_BITMAP', \verb'GrB_COLMAJOR': export as \verb'GrB_CSC_FORMAT'
%\item \verb'GxB_FULL', \verb'GrB_ROWMAJOR': export as \verb'GrB_DENSE_ROW_FORMAT'
%\item \verb'GxB_FULL', \verb'GrB_COLMAJOR': export as \verb'GrB_DENSE_COL_FORMAT'
\item \verb'GxB_FULL', \verb'GrB_ROWMAJOR': export as \verb'GrB_CSR_FORMAT'
\item \verb'GxB_FULL', \verb'GrB_COLMAJOR': export as \verb'GrB_CSC_FORMAT'
\end{packed_itemize}





\newpage
%===============================================================================
\subsection{Sorting methods}
%===============================================================================
\label{sorting_methods}

\verb'GxB_Matrix_sort' provides a mechanism to sort all the rows or
all the columns of a matrix, and \verb'GxB_Vector_sort' sorts all the
entries in a vector.

%-------------------------------------------------------------------------------
\subsubsection{{\sf GxB\_Vector\_sort:} sort a vector}
%-------------------------------------------------------------------------------
\label{vector_sort}

\begin{mdframed}[userdefinedwidth=6in]
{\footnotesize
\begin{verbatim}
GrB_Info GxB_sort
(
    // output:
    GrB_Vector w,           // vector of sorted values
    GrB_Vector p,           // vector containing the permutation
    // input
    GrB_BinaryOp op,        // comparator op
    GrB_Vector u,           // vector to sort
    const GrB_Descriptor desc
) ;
\end{verbatim}
} \end{mdframed}

\verb'GxB_Vector_sort' is identical to sorting the single column of an
\verb'n'-by-1 matrix.
Refer to Section \ref{matrix_sort} for details.
%
The \verb'op' cannot be a binary operator
created by \verb'GxB_BinaryOp_new_IndexOp'.

%-------------------------------------------------------------------------------
\subsubsection{{\sf GxB\_Matrix\_sort:} sort the rows/columns of a matrix}
%-------------------------------------------------------------------------------
\label{matrix_sort}

\begin{mdframed}[userdefinedwidth=6in]
{\footnotesize
\begin{verbatim}
GrB_Info GxB_sort
(
    // output:
    GrB_Matrix C,           // matrix of sorted values
    GrB_Matrix P,           // matrix containing the permutations
    // input
    GrB_BinaryOp op,        // comparator op
    GrB_Matrix A,           // matrix to sort
    const GrB_Descriptor desc
) ;
\end{verbatim}
} \end{mdframed}

\verb'GxB_Matrix_sort' sorts all the rows or all the columns of a matrix.
Each row (or column) is sorted separately.  The rows are sorted by default.
To sort the columns, use \verb'GrB_DESC_T0'.  A comparator operator is
provided to define the sorting order (ascending or descending).
For example, to sort a \verb'GrB_FP64' matrix in ascending order,
use \verb'GrB_LT_FP64' as the \verb'op', and to sort in descending order,
use \verb'GrB_GT_FP64'.

The \verb'op' must have a return value of \verb'GrB_BOOL', and the types of
its two inputs must be the same.  The entries in \verb'A' are typecasted to
the inputs of the \verb'op', if necessary.  Matrices with user-defined types
can be sorted with a user-defined comparator operator, whose two input types
must match the type of \verb'A', and whose output is \verb'GrB_BOOL'.

The two matrix outputs are \verb'C' and \verb'P'.  Any entries present on input
in \verb'C' or \verb'P' are discarded on output.  The type of \verb'C' must
match the type of \verb'A' exactly.  The dimensions of \verb'C', \verb'P', and
\verb'A' must also match exactly (even with the \verb'GrB_DESC_T0'
descriptor).

With the default sort (by row), suppose \verb'A(i,:)' contains \verb'k'
entries.  In this case, \verb'C(i,0:k-1)' contains the values of those entries
in sorted order, and \verb'P(i,0:k-1)' contains their corresponding column
indices in the matrix \verb'A'.  If two values are the same, ties are broken
according column index.

If the matrix is sorted by column, and \verb'A(:,j)' contains \verb'k' entries,
then \verb'C(0:k-1,j)' contains the values of those entries in sorted order,
and \verb'P(0:k-1,j)' contains their corresponding row indices in the matrix
\verb'A'.  If two values are the same, ties are broken according row index.

The outputs \verb'C' and \verb'P' are both optional; either one (but not both)
may be \verb'NULL', in which case that particular output matrix is not
computed.
%
The \verb'op' cannot be a binary operator
created by \verb'GxB_BinaryOp_new_IndexOp'.




\newpage
%===============================================================================
\subsection{GraphBLAS descriptors: {\sf GrB\_Descriptor}} %=====================
%===============================================================================
\label{descriptor}

A GraphBLAS {\em descriptor} modifies the behavior of a GraphBLAS operation.
If the descriptor is \verb'GrB_NULL', defaults are used.

The access to these parameters and their values is governed
by two \verb'enum' types, \verb'GrB_Desc_Field' and \verb'GrB_Desc_Value':

\begin{mdframed}[userdefinedwidth=6in]
{\footnotesize
\begin{verbatim}
typedef enum
{
    GrB_OUTP = 0,   // descriptor for output of a method
    GrB_MASK = 1,   // descriptor for the mask input of a method
    GrB_INP0 = 2,   // descriptor for the first input of a method
    GrB_INP1 = 3,   // descriptor for the second input of a method
    GxB_AxB_METHOD = 1000, // descriptor for selecting C=A*B algorithm
    GxB_SORT = 35   // control sort in GrB_mxm
    GxB_COMPRESSION = 36,   // select compression for serialize
    GxB_ROWINDEX_LIST = 7062,       // how GrB_Vector I is intrepretted
    GxB_COLINDEX_LIST = 7063,       // how GrB_Vector J is intrepretted
    GxB_VALUE_LIST = 7064,          // how GrB_Vector X is intrepretted
}
GrB_Desc_Field ;

typedef enum
{
    // for all GrB_Descriptor fields:
    GrB_DEFAULT = 0,    // default behavior of the method
    // for GrB_OUTP only:
    GrB_REPLACE = 1,    // clear the output before assigning new values to it
    // for GrB_MASK only:
    GrB_COMP = 2,       // use the complement of the mask
    GrB_STRUCTURE = 4,  // use the structure of the mask
    // for GrB_INP0 and GrB_INP1 only:
    GrB_TRAN = 3,       // use the transpose of the input
    // for GxB_AxB_METHOD only:
    GxB_AxB_GUSTAVSON = 1001,   // gather-scatter saxpy method
    GxB_AxB_DOT       = 1003,   // dot product
    GxB_AxB_HASH      = 1004,   // hash-based saxpy method
    GxB_AxB_SAXPY     = 1005    // saxpy method (any kind)
    // for GxB_ROWINDEX_LIST, GxB_COLINDEX_LIST, and GxB_VALUE_LIST:
    // GxB_USE_VALUES = ((int) GrB_DEFAULT) // use the values of the vector
    GxB_USE_INDICES = 7060,  // use the indices of the vector
    GxB_IS_STRIDE = 7061,    // use the values, of size 3, for lo:hi:inc
}
GrB_Desc_Value ;
\end{verbatim} } \end{mdframed}

\newpage

\begin{itemize}
\item \verb'GrB_OUTP' is a parameter that modifies the output of a
    GraphBLAS operation.  In the default case, the output is not cleared, and
    ${\bf Z = C \odot T}$ then ${\bf C \langle M \rangle = Z}$ are computed
    as-is, where ${\bf T}$ is the results of the particular GraphBLAS
    operation.

    In the non-default case, ${\bf Z = C \odot T}$ is first computed, using the
    results of ${\bf T}$ and the accumulator $\odot$.  After this is done, if
    the \verb'GrB_OUTP' descriptor field is set to \verb'GrB_REPLACE', then the
    output is cleared of its entries.  Next, the assignment ${\bf C \langle M
    \rangle = Z}$ is performed.

\item \verb'GrB_MASK' is a parameter that modifies the \verb'Mask',
    even if the mask is not present.

    If this parameter is set to its default value, and if the mask is not
    present (\verb'Mask==NULL') then implicitly \verb'Mask(i,j)=1' for all
    \verb'i' and \verb'j'.  If the mask is present then \verb'Mask(i,j)=1'
    means that \verb'C(i,j)' is to be modified by the ${\bf C \langle M \rangle
    = Z}$ update.  Otherwise, if \verb'Mask(i,j)=0', then \verb'C(i,j)' is not
    modified, even if \verb'Z(i,j)' is an entry with a different value; that
    value is simply discarded.

    If the \verb'GrB_MASK' parameter is set to \verb'GrB_COMP', then the
    use of the mask is complemented.  In this case, if the mask is not present
    (\verb'Mask==NULL') then implicitly \verb'Mask(i,j)=0' for all \verb'i' and
    \verb'j'.  This means that none of ${\bf C}$ is modified and the entire
    computation of ${\bf Z}$ might as well have been skipped.  That is, a
    complemented empty mask means no modifications are made to the output
    object at all, except perhaps to clear it in accordance with the
    \verb'GrB_OUTP' descriptor.  With a complemented mask, if the mask is
    present then \verb'Mask(i,j)=0' means that \verb'C(i,j)' is to be modified
    by the ${\bf C \langle M \rangle = Z}$ update.  Otherwise, if
    \verb'Mask(i,j)=1', then \verb'C(i,j)' is not modified, even if
    \verb'Z(i,j)' is an entry with a different value; that value is simply
    discarded.

    If the \verb'GrB_MASK' parameter is set to \verb'GrB_STRUCTURE',
    then the values of the mask are ignored, and just the pattern of the
    entries is used.  Any entry \verb'M(i,j)' in the pattern is treated as if
    it were true.

    The \verb'GrB_COMP' and \verb'GrB_STRUCTURE' settings can be combined,
    either by setting the mask option twice (once with each value), or by
    setting the mask option to \verb'GrB_COMP+GrB_STRUCTURE' (the latter is an
    extension to the specification).

    Using a parameter to complement the \verb'Mask' is very useful because
    constructing the actual complement of a very sparse mask is impossible
    since it has too many entries.  If the number of places in \verb'C'
    that should be modified is very small, then use a sparse mask without
    complementing it.  If the number of places in \verb'C' that should
    be protected from modification is very small, then use a sparse mask
    to indicate those places, and use a descriptor \verb'GrB_MASK' that
    complements the use of the mask.

\item \verb'GrB_INP0' and \verb'GrB_INP1' modify the use of the
    first and second input matrices \verb'A' and \verb'B' of the GraphBLAS
    operation.

    If the \verb'GrB_INP0' is set to \verb'GrB_TRAN', then \verb'A' is
    transposed before using it in the operation.  Likewise, if
    \verb'GrB_INP1' is set to \verb'GrB_TRAN', then the second input,
    typically called \verb'B', is transposed.

    Vectors and scalars are never transposed via the descriptor.  If a method's
    first parameter is a matrix and the second a vector or scalar, then
    \verb'GrB_INP0' modifies the matrix parameter and
    \verb'GrB_INP1' is ignored.  If a method's first parameter is a
    vector or scalar and the second a matrix, then \verb'GrB_INP1'
    modifies the matrix parameter and \verb'GrB_INP0' is ignored.

    To clarify this in each function, the inputs are labeled as
    \verb'first input:' and \verb'second input:' in the function signatures.

\item \verb'GxB_AxB_METHOD' suggests the method that should be
    used to compute \verb'C=A*B'.  All the methods compute the same result,
    except they may have different floating-point roundoff errors.  This
    descriptor should be considered as a hint; SuiteSparse:GraphBLAS is
    free to ignore it.

    \begin{itemize}

    \item \verb'GrB_DEFAULT' means that a method is selected automatically.

    \item \verb'GxB_AxB_SAXPY': select any saxpy-based method:
        \verb'GxB_AxB_GUSTAVSON', and/or
        \verb'GxB_AxB_HASH', or any mix of the two,
        in contrast to the dot-product method.

    \item \verb'GxB_AxB_GUSTAVSON':  an extended version of Gustavson's method
    \cite{Gustavson78}, which is a very good general-purpose method, but
    sometimes the workspace can be too large.  Assuming all matrices are stored
    by column, it computes \verb'C(:,j)=A*B(:,j)' with a sequence of {\em
    saxpy} operations (\verb'C(:,j)+=A(:,k)*B(k:,j)' for each nonzero
    \verb'B(k,j)').  In the {\em coarse Gustavson} method, each internal thread
    requires workspace of size $m$, to the number of rows of \verb'C', which is
    not suitable if the matrices are extremely sparse or if there are many
    threads.  For the {\em fine Gustavson} method, threads can share workspace
    and update it via atomic operations.  If all matrices are stored by row,
    then it computes \verb'C(i,:)=A(i,:)*B' in a sequence of sparse {\em saxpy}
    operations, and using workspace of size $n$ per thread, or group of
    threads, corresponding to the number of columns of \verb'C'.

    \item \verb'GxB_AxB_HASH':  a hash-based method, based on
        \cite{10.1145/3229710.3229720}.  It is very efficient for hypersparse
        matrices, matrix-vector-multiply, and when $|{\bf B}|$ is small.
        SuiteSparse:GraphBLAS includes a {\em coarse hash} method, in which
        each thread has its own hash workspace, and a {\em fine hash}
        method, in which groups of threads share a single hash workspace,
        as concurrent data structure, using atomics.

% [2] Yusuke Nagasaka, Satoshi Matsuoka, Ariful Azad, and Aydin Buluc. 2018.
% High-Performance Sparse Matrix-Matrix Products on Intel KNL and Multicore
% Architectures. In Proc. 47th Intl. Conf. on Parallel Processing (ICPP '18).
% Association for Computing Machinery, New York, NY, USA, Article 34, 1–10.
% DOI:https://doi.org/10.1145/3229710.3229720

\item \verb'GxB_AxB_DOT': computes \verb"C(i,j)=A(i,:)*B(j,:)'", for each
    entry \verb'C(i,j)'.  If the mask is present and not complemented, only
    entries for which \verb'M(i,j)=1' are computed.  This is a very specialized
    method that works well only if the mask is present, very sparse, and not
    complemented, when \verb'C' is small, or when \verb'C' is bitmap or full.
    For example, it works very well
    when \verb'A' and \verb'B' are tall and thin, and \verb"C<M>=A*B'" or
    \verb"C=A*B'" are computed.  These expressions assume all matrices are in
    CSR format.  If in CSC format, then the dot-product method used for
    \verb"A'*B".  The method is impossibly slow if \verb'C' is large and the
    mask is not present, since it takes $\Omega(mn)$ time if \verb'C' is
    $m$-by-$n$ in that case.  It does not use any workspace at all.  Since it
    uses no workspace, it can work very well for extremely sparse or
    hypersparse matrices, when the mask is present and not complemented.

    \end{itemize}

\item \verb'GxB_SORT' provides a hint to \verb'GrB_mxm', \verb'GrB_mxv',
    \verb'GrB_vxm', and \verb'GrB_reduce' (to vector).  These methods can leave
    the output matrix or vector in a jumbled state, where the final sort is
    left as pending work.  This is typically fastest, since some algorithms can
    tolerate jumbled matrices on input, and sometimes the sort can be skipped
    entirely.  However, if the matrix or vector will be immediately exported in
    unjumbled form, or provided as input to a method that requires it to not be
    jumbled, then sorting it during the matrix multiplication is faster.
    By default, these methods leave the result in jumbled form (a {\em lazy
    sort}), if \verb'GxB_SORT' is set to zero (\verb'GrB_DEFAULT').  A nonzero
    value will inform the matrix multiplication to sort its result, instead.

\item \verb'GxB_COMPRESSION' selects the compression method for serialization.
    The default is ZSTD (level 1).  See Section~\ref{serialize_deserialize} for
    other options.

\end{itemize}

The next sections describe the methods for a \verb'GrB_Descriptor':

\vspace{0.2in}
{\footnotesize
\begin{tabular}{lll}
\hline
GraphBLAS function          & purpose                              & Section \\
\hline
\verb'GrB_Descriptor_new'   & create a descriptor                  & \ref{descriptor_new} \\
\verb'GrB_Descriptor_wait'  & wait for a descriptor                & \ref{descriptor_wait} \\
\verb'GrB_Descriptor_free'  & free a descriptor                    & \ref{descriptor_free} \\
\verb'GrB_get'              & get a parameter from a descriptor    & \ref{get_set_descriptor}  \\
\verb'GrB_set'              & set a parameter in a descriptor      & \ref{get_set_descriptor}  \\
\hline
\end{tabular}
}

%-------------------------------------------------------------------------------
\subsubsection{{\sf GrB\_Descriptor\_new:}  create a new descriptor}
%-------------------------------------------------------------------------------
\label{descriptor_new}

\begin{mdframed}[userdefinedwidth=6in]
{\footnotesize
\begin{verbatim}
GrB_Info GrB_Descriptor_new     // create a new descriptor
(
    GrB_Descriptor *descriptor  // handle of descriptor to create
) ;
\end{verbatim} } \end{mdframed}

\verb'GrB_Descriptor_new' creates a new descriptor, with all fields set to
their defaults (output is not replaced, the mask is not complemented, the mask
is valued not structural, neither input matrix is transposed, the method
used in \verb'C=A*B' is selected automatically, and \verb'GrB_mxm' leaves
the final sort as pending work).

%-------------------------------------------------------------------------------
\subsubsection{{\sf GrB\_Descriptor\_wait:} wait for a descriptor}
%-------------------------------------------------------------------------------
\label{descriptor_wait}

\begin{mdframed}[userdefinedwidth=6in]
{\footnotesize
\begin{verbatim}
GrB_Info GrB_wait                   // wait for a descriptor
(
    GrB_Descriptor descriptor,      // descriptor to wait for
    int mode                        // GrB_COMPLETE or GrB_MATERIALIZE
) ;
\end{verbatim}
}\end{mdframed}

After creating a user-defined descriptor, a GraphBLAS library may choose to
exploit non-blocking mode to delay its creation.  Currently,
SuiteSparse:GraphBLAS does nothing except to ensure that \verb'd' is valid.

%-------------------------------------------------------------------------------
\subsubsection{{\sf GrB\_Descriptor\_free:} free a descriptor}
%-------------------------------------------------------------------------------
\label{descriptor_free}

\begin{mdframed}[userdefinedwidth=6in]
{\footnotesize
\begin{verbatim}
GrB_Info GrB_free               // free a descriptor
(
    GrB_Descriptor *descriptor  // handle of descriptor to free
) ;
\end{verbatim} } \end{mdframed}

\verb'GrB_Descriptor_free' frees a descriptor.
Either usage:

    {\small
    \begin{verbatim}
    GrB_Descriptor_free (&descriptor) ;
    GrB_free (&descriptor) ; \end{verbatim}}

\noindent
frees the \verb'descriptor' and sets \verb'descriptor' to \verb'NULL'.  It
safely does nothing if passed a \verb'NULL' handle, or if
\verb'descriptor == NULL' on input.

%-------------------------------------------------------------------------------
\subsubsection{Descriptor settings for \sf{GrB\_Vector} parameters}
%-------------------------------------------------------------------------------
\label{ijxvector}

Several methods GraphBLAS v10 accept \verb'GrB_Vector' parameters for their index
lists \verb'I' and \verb'J', which appear only as \verb'uint64_t *' C arrays in
the v2.1 C Specification.  Likewise, several methods accept a \verb'GrB_Vector'
parameter \verb'X', where the related method in the Specification accepts only
a raw C array of a given type.

By default, \verb'GrB_Vector' inputs \verb'I', \verb'J', and \verb'X' are
interpretted as if their values are first extracted with
\verb'GrB_Vector_extractTuples', where the values are extracted in order (with
ascending indices), and their values are then passed to the method.
The actual method is much faster; GraphBLAS uses the values directly.

This behavior can be revised via the descriptor for the method.  Three settings
are available:

\begin{itemize}
\item \verb'GxB_ROWINDEX_LIST': how the \verb'GrB_Vector I' is intrepretted.
\item \verb'GxB_COLINDEX_LIST': how the \verb'GrB_Vector J' is intrepretted.
\item \verb'GxB_VALUE_LIST': how \verb'GrB_Vector X' is intrepretted (for \verb'GrB_build' only).
\end{itemize}

These can be set to one of the following values:

\begin{itemize}
\item \verb'GrB_DEFAULT' or \verb'GxB_USE_VALUES': use the values of the vector (default).

\item \verb'GxB_USE_INDICES': use the indices of the vector.
    This acts as if the indices are first extracted into a C array with
    \verb'GrB_Vector_extractTuples', where the indices are extracted in ascending order,
    and then this C array is then passed to the method.
    The actual method is much faster; GraphBLAS uses the indices directly.

\item \verb'GxB_IS_STRIDE': use the values, of size 3, for a strided range,
    or \verb'lo:inc:hi' in MATLAB notation.  This usage is limited to the
    \verb'I' and \verb'J' vectors (except this option may not be used for
    \verb'GrB_build').  The vector must have exactly three entries, 
    \verb'lo', \verb'hi', and \verb'inc', in that order.

\end{itemize}

The \verb'GxB_IS_STRIDE' option is fully described in Section~\ref{colon}.  In
that section, there are many options available.  Here, the \verb'GrB_Vector'
\verb'I' or \verb'J' must have length exactly three.  The first entry present
is the start of the sequence (\verb'lo'), the second entry is the end of the
sequence (\verb'hi') and the third entry is the stride (\verb'inc').  This
corresponds to the \verb'GxB_STRIDE' option when passing a \verb'uint64_t *'
array.  To use a stride of one, simply set the third entry to 1; this
corresponds to the \verb'GxB_RANGE' option when passing a \verb'uint64_t *'
array.  To use a negative stride, simply pass in the vector with a signed data
type (\verb'GrB_INT32' or \verb'GrB_INT64' as appropriate; this corresponds to
the \verb'GxB_BACKWARDS' option desribed in Section~\ref{colon}).
These three values appear in this order to be consistent \verb'GxB_BEGIN' (0),
\verb'GxB_END' (1), and \verb'GxB_INC' (2).

When using the \verb'_Vector' methods, the \verb'GrB_Vector' objects \verb'I',
\verb'J', and \verb'X' may be sparse.  If the vectors are sparse,
\verb'GrB_Vector_extractTuples' returns a dense list of indices or values, and
this is how the \verb'I,J,X' vectors may be used in the new methods in
GraphBLAS v10 with the \verb'_Vector' suffix at then end of their name.

To use the \verb'GrB_ALL' option, specifying all the rows or columns of a
matrix or all indices of a vector, pass in the corresponding \verb'GrB_Vector'
\verb'I' or \verb'J' as a NULL pointer.


\newpage
%-------------------------------------------------------------------------------
\subsubsection{{\sf GrB\_DESC\_*:}  built-in descriptors}
%-------------------------------------------------------------------------------
\label{descriptor_predefined}

Built-in descriptors are listed in the table below.  A dash in the table
indicates the default.  These descriptors may not be modified or freed.
Attempts to modify them result in an error (\verb'GrB_INVALID_VALUE'); attempts
to free them are silently ignored.

% \verb'GrB_NULL' is the default descriptor, with all settings at their defaults:
% \verb'OUTP': do not replace the output,
% \verb'MASK': mask is valued and not complemented,
% \verb'INP0': first input not transposed, and
% \verb'INP1': second input not transposed.
% For these pre-defined descriptors, the
% \verb'GxB_SORT' setting is at their default values.

\vspace{0.2in}
\noindent
{\footnotesize
\begin{tabular}{|l|lllll|}
\hline
Descriptor              &  \verb'OUTP'          & \verb'MASK'           & \verb'MASK'       & \verb'INP0'       & \verb'INP1'       \\
                        &                       & structural            & complement        & & \\
\hline
\verb'GrB_NULL'         &   -                   & -                     & -                 & -                 & -                 \\
\verb'GrB_DESC_T1'      &   -                   & -                     & -                 & -                 & \verb'GrB_TRAN'   \\
\verb'GrB_DESC_T0'      &   -                   & -                     & -                 & \verb'GrB_TRAN'   & -                 \\
\verb'GrB_DESC_T0T1'    &   -                   & -                     & -                 & \verb'GrB_TRAN'   & \verb'GrB_TRAN'   \\
\hline
\verb'GrB_DESC_C'       &   -                   & -                     & \verb'GrB_COMP'   & -                 & -                 \\
\verb'GrB_DESC_CT1'     &   -                   & -                     & \verb'GrB_COMP'   & -                 & \verb'GrB_TRAN'   \\
\verb'GrB_DESC_CT0'     &   -                   & -                     & \verb'GrB_COMP'   & \verb'GrB_TRAN'   & -                 \\
\verb'GrB_DESC_CT0T1'   &   -                   & -                     & \verb'GrB_COMP'   & \verb'GrB_TRAN'   & \verb'GrB_TRAN'   \\
\hline
\verb'GrB_DESC_S'       &   -                   & \verb'GrB_STRUCTURE'  & -                 & -                 & -                 \\
\verb'GrB_DESC_ST1'     &   -                   & \verb'GrB_STRUCTURE'  & -                 & -                 & \verb'GrB_TRAN'   \\
\verb'GrB_DESC_ST0'     &   -                   & \verb'GrB_STRUCTURE'  & -                 & \verb'GrB_TRAN'   & -                 \\
\verb'GrB_DESC_ST0T1'   &   -                   & \verb'GrB_STRUCTURE'  & -                 & \verb'GrB_TRAN'   & \verb'GrB_TRAN'   \\
\hline
\verb'GrB_DESC_SC'      &   -                   & \verb'GrB_STRUCTURE'  & \verb'GrB_COMP'   & -                 & -                 \\
\verb'GrB_DESC_SCT1'    &   -                   & \verb'GrB_STRUCTURE'  & \verb'GrB_COMP'   & -                 & \verb'GrB_TRAN'   \\
\verb'GrB_DESC_SCT0'    &   -                   & \verb'GrB_STRUCTURE'  & \verb'GrB_COMP'   & \verb'GrB_TRAN'   & -                 \\
\verb'GrB_DESC_SCT0T1'  &   -                   & \verb'GrB_STRUCTURE'  & \verb'GrB_COMP'   & \verb'GrB_TRAN'   & \verb'GrB_TRAN'   \\
\hline
\verb'GrB_DESC_R'       &   \verb'GrB_REPLACE'  & -                     & -                 & -                 & -                 \\
\verb'GrB_DESC_RT1'     &   \verb'GrB_REPLACE'  & -                     & -                 & -                 & \verb'GrB_TRAN'   \\
\verb'GrB_DESC_RT0'     &   \verb'GrB_REPLACE'  & -                     & -                 & \verb'GrB_TRAN'   & -                 \\
\verb'GrB_DESC_RT0T1'   &   \verb'GrB_REPLACE'  & -                     & -                 & \verb'GrB_TRAN'   & \verb'GrB_TRAN'   \\
\hline
\verb'GrB_DESC_RC'      &   \verb'GrB_REPLACE'  & -                     & \verb'GrB_COMP'   & -                 & -                 \\
\verb'GrB_DESC_RCT1'    &   \verb'GrB_REPLACE'  & -                     & \verb'GrB_COMP'   & -                 & \verb'GrB_TRAN'   \\
\verb'GrB_DESC_RCT0'    &   \verb'GrB_REPLACE'  & -                     & \verb'GrB_COMP'   & \verb'GrB_TRAN'   & -                 \\
\verb'GrB_DESC_RCT0T1'  &   \verb'GrB_REPLACE'  & -                     & \verb'GrB_COMP'   & \verb'GrB_TRAN'   & \verb'GrB_TRAN'   \\
\hline
\verb'GrB_DESC_RS'      &   \verb'GrB_REPLACE'  & \verb'GrB_STRUCTURE'  & -                 & -                 & -                 \\
\verb'GrB_DESC_RST1'    &   \verb'GrB_REPLACE'  & \verb'GrB_STRUCTURE'  & -                 & -                 & \verb'GrB_TRAN'   \\
\verb'GrB_DESC_RST0'    &   \verb'GrB_REPLACE'  & \verb'GrB_STRUCTURE'  & -                 & \verb'GrB_TRAN'   & -                 \\
\verb'GrB_DESC_RST0T1'  &   \verb'GrB_REPLACE'  & \verb'GrB_STRUCTURE'  & -                 & \verb'GrB_TRAN'   & \verb'GrB_TRAN'   \\
\hline
\verb'GrB_DESC_RSC'     &   \verb'GrB_REPLACE'  & \verb'GrB_STRUCTURE'  & \verb'GrB_COMP'   & -                 & -                 \\
\verb'GrB_DESC_RSCT1'   &   \verb'GrB_REPLACE'  & \verb'GrB_STRUCTURE'  & \verb'GrB_COMP'   & -                 & \verb'GrB_TRAN'   \\
\verb'GrB_DESC_RSCT0'   &   \verb'GrB_REPLACE'  & \verb'GrB_STRUCTURE'  & \verb'GrB_COMP'   & \verb'GrB_TRAN'   & -                 \\
\verb'GrB_DESC_RSCT0T1' &   \verb'GrB_REPLACE'  & \verb'GrB_STRUCTURE'  & \verb'GrB_COMP'   & \verb'GrB_TRAN'   & \verb'GrB_TRAN'   \\
\hline
\end{tabular}}



\newpage
%===============================================================================
\subsection{{\sf GrB\_free:} free any GraphBLAS object} %=======================
%===============================================================================
\label{free}

Each of the ten objects has \verb'GrB_*_new' and \verb'GrB_*_free' methods
that are specific to each object.  They can also be accessed by a generic
function, \verb'GrB_free', that works for all ten objects.  If \verb'G' is any
of the ten objects, the statement

    {\footnotesize
    \begin{verbatim}
    GrB_free (&G) ; \end{verbatim} }

\noindent
frees the object and sets the variable \verb'G' to \verb'NULL'.  It is safe to
pass in a \verb'NULL' handle, or to free an object twice:

    {\footnotesize
    \begin{verbatim}
    GrB_free (NULL) ;       // SuiteSparse:GraphBLAS safely does nothing
    GrB_free (&G) ;         // the object G is freed and G set to NULL
    GrB_free (&G) ;         // SuiteSparse:GraphBLAS safely does nothing \end{verbatim} }

\noindent
However, the following sequence of operations is not safe.  The first two are
valid but the last statement will lead to undefined behavior.

    {\footnotesize
    \begin{verbatim}
    H = G ;                // valid; creates a 2nd handle of the same object
    GrB_free (&G) ;        // valid; G is freed and set to NULL; H now undefined
    GrB_some_method (H) ;  // not valid; H is undefined \end{verbatim}}

Some objects are predefined, such as the built-in types.  If a user application
attempts to free a built-in object, SuiteSparse:GraphBLAS will safely do
nothing.  The \verb'GrB_free' function in SuiteSparse:GraphBLAS always
returns \verb'GrB_SUCCESS'.



